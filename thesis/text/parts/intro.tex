\startprefacepage

В микроэлектронике для изготовления интегральных схем в основном используются полевые транзисторы (англ. MOSFET). Полевой транзистор — прибор, в простейшем случае состоящий из трех контактов:
\begin{easylist}[itemize]
# исток — контакт, на который подаются носители заряда
# сток — контакт, с которого уходят носители заряда
# затвор — контакт, напряжением на котором можно регулировать ток, идущий от истока к стоку
\end{easylist}
В некотором смысле, транзистор является «управляемым сопротивлением», то есть прибором, проводимость которого можно контроллировать напряжением на затворе, что и позволяет использовать его для реализации логических элементов.

Возникает естественная необходимость в уменьшении размеров транзисторов, так как чем меньше транзистор, тем:
\begin{easylist}[itemize]
# выше производительность
## скорость распространения сигнала в проводнике конечна, и чем ближе расположены транзисторы друг к другу в интегральной схеме, тем быстрее сигнал проходит через эту схему
## транзистор быстрее реагирует на изменение тока затвора, и быстрее переключает свое состояние проводимости
# меньше тепловыделение, так как затвор работает на меньшем напряжении
# более эффективно используется полупроводниковая заготовка:
## можно более плотно упаковать транзисторы
## можно напечатать больше транзисторов за один прогон станка
\end{easylist}

Уже около 50 лет в микроэлектроние действительно наблюдается экспоненциальный рост количества транзисторов на единицу площади, которую называют законом Мура (\todo{ссылку}). В настоящее время Intel изготавливает транзисторы размером 22 нм (\todo{ссылку}), в 2014 ожидается 14 нм (\todo{ссылку}).

Однако, на размерах около 10 нм становятся заметными квантомеханические эффекты. В частности, рассмотрим проблему тока утечки через затвор, возникающего за счет квантового туннелирования.

Когда на затвор транзистора не подается напряжение, через транзистор не должен идти ток. В полупроводниковых транзисторах это достигается засчет того, что в отстутствие напряжения затвора присутствует потенциальный барьер. Вообще говоря, всегда есть ненулевая вероятность туннелирования электронов через потенциальный барьер (и, соответственно, ненулевой ток утечки), которая в простейшем приближении \todo{зафигачить сюда формулу} зависит от:
\begin{easylist}[itemize]
# ширины потенциального барьера
# высоты потенциального барьера
# эффективной массы переносчиков заряда % https://ru.wikipedia.org/wiki/%D0%AD%D1%84%D1%84%D0%B5%D0%BA%D1%82%D0%B8%D0%B2%D0%BD%D0%B0%D1%8F_%D0%BC%D0%B0%D1%81%D1%81%D0%B0
\end{easylist}

Соответственно, уменьшать вероятность туннелирования можно следующим образом:

\begin{easylist}[itemize]
# увеличить ширину потенциального барьера; это непосредственно противоречит уменьшению размера транзистора
# увеличить высоту потенциального барьера; \todo{???}
# увеличить эффективную массу носителей заряда; это влечет за собой уменьшение мобильности носителей заряда и, соответственно, скорости переключения состояния транзистора
\end{easylist}

В связи с данной проблемой происходит поиск альтернативных способов реализации транзисторов (то есть, приборов с «управляемым сопротивлением»), использующие резонансные квантовомеханические эффекты для реализации нелинейности в проводимости. Как правило, они представляют их себя квантовый волновод с барьерами или резонаторами, параметры которых можно контроллировать, например, внешним электрическим полем. \todo{что-то про квантовые точки} Изменение параметров резонатора влечет за собой изменение коэффициента проводимости, и, соответственно, проводимости транзистора.

К сожалению, исследовать аналитически такие квантомеханические системы очень сложно. \todo{Кстати, а как полевые транзисторы исследуют?} Уже в простейшей модели одномерной прямоугольной квантовой ямы (англ. 1D finite potential well) для расчета собственных энергий и функций, необходимых для расчета коэффициента проводимости, нужно решать трансцедентные уравнения. Естественно, в двумерном и трехмерном случаях, уравнения решать тем более сложнее.

В рамках решения этой проблемы разработан подход, позволяющий изменить исходную модель, получив ее аппроксимацию, которая допускает аналитическое решение. \todo{ссылку} В работах \todo{ссылку} этот подход был применен для граничного условия Неймана. Однако, физически обоснованным граничным условием является граничное условие Дирихле, которое еще не было исследовано в рамках этого подхода.

Целью данной работы является исследование различных конфигураций квантовых волноводов с граничным условием Дирихле, получение их аналитических решений методом аппроксимации моделью нулевого радиуса и расчет характеристик проводимости этих волноводов.


% https://en.wikipedia.org/wiki/Metal_oxide_semiconductor_field_effect_transistor


% It is also expected that smaller transistors switch faster. For example, one approach to size reduction is a scaling of the MOSFET that requires all device dimensions to reduce proportionally. The main device dimensions are the channel length, channel width, and oxide thickness. When they are scaled down by equal factors, the transistor channel resistance does not change, while gate capacitance is cut by that factor. Hence, the RC delay of the transistor scales with a similar factor.

% While this has been traditionally the case for the older technologies, for the state-of-the-art MOSFETs reduction of the transistor dimensions does not necessarily translate to higher chip speed because the delay due to interconnections is more significant.


