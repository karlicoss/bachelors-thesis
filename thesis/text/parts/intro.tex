\startprefacepage

В микроэлектронике для изготовления интегральных схем в основном используются полевые транзисторы. Полевой транзистор — прибор, в простейшем случае состоящий из трех контактов:
\begin{easylist}[itemize]
# исток — контакт, на который подаются носители заряда;
# сток — контакт, с которого уходят носители заряда;
# затвор — контакт, напряжением на котором можно регулировать ток, идущий от истока к стоку.
\end{easylist}
Фактически, транзистор является «управляемым сопротивлением», то есть прибором, проводимость которого можно контроллировать напряжением на затворе, что и позволяет использовать его для реализации логических элементов.

Возникает естественная необходимость в уменьшении размеров транзисторов, так как чем меньше транзистор, тем:
\begin{easylist}[itemize]
# выше производительность:
## скорость распространения сигнала в проводнике конечна, и чем ближе расположены транзисторы друг к другу в интегральной схеме, тем быстрее сигнал проходит через эту схему;
## транзистор быстрее реагирует на изменение тока затвора, и быстрее переключает свое состояние проводимости.
# меньше тепловыделение, так как затвор работает на меньшем напряжении;
# более эффективно используется полупроводниковая заготовка:
## можно более плотно упаковать транзисторы;
## можно напечатать больше транзисторов за один прогон станка.
\end{easylist}

Уже около 50 лет в микроэлектроние действительно наблюдается экспоненциальный рост количества транзисторов на единицу площади, которую называют законом Мура. В настоящее время Intel изготавливает транзисторы размером 22 нм \cite{intel_22_nm}, в 2014 ожидается 14 нм \cite{14_nm}. Однако, на размерах около 10 нм становятся заметными квантомеханические эффекты, что может приводить к трудностям в проектировании транзисторов. В частности, рассмотрим проблему тока утечки через затвор, возникающего за счет квантового туннелирования.

Когда на затвор транзистора не подается напряжение, через транзистор не должен идти ток. В полупроводниковых транзисторах это достигается засчет того, что в отстутствие напряжения затвора присутствует потенциальный барьер. Вообще говоря, всегда есть ненулевая вероятность туннелирования электронов через потенциальный барьер (и, соответственно, ненулевой ток утечки), которая в простейшем приближении зависит от:
\begin{easylist}[itemize]
# ширины потенциального барьера;
# высоты потенциального барьера;
# эффективной массы переносчиков заряда.
\end{easylist}

Соответственно, уменьшать вероятность туннелирования можно следующим образом:

\begin{easylist}[itemize]
# увеличить ширину потенциального барьера, но это непосредственно противоречит уменьшению размера транзистора;
# увеличить высоту потенциального барьера, но это влечет за собой увеличение напряжения затвора и потери тепла;
# увеличить эффективную массу носителей заряда, но это влечет за собой уменьшение мобильности носителей заряда и, соответственно, скорости переключения состояния транзистора.
\end{easylist}

В связи с данной проблемой происходит поиск альтернативных способов реализации транзисторов (то есть, приборов с «управляемым сопротивлением»). В данной работе будет предложена модель транзистора, использующие резонансные квантовомеханические эффекты для реализации нелинейности в проводимости. Она представляет из себя квантовый волновод с резонатором, геометрию которого можно контроллировать, например, внешним электрическим полем. Физически такие резонаторы могут быть реализованы на основе квантовых точек (англ. quantum dots), для которых в настоящее время существуют технологии, позволяющие сделать их размер в 2-10 нм. Изменение геометрии резонатора влечет за собой изменение коэффициента проводимости, и, соответственно, проводимости транзистора.

К сожалению, исследовать аналитически квантомеханические системы очень сложно, к примеру, уже в простейшей модели одномерной прямоугольной квантовой ямы (англ. 1D finite potential well) для расчета собственных энергий и функций, необходимых для расчета коэффициента проводимости, нужно решать трансцедентные уравнения. Естественно, в двумерном и трехмерном случаях, уравнения решать тем более сложнее.

В рамках решения этой проблемы разработан подход, позволяющий изменить исходную модель, получив ее аппроксимацию, которая допускает аналитическое решение. В работах \cite{popov1992extension, popov1992resonator, popov1993zero} этот подход был применен для граничного условия Неймана. Однако, физически обоснованным граничным условием является граничное условие Дирихле, которое еще не было исследовано в рамках этого подхода.

Целью данной работы является исследование квантового волновода с граничным условием Дирихле, аналитического решения уравнения Шредингера для этого волновода методом аппроксимации моделью нулевого радиуса и расчет его спектральных характеристик.