\startprefacepage

В микроэлектронике уже около 50 лет наблюдается естественная тенденция к уменьшению размеров интегральных схем, и, соответственно, размеров их элементарных составляющих (таких как транзисторов, диодов и т.п.), которую часто называют законом Мура (TODO ссылку). В настоящее время существуют технологиии изготовления наноэлектронных структур (размером от нескольких до нескольких десятков нанометров), что яв

КМОП (комплементарная структура металл-оксид-полупроводник, англ. CMOS) -- технология построения микроэлектронных схем

MOSFET

% https://en.wikipedia.org/wiki/Metal_oxide_semiconductor_field_effect_transistor


It is also expected that smaller transistors switch faster. For example, one approach to size reduction is a scaling of the MOSFET that requires all device dimensions to reduce proportionally. The main device dimensions are the channel length, channel width, and oxide thickness. When they are scaled down by equal factors, the transistor channel resistance does not change, while gate capacitance is cut by that factor. Hence, the RC delay of the transistor scales with a similar factor.

While this has been traditionally the case for the older technologies, for the state-of-the-art MOSFETs reduction of the transistor dimensions does not necessarily translate to higher chip speed because the delay due to interconnections is more significant.


TODO написать про кубиты?


При таких маленьких размерах полупроводниковых транзисторов, приходится принимать во внимание квантовые эффекты, так как электроны способны туннелировать через потенциальный барьер, что создает ток утечки. Это является мотивацией к изучению альтернативных способов изготовления транзисторов.


Когда на базу транзистора не подается напряжение, через него не должен идти ток. В полупроводниковых транзисторах это достигается засчет эффекта TODO, что создает потенциальный барьер. Вообще говоря, всегда есть ненулевая вероятность туннелирования электронов через потенциальный барьер (и, соответственно, ненулевой ток утечки), которая в простейшем приближении зависит от

\begin{itemize}
\item Ширины потенциального барьера
\item Высоты потенциального барьера
\item Эффективной массы переносчиков заряда % https://ru.wikipedia.org/wiki/%D0%AD%D1%84%D1%84%D0%B5%D0%BA%D1%82%D0%B8%D0%B2%D0%BD%D0%B0%D1%8F_%D0%BC%D0%B0%D1%81%D1%81%D0%B0
\end{itemize}

TODO зафигачить сюда формулу

Как видно, вероятность туннелирования можно уменьшить следующим образом:

\begin{itemize}
\item Уменьшение размера транзистора прямо влечет за собой уменьшение ширины потенциального барьера
\item TODO ???
\item Увеличение эффективной массы переносчиков влечет за собой уменьшение мобильности носителей заряда и, соответственно, производительности транзистора
\end{itemize}

Таким образом, текущие размеры транзисторов подошли к тому пределу, при котором туннелирование уже нельзя игнорировать (TODO ссылку), в связи с чем интересно исследовать альтернативные реализации транзисторов, использующие резонансные квантовомеханические эффекты для реализации нелинейности в проводимости. Как правило, они представляют их себя квантовый волновод с барьерами или резонаторами (TODO написать что это TODO рисунов может?), соединенных с волноводом отверстиями. Параметры резонаторов легко регулировать, например, внешним электрическим полем. (TODO что-то про квантовые точки) Изменение параметров резонатора влечет за собой изменение коэффициента проводимости, и, соответственно, проводимости транзистора.

К сожалению, исследовать аналитически такие системы очень сложно. В качестве примера, можно привести одну из самых простых квантомеханических моделей: модель одномерной прямоугольной квантовой ямы (англ. 1D finite potential well) глубины $V_0$ и ширины $L$. Собственные энергии связанных состояний этой системы являются решениями трансцендетного уравнения вида $\alpha = k \tan \frac{k L}{2}$, где $k = \frac{\sqrt{2 m E}}{\hbar}$, $\alpha = \frac{\sqrt{2 m (E - V_0)}}{\hbar}$. Естественно, при увеличении размерности и сложности структуры, поиск собственных энергий и функций, необходимый для расчета коэффициента проводимости, становится еще более сложной задачей.

TODO разная эффективная масса?

Однако, существуют подходы, позволяющие изменить модель, получив некоторую ее аппроксимацию, которая допускает аналитические решения. TODO написать про потенциалы нулевого радиуса. Более того, в рамках этого подхода решения для всей системы выражаются как композиция решений резонатора и волновода. (TODO мутно)

В данной работе анализируются некоторые конфигурации квантовых волноводов с различными граничными условиями с помощью аппроксимации моделью с отверстиями нулевого радиуса.