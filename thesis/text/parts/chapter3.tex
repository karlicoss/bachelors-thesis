\chapter{Результаты} 
\label{chapter3}

Зафиксируем следующую геометрию волновода:
\begin{ilist}
# $L_x = 200$ боровских радиусов (порядка 10 нм);
# $L_y = 100$ боровских радиусов (порядка 5 нм);
# $H = 100$ боровских радиусов (порядка 5 нм);
# $S = 10$ боровских радиусов (порядка 0.5 нм).
\end{ilist}
Пусть слева поступает входящая волна на первой моде:
\[
\psi_{inc}(x, y) = \sqrt{\frac{2}{H}} \sin(\frac{\pi}{H} y) e^{i \sqrt{E - \pi^2 / H^2}}
\]
Далее приведены результаты, полученные в рамках подхода, описанного в главе 2.
\section{Зависимость коэффициента прохождения от энергии входящей волны при фиксированной геометрии резонатора}
\begin{figure}[H]
\includegraphics[width=1.0\textwidth]{transmission_all.png}
\caption{Зависимость коэффициента прохождения от энергии при фиксированной геометрии. Вертикальные пунктирные линии соответствуют собственным энергиям резонатора, красными парами чисел обозначены номера состояний.}
\label{fig:transmission_all}
\end{figure}
Как и ожидалось, в большинстве случаев коэффициент прохождения равен $1$. Падения коэффициента прохождения соответствуют тому, что волновая функция «чувствует» резонатор при некоторых энергиях, имеющих некоторую связь с собственными энергиями резонатора.

\begin{figure}[H]
\includegraphics[width=1.0\textwidth]{transmission_31.png}
\caption{Зависимость коэффициента прохождения от энергии при фиксированной геометрии в окрестности собственной энергии резонатора, соответствующей состоянию (3, 1).}
\label{fig:transmission_31}
\end{figure}
На рисунке~\ref{fig:transmission_31} приведена часть графика~\ref{fig:transmission_all} в окрестности собственного состояния резонатора (3, 1) в увеличенном масштабе.

\section{Плотности вероятности}
\begin{figure}[H]
\includegraphics[width=1.0\textwidth]{pdensity_31_r.png}
\caption{Плотность вероятности в резонансной точке.}
\label{fig:pdensity_31_r}
\end{figure}
На рисунке~\ref{fig:pdensity_31_r} можно наблюдать плотность вероятности волновой функции в точке, соответствующей  резонансу на рисунке~\ref{fig:transmission_31}. Как и ожидалось, она вся сконцентрирована в области резонатора, и потока вероятности через волновод почти не наблюдается, коэффициент прохождения равен $0$.

\begin{figure}[H]
\includegraphics[width=1.0\textwidth]{pdensity_31_nr.png}
\caption{Плотность вероятности в окресности резонансной точки.}
\label{fig:pdensity_31_nr}
\end{figure}
На рисунке~\ref{fig:pdensity_31_nr} можно наблюдать плотность вероятности волновой функции в точке, находящейся чуть правее резонанса на рисунке~\ref{fig:transmission_31}. Здесь наблюдается поток вероятности через волновод, коэффициент прохождения — почти $1$.

\section{Зависимость коэффициента прохождения от геометрии резонатора}
Из рисунка~\ref{fig:transmission_all} видно, что падения коэффициента прохождения достаточно резкие, поэтому если прибор будет работать на энергии, близкой к энергии, при которой происходит падение коэффициента прохождения, небольшим изменением геометрии резонатора можно получить большой скачок коэффициента прохождения. Приведем пример варьирования ширины резонатора в окресности резонанса.

\begin{figure}[H]
\includegraphics[width=1.0\textwidth]{transmission_size.png}
\caption{Зависимость проводимости от ширины резонатора в окрестности резонанса.}
\label{fig:transmission_size}
\end{figure}
На рисунке~\ref{fig:transmission_size} можно видеть резкую впадину в коэффициенте прохождения в районе $L_y = 200$. Если физически реализовать резонатор как квантовую точку, ее ширину можно будет контроллировать, и при изменении ширины резонатора менее чем на 0.1\%, коэффициент прохождения изменится от нуля до единицы (или наоборот), что соответствует переключению состояния транзистора.

% \todo{Полюса резольвенты}
% \todo{Вольт-амперная характеристика}
% \todo{Сравнение с численным решением или просто обоснование адекватности результата}
% \todo{Объяснить, почему нужно рассматривать бесконечный домен (на самом деле просто граничное условие такое)}