\chapter{Результаты} 
\label{chapter3}

Примеры результатов, полученых при применении расширения, описанного в предыдущей главе. Во всех случаях предполагается, что слева поступает входящая волна:
\[
\psi_{inc}(x, y) = \sqrt{\frac{2}{H}} \sin(\frac{\pi}{H} y) e^{i \sqrt{E - \pi^2 / H^2}}
\]

\section{Зависимость коэффициента прохождения от энергии входящей волны при фиксированной геометрии резонатора}
\begin{figure}[h!]
\includegraphics[width=1.0\textwidth]{transmission_2_3-1-2_0.png}
\caption{Зависимость коэффициента прохождения от энергии при фиксированной геометрии.}
\label{fig:trans_over_energy}
\end{figure}
Рисунок~\ref{fig:trans_over_energy} соответствует следующей геометрии:
\begin{ilist}
# $L_x = 2.3$;
# $L_y = 1.0$;
# $H = 2.0$.
\end{ilist}
Как и ожидалось, в большинстве случаев коэффициент прохождения равен $1$. Падения коэффициента прохождения соответствуют тому, что волновая функция «чувствует» резонатор при некоторых энергиях, имеющих некоторую связь с собственными энергиями резонатора (отмечены вертикальными пунктирными полосами). 

\section{Плотности вероятности}
\begin{figure}[h!]
\includegraphics[width=1.0\textwidth]{pdensity_12_2.png}
\caption{Плотность вероятности при энергии $E = 12.2$.}
\label{fig:pdensity_12_2}
\end{figure}
На рисунке~\ref{fig:pdensity_12_2} можно наблюдать плотность вероятности волновой функции в точке, соответствующей первому резонансу на рисунке~\ref{fig:trans_over_energy}. Как и ожидалось, она вся сконцентрирована в области резонатора, и потока вероятности через волновод не наблюдается, коэффициент прохождения равен $0$.

\begin{figure}[h!]
\includegraphics[width=1.0\textwidth]{pdensity_18_4.png}
\caption{Плотность вероятности при энергии $E = 18.4$.}
\label{fig:pdensity_18_4}
\end{figure}
На рисунке~\ref{fig:pdensity_18_4} можно наблюдать плотность вероятности волновой функции в точке, соответствующей второму резонансу на рисунке~\ref{fig:trans_over_energy}. Аналогично, потока вероятности через волновод практически не наблюдается, коэффициент прохождения равен $0$, плотность вероятности сконцентрирована в области резонатора. Артефакты, которые можно наблюдать посередине не имеют физической природы, и являются особенностью вычислений функций Грина и «обрезании» бесконечных сумм. Также намеренно затемнена область отверстия, так как там находится сингулярность, и в противном случае все изображение, кроме нее, было бы полностью черным.

\begin{figure}[h!]
\includegraphics[width=1.0\textwidth]{pdensity_19_0.png}
\caption{Плотность вероятности при энергии $E = 19.0$.}
\label{fig:pdensity_19_0}
\end{figure}
На рисунке~\ref{fig:pdensity_19_0} можно наблюдать плотность вероятности волновой функции в точке, находящейся чуть правее второго резонанса на рисунке~\ref{fig:trans_over_energy}. Здесь наблюдается поток вероятности через резонатор, коэффициент прохождения — почти $1$.

\section{Зависимость коэффициента прохождения от геометрии резонатора}
Из рисунка~\ref{fig:trans_over_energy} видно, что падения коэффициента прохождения достаточно резкие, поэтому если прибор будет работать на энергии, близкой к энергии, при которой происходит падение коэффициента прохождения, небольшим изменением геометрии резонатора можно получить большой скачок коэффициента прохождения. Приведем пример варьирования ширины резонатора в окресности второго резонанса.

\begin{figure}[h!]
\includegraphics[width=1.0\textwidth]{transmission_size.png}
\caption{Зависимость проводимости от ширины резонатора при энергии $E = 19.0$.}
\label{fig:transmission_size}
\end{figure}
На рисунке~\ref{fig:transmission_size} можно видеть резкое падение коэффициента прохождения от $L_y = 2.1$ к $L_y = 2.2$, и резкое возрастание к $L_y = 2.3$. Если физически реализовать резонатор как квантовую точку, ее ширину можно будет контроллировать, что позволит получать нелинейную зависимость коэффициента прохождения от напряжения, приложенного на квантовую точку и реализовать микроприбор, ведущий себя как транзистор.

% \todo{Полюса резольвенты}
% \todo{Вольт-амперная характеристика}
% \todo{Сравнение с численным решением или просто обоснование адекватности результата}
% \todo{Объяснить, почему нужно рассматривать бесконечный домен (на самом деле просто граничное условие такое)}