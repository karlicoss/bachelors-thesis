\startconclusionpage

В работе предложен метод получения приближенных спектральных и проводящих характеристик волновода сложной структуры, основанный на самосопряженных расширениях симметрических операторов и выходе в понрягинское пространство функций. Для моделей подобных конфигураций не существует аналитических решений, и автору неизвестны способы их численного решения. \todo{ну мало ли что автору неизвестно}.

На основе полученных данных сделан вывод, что данная модель подходит в качестве возможной реализации наноразмерных транзисторов, непосредственно использующих квантовые эффекты. Построена вольт-амперная характеристика квантового транзистора, на которой явно прослеживаются нелинейности, критичные для прибора.

Данная работа отличается новизной, так как ранее подобные вычисления были проделаны только для волноводов с граничным условием Неймана \todo{ссылки}, которое, во-первых, не является физически обоснованным, а во-вторых, значительно проще для вычислений в рамках теории расширений симметрических операторов в том смысле, что расчеты могут быть проделаны в пространстве $\mcL^2$. \todo{чето мало}