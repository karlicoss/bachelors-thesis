\chapter{Описание реализованного подхода}
\label{chapter2}

\section{Модель двумерного волновода с прямоугольным резонатором}

\begin{figure}[h!]
\begin{tikzpicture}[scale=0.7]
\newcommand{\Lx}{3.0};
\newcommand{\Ly}{1.5};
\newcommand{\Sw}{0.2};
\newcommand{\wgH}{2.0};
\newcommand{\wgL}{-10.0};
\newcommand{\wgR}{10.0};
\coordinate (Wul) at (\wgL, 0);
\coordinate (Wur) at (\wgR, 0);
\coordinate (Wbl) at (\wgL, -\wgH);
\coordinate (Wbr) at (\wgR, -\wgH);
\coordinate (Rbl) at (-\Lx / 2, 0);
\coordinate (Rbr) at (\Lx / 2, 0);
\coordinate (Rul) at (-\Lx / 2, \Ly);
\coordinate (Rur) at (\Lx / 2, \Ly);
\coordinate (Sl) at (-\Sw / 2, 0);
\coordinate (Sr) at (\Sw / 2, 0);

% waveguide + resonator
\draw[ultra thick, dotted] (\wgL - 1.0, 0) -- (Wul);
\draw[ultra thick] (Wul)--(Sl);
\draw[ultra thick] (Sr)--(Wur);
\draw[ultra thick, dotted] (Wur) -- (\wgR + 1.0, 0);
\draw[ultra thick, dotted] (\wgL - 1.0, -\wgH) -- (Wbl);
\draw[ultra thick] (Wbl)--(Wbr);
\draw[ultra thick, dotted] (Wbr) -- (\wgR + 1.0, -\wgH);
\draw[ultra thick] (Rbl) -- node [left] {$L_y$} (Rul) -- node [above] {$L_x$} (Rur) -- (Rbr);
%

% labels
\draw (0, \Ly / 2) node {$\Omega_R$};
\draw (0, -\wgH / 2) node {$\Omega_W$};
\draw[|<->|] (\wgL / 2, 0) -- node [left]{$H$} (\wgL / 2, -\wgH);
%
\end{tikzpicture}
\caption{Модель двумерного волновода с резонатором}
\end{figure}

\begin{ilist}
# область волновода $\Omega_W$ бесконечна по координате $x$ в обоих направлениях, и имеет ширину $H$;
# область резонатора $\Omega_R$ имеет размеры $L_x \times L_y$;
# резонатор и волновод соединены небольшим отверстием, которое симметрично относительно стенок резонатора;
# в левую часть волновода поступает входящая волна \todo{написать тут че-то нормальное};
# на стенках волновода и резонатора $\Gamma_W, \Gamma_R$ стоит условие Дирихле: $\eval{\psi}_{\Gamma_W} = 0$, $\eval{\psi}_{\Gamma_R} = 0$, которое означает, что элекртоны не проникают за стенки.
\end{ilist}

Целью работы является получение спектральных характеристик данной системы, в частности, резонансов и построение зависимости проводимости от геометрии резонатора. Как уже было сказано во \todo{введении}, для данной задачи не известны аналитические решения, и получение численных решений также представляет из себя значительную трудность.

Далее зафиксируем начало координат в точке, в которой находится центр отверстия отверстия $(x_0, y_0)$, то есть, $x_0 = 0, y_0 = 0$. В таких координатах: $\Omega_R = [- \frac{L_x}{2}, \frac{L_x}{2}] \times [0, L_y]$, $\Omega_W = [-\infty, \infty] \times [-H, 0]$.

\section{Собственные состояния, собственные энергии и функции Грина областей задачи}
Для дальнейших расчетов понадобятся решения следующих задач:

\subsection{Одномерная яма с бесконечными стенками}
Пусть доменом является $\Omega = [a, b]$ с условиями Дирихле: $\psi(a) = 0, \psi(b) = 0$. Обозначим $L = b - a$, тогда широко известно, что для этой задачи:
\begin{ilist}
# волновые векторы: $\vb{k_n} = \frac{\pi n}{L} \vb{\hat{x}}$, $n \in \bbN^+$;
# собственные энергии: $E_n = \frac{1}{2} k_n^2$, имеем чисто дискретный спектр;
# собственные состояния в координатном представлении: $\psi_n(x) = \sqrt{\frac{2}{L}} \sin(k_n (x - a))$.
# функция Грина в спектральном разложении выглядит как:
\[
G(x, s; E) = \frac{2}{L} \sum\limits_{n = 1}^\infty \frac{\sin(\frac{\pi n}{L} x) \sin(\frac{\pi n}{L} s)}{E_n - E}
\]
, что допускает замкнутый вид:
\[
G_{1D}(x, s; E) = - 2 \begin{cases}
\frac{\sin(k(x - a)) \sin(k(s - b))}{k \sin(k(b - a))}, & x < s \\
\frac{\sin(k(x - b)) \sin(k(s - a))}{k \sin(k(b - a))}, & x > s \\
\end{cases}
\]
, где $k = \sqrt{2 E}$.
% (from ex\_8370\_sol\_Y11.pdf):
% тут вроде все ок со знаком и константой
\end{ilist}
% # В импульсном представлении: $\Psi_n(p) = \sqrt{\pi L} n \left[ \frac{1 - (-1)^n e^{-i p L}}{\pi^2 n^2 - L^2 p^2} \right]$
% \todo{смещение}
% http://www.users.csbsju.edu/~frioux/momentum/pib-momentum.pdf

\subsection{Одномерная свободная частица}
Пусть доменом является $\Omega = (-\infty, \infty)$. Широко известно, что:
\begin{ilist}
# волновые векторы: $\vb{k_{p}} = p \vb{\hat{x}}$, $p \in \bbR$;
# собственные энергии: $E_p = \frac{1}{2} p^2$, имеем чисто непрерывный спектр;
# собственные состояния в координатном представлении: $\psi_p(x) = e^{i k_p x}$. \todo{хорошо бы нормализовать}
# \todo{функция Грина}
\end{ilist}

\subsection{Двумерная яма с бесконечными стенками}
Пусть доменом является $\Omega = [a_x, b_x] \times [a_y, b_y]$ с условиями Дирихле на границе: $\eval{\psi}_{\Gamma} = 0$. Данная задача соответстует области резонатора $\Omega_R$.

Для решения уравнения Шредингера разделим переменные, после чего получим уравнения:
\begin{ilist}
# $-\frac{1}{2} \laplacian \psi^x(x) = E^x \psi(x)$;
# $-\frac{1}{2} \laplacian \psi^y(y) = E^y \psi(y)$;
# $E^x + E^y = E$.
\end{ilist}
Эти уравнения соответствуют уже известным уравнениям для одномерной ямы с бесконечными стенками, они определяют полные системы одномерных собственных функций $\{\psi^x_n\}_n$ в $[a_x, b_x]$ и $\{\psi^y_m\}_m$ в $[a_y, b_y]$. Известно, что система функций $\psi_{n, m}(x, y) = \psi_n^x(x) \psi_m^y(y)$ в таком случае тоже будет полной системой функций в $[a_x, b_x] \times [a_y, b_y]$, значит, разделение переменных порождает все решения. Таким образом, получили:

\begin{ilist}
# волновые векторы: $k_{n, m} = \frac{\pi n}{L_x} \vb{\hat{x}} + \frac{\pi m}{L_y} \vb{\hat{y}}, n \in \bbN^+, m \in \bbN^+$;
# собственные энергии: $E_{n, m} = \frac{1}{2} k_{n, m}^2$;
# собственные состояния в координатном представлении: $\psi_{n, m}(x, y) = \sqrt{\frac{4}{L_x L_y}} \sin(k^x_n (x - a_x)) \sin(k^y_m (y - a_y))$
# \todo{функцию Грина}
\end{ilist}

\subsection{Бесконечная квазиодномерная полоса с бесконечными стенками}
Пусть доменом является $\Omega = [-\infty; +\infty] \times [a, b]$ с условием Дирихле на границе: $\eval{\psi}_{\Gamma} = 0$. Данная задача соответстует области волновода $\Omega_W$.

Для решения уравнения Шредингера разделим переменные, после чего получим уравнения:
\begin{ilist}
# $-\frac{1}{2} \laplacian \psi^x(x) = E^x \psi(x)$;
# $-\frac{1}{2} \laplacian \psi^y(y) = E^y \psi(y)$;
# $E^x + E^y = E$.
\end{ilist}
По координате $x$ получаем уравнение, соответствующие одномерной свободной частице, по координате $y$ — уравнения для одномерной ямы с бесконечными стенками. Решения каждой из них опредяют полные системы функций, значит, разделение переменных порождает все решения. Таким образом:
\begin{ilist}
# волновые векторы: $k_{p, m} = p \vb{\hat{x}} + \frac{\pi m}{L_y} \vb{\hat{y}}, p \in \bbR, m \in \bbN^+$;
# собственные энергии: $E_{p, m} = \frac{1}{2} k_{p, m}^2$;
# собственные состояния в координатном представлении: $\psi_{p, m}(x, y) = \sqrt{\frac{2}{L_y}} e^{\iu p x} \sin(k^y_m y)$
# \todo{функцию Грина}
\end{ilist}
Интересной особенностью данной задачи является то, что ее спектр состоит из непрерывного спектра по измерению $x$ и дискретного спектра по измерению $y$. \todo{можно картиночку}. В частности, для энергии $E$ собственное подпространство в общем случае вырожденное, и подпространство решений задается произвольными коэффициентами $R_m, L_m$:
\begin{align*}
\psi(x, y)
& = \sum\limits_{m = 1}^{\frac{\pi^2 m^2}{L_y^2} < E} R_m \sin(\frac{\pi m}{L_y} y) e^{\iu \sqrt{2 E - \pi^2 m^2 / L_y^2} x} \\
& + \sum\limits_{m = 1}^{\frac{\pi^2 m^2}{L_y^2} < E} L_m \sin(\frac{\pi m}{L_y} y) e^{-\iu \sqrt{2 E - \pi^2 m^2 / L_y^2} x}
\end{align*}
, где коэффициенты $R_m$ соответствуют волнам, распространяющимся вправо, а $L_m$ — волнам, распространающимся влево. \todo{что-нибудь про моды сказать}

\section{Построение модели с отверстием нулевой ширины}
Чтобы получить приближение спектральных характеристик и проводящих свойств данной модели, построем соответствующую ей модель с отверстием нулевой ширины. Заметим, что можем анализировать вместо оператора Гельмгольца $-\laplacian + E$, просто оператор $\Delta = \laplacian$, так как их самосопряженность совпадает \todo{как тут получше сказать?}.

Сузим домены операторов $\Delta_R$ и $\Delta_W$ до множеств гладких функций, удовлетворяющих условию Дирихле на замыканиях границ $\Gamma_R$ и $\Gamma_W$ (то есть, в том числе, и в точке $(x_0, y_0)$), обозначим полученные операторы $\Delta_{0R}$ и $\Delta_{0W}$, оператор для всей области $\Omega$ будет их прямой суммой: $\Delta_0 = \Delta_{0R} \oplus \Delta_{0W}$. Полученный оператор является симметрическим, проверим его самосопряженность расчетом индексов дефекта:

\todo{считаем...}

Видно, что у данного оператора индексы дефекта $(0, 0)$, то есть он существенно самосопряжен, что нам не подходит, так как нам хочется чтобы существовало расширение, в котором операторы взаимодействуют. Рассмотрим причину этого:

\todo{найти правильную формулу}

Мультипольное разложение элементов $\dom \Delta_0^*$:

\[
f(x, y) = \sum\limits_{i = 0}^\infty c_i d^i \pdv[i]{G}{n_s} (x, y, x_0, y_0; E_0)
\]

, где $d$ — радиус отверстия.

Так как функция Грина удовлетворяет тем же граничным условиям, то нулевое слагаемое будет равно нулю. Следующий кандидат на дефектный элемент — первая производная функции Грина. Однако, она не принадлежит $\mcL^2(\Omega)$, что и является причиной зануления индексов дефекта.

Чтобы получить расширение, необходимо:
\begin{ilist}
# заменить $\mcL^2(\Omega_R), \mcL^2(\Omega_W)$ более широкими пространствами функций, в котором лежит первая производная функции Грина;
# дать интерпретацию в этом пространстве формальному условию зануления граничной формы, которое определяет возможный выбор домена расширения:
\begin{align*}
J(f, g)
& = J_R(f, g) + J_W(f, g) \\
& = (\ip{\Delta_{0R}^* f}{g} - \ip{f}{\Delta_{0R}^* g}) + (\ip{\Delta_{0W}^* f}{g} - \ip{f}{\Delta_{0W}^* g})
\end{align*}
, то есть определить новое скалярное произведение.
\end{ilist}
В данной работе выбирается подход с выходом в понтрягинское простарнство. \todo{написать, почему именно в него, ссылки какие-нибудь}. Мы сделаем переход от гильбертова пространства $\mcL^2(\Omega_R) \oplus \mcL^2(\Omega_W)$ со скалярным произведением $\ipcdot_\Omega = \ipcdot_{\Omega_R} + \ipcdot_{\Omega_W}$ к расширенному понтрягинскому пространству $\Pi = \Pi_R \oplus \Pi_W$, которое будет содержать необходимые для расчетов функции и в котором будет корректно определено скалярное произведение на них.

% Гамильтониан в понтрягинском пространстве Dym 119 Albeverio 155

% Домен сопряженного оператора состоит из элементов вида

% \[
% \dom \Delta^* = \begin{cases}
% a_R \pdv{}{n_0} G_R(x, y, x_0, y_0; E_0) + u_R(x), & x \in \Omega_R \\
% a_W \pdv{}{n_0} G_W(x, y, x_0, y_0; E_0) + u_W(x), & x \in \Omega_W
% \end{cases}
% \]

% Хотелось бы воспользоваться формулой Грина для многомерного интегрирования по частям: \todo{сопряжение}
% \[
% \int\limits_\Omega (f \laplacian g - g \laplacian f) d \Omega = \oint\limits_\Gamma \left( f \pdv{g}{n} - g \pdv{f}{n} \right) d \Gamma
% \]


\subsection{Расчет дефектного элемента для резонатора}
Дефектный элемент — производная функции Грина по нормали в точке $x_s = x_0, y_s = y_0$, то есть для резонатора совпадает с направлением производной по $y$.

Функция Грина:

\begin{align*}
G(x, y, x_s, y_s; E) = N_x^2 N_y^2 \sum\limits_{n = 1, m = 1}^\infty \frac{\sin(k_n x + \frac{\pi n}{2}) \sin(k_n x_s  + \frac{\pi n}{2}) \sin(k_m y) \sin(k_m y_s)}{k_n^2 + k_m^2 - E}
\end{align*}

\begin{align*}
d(x, y)
&= N_x^2 N_y^2 \sum\limits_{n = 1, m = 1}^\infty \frac{\sin(k_n x + \frac{\pi n}{2}) \sin(\frac{\pi n}{2}) \sin(k_m y) k_m}{E_{n, m} - E} \\
& = N_x^2 N_y^2 \sum\limits_{n = 1,2, m = 1}^\infty \frac{\cos(k_n x) \sin(k_m y) k_m}{E_{n, m} - E}
\end{align*}

\begin{align*}
\psi(x, y)
& = H_0^* d(x, y) = - \laplacian d(x, y) \\
& = N_x^2 N_y^2 \sum\limits_{n = 1,2, m = 1}^\infty \frac{(k_n^2 + k_m^2) \cos(k_n x) \sin(k_m y) k_m}{E_{n, m} - E}
\end{align*}

\subsection{Построение предпонтрягинского пространства}
В качестве оператора, на котором будет построена шкала, выберем резольвенту $Z_0 = R_0(\iu S) = \frac{1}{\Delta_0 - \iu S}$, где $S$ — произвольное ненулевое вещественное число. $\iu S$ лежит в резольвентном множестве оператора $\Delta_0$ (его спектр — неотрицательные вещественные числа), поэтому оператор $Z_0$ корректно определен. Структура шкал не зависит от выбора $S$, поэтому пока оставим его без конкретного значения, его можно будет зафиксировать потом с учетом возможных упрощений вычислений.

Рассмотрим применение оператора $Z_0$:
\begin{align*}
& (Z_0 \psi)(x, y) \\
&= \int\limits_{\Omega_R} \sum\limits_{n = 1, m = 1}^\infty N_x^2 N_y^2 \frac{\sin(k_n x + \frac{\pi n}{2}) \sin(k_n x' + \frac{\pi n}{2}) \sin(k_m y) \sin(k_m y')}{k^2_n + k^2_m - \iu S} \\
& N_x^2 N_y^2 \sum\limits_{n' = 1, m' = 1}^\infty \frac{(k_{n'}^2 + k_{m'}^2) \sin(k_{n'} x' + \frac{\pi n'}{2}) \sin(\frac{\pi n'}{2}) \sin(k_{m'} y') k_{m'}}{k_{n'}^2 + k_{m'}^2 - E_D} \\
&= N_x^2 N_y^2 \sum\limits_{n = 1, m = 1}^\infty \frac{(k_n^2 + k_m^2) \sin(k_n x + \frac{\pi n}{2}) \sin(\frac{\pi n}{2}) \sin(k_m y) k_m}{(k_n^2 + k_m^2 - \iu S) (k_n^2 + k_m^2 - E_D)}
\end{align*}

Легко заметить, что 

\begin{align*}
(Z_0^t \psi)(x, y)
&= N_x^2 N_y^2 \sum\limits_{n = 1, m = 1}^\infty \frac{(k_n^2 + k_m^2) \sin(k_n x + \frac{\pi n}{2}) \sin(\frac{\pi n}{2}) \sin(k_m y) k_m}{(k_n^2 + k_m^2 - \iu S)^t (k_n^2 + k_m^2 - E_D)} \\
&= N_x^2 N_y^2 \sum\limits_{n = 1,2, m = 1}^\infty \frac{(k_n^2 + k_m^2) \cos(k_n x) \sin(k_m y) k_m}{(k_n^2 + k_m^2 - \iu S)^t (k_n^2 + k_m^2 - E_D)}
\end{align*}


Определим порядок предпонтрягинского пространства — такое минимальное $t$, что $Z_0^t \psi \in \hilb{0}$, то есть что $Z_0^t \psi$ квадратично интегрируема на $\Omega_R$.

\begin{align*}
& \ip{Z_0^t \psi}{Z_0^t \psi} = \int\limits_{\Omega_R} \cconj{Z_0^t \psi(x, y))} Z_0^t \psi(x, y) \dd{x} \dd{y} \\
&\propto \sum\limits_{n = 1,2, m = 1}^\infty \frac{(n^2 + m^2) m}{\cconj{(n^2 + m^2 - \iu S)^t (n^2 + m^2 - E_D)}} \frac{(n^2 + m^2) m}{(n^2 + m^2 - \iu S)^t (n^2 + m^2 - E_D)} \\
&= \sum\limits_{n = 1,2, m = 1}^\infty \frac{(n^2 + m^2)^2 m^2}{((n^2 + m^2)^{2t} + S^2) ((n^2 + m^2)^{2} - |E_D|^2)} 
\end{align*}

Для исследования сходимости данного ряда воспользуемся интегральным признаком сходимости и перейдем в полярные координаты:
\begin{align*}
\int\limits_{r = 1}^\infty \int\limits_{\phi = 0}^{\pi / 2} \frac{r^4 r^2 \sin^2 \phi}{(r^{4t} + S^2) (r^{4} - |E_D|^2)} r \dd{r} \dd{\phi}
= \int\limits_{r = 1}^\infty \frac{r^7}{(r^{4t} + S^2) (r^{4} - |E_D|^2)} \dd{r}
\end{align*}

Минимальное $t$, для которого интеграл конечен, равно $2$, значит, $\psi \in \hilb{-2}$. Пространство $\mcP$ состоит из:
\[
f(x, y) = f_\phi(x, y) + f_{1} (Z_0^3 \psi)(x, y) + f_{0} (Z_0^2 \psi)(x, y) + f_{-1} (Z_0 \psi)(x, y) + f_{-2} \psi(x, y)
\]

Матрица $G$ для скалярного произведения:

\[
\kbordermatrix{
   & 1                        &       0 &      -1 & -2      \\
1  & \ip{}{}_0 & \ip{}{}_0 & \ip{}{}_0 & \ip{}{}_0 \\
0  & \ip{}{}_0 & \ip{}{}_0 & \ip{}{}_0 & G_{0, -2} \\
-1 & \ip{}{}_0 & \ip{}{}_0 & G_{-1,-1} & G_{-1, -2} \\
-2 & \ip{}{}_0 & G_{-2,0} & G_{-2,-1} & G_{-2,-2} \\
}
\]

Скалярные произведения для элементов $G_{i, j}$ c $i + j \ge -1$, сходятся по интегральном признаку, поэтому их можно получить непосредственно. Остальные элементы определяются следующим образом:

\begin{elist}
# зафискируем
## $\Re G_{-1, -1} = a$;
## $\Re G_{-2, 0} = b$;
## $\Re G_{-2, -1} = c$;
## $\Re G_{-2, -2} = d$;
# $G_{-1, -1} = a = \frac{G_{-2, -1} - G_{-1, -2}}{2 \iu} = \frac{(c + \iu \Im G_{-2, -1}) - (c - \iu \Im G_{-2, -1})}{2 \iu} = \Im G_{-2, -1}$
# $G_{-1, 0} = \frac{G_{-2, 0} - G_{-1, -1}}{2 \iu} = \frac{b + \iu \Im G_{-2, 0} - a}{2 \iu}$, из чего можно выразить $\Im G_{-2, 0} = 2 G_{-1, 0} + \iu (b - a)$.
\end{elist}


% Имеем ограничения:

% \begin{ilist}
% # $G_{-2,0} = \cconj{G_{0, -2}}$
% # $G_{-2,-1} = \cconj{G_{-1, -2}}$
% # $\Im G_{-2,-2} = 0$
% # $\Im G_{-1,-1} = 0$
% # Ограничение на $G_{-2, 0}$ и $G_{-1, -1}$
% # Ограничение на $G_{-2, -1}$ и $G_{-1, -1}$
% # $G_{-2, -2}$ на самом деле не понадобится вообще
% \end{ilist}

После этого, воспользовавшись эрмитовой симметричностью, доопределим оставшиеся элементы. Таким образом, у нас имеется четыре свободных вещественных параметра $a, b, c, d$.
\todo{и че с этим делать? $d$ вроде не нужен вообще}


Напомним, что нас будет интересовать скалярное произведение $\ip{\Delta f}{g}$. Надо вложить в понтгрягинское пространство:

\begin{elist}
# $\Delta_0 d$

Пользуясь тем, что по определению, $\psi = \Delta_0 d$, получаем
\[
\Delta_0 d \xmapsto[]{\mcP} (0, 0, 0, 0, 1) \xmapsto[]{\Pi} (0; 0, 0; 0, 1)
\]
# $d$

Заметим:
\begin{align*}
& \psi_{-1} = Z_0 \psi = \frac{1}{\Delta_0 - z_0} \Delta_0 d = d + z_0 Z_0 d \in \hilb{-1} \\
& z_0 \psi_{0} = z_0 Z_0 d + z_0^2 Z_0^2 d \in \hilb{0} \\
& z_0^2 \psi_{1} = z_0^2 Z_0^2 d + z_0^3 Z_0^3 d \in \hilb{1} \\
& f_\phi = z_0^3 Z_0^3 d \in \hilb{2}
\end{align*}

Поочередно вычитаем и прибавляем, получаем  $\psi_{-1} - z_0 \psi_0 + z_0^2 \psi_1 - f_\phi =  d$, соответственно:
\[
d \xmapsto[]{\mcP} (-z_0^3 Z_0^3 d, z_0^2, -z_0, 1, 0) \xmapsto[]{\Pi} (-z_0 Z_0 d; \left[ \ip{\psi_i}{-z_0^3 Z_0^3 d}_0 + \sum\limits_{j = 0}^1 f_j G_{i, j} \right]_{i = -1}^{-2};1, 0)
\]
# $u \in W_2^2$

Элемент пространства $W_2^2$ находится в $\hilb{2}$ \todo{точно?}, поэтому:
\[
u \xmapsto[]{\mcP} (u, 0, 0, 0, 0) \xmapsto[]{\Pi} (u;\left[ \ip{\psi_i}{u}_0 \right]_{i = -1}^{-2}; 0, 0)
\]
# $\Delta_0 u$

$\Delta_0 u \in \hilb{0}$ по определению пространства $W_2^2$. $\hilb{2}$ плотно в $\hilb{0}$, значит, существует последовательность $f_n \in \hilb{2}$: $\lim\limits_{n \to \infty} f_n = \Delta_0 u$.

\[f_n \xmapsto[]{\mcP} (f_n, 0, 0, 0, 0) \xmapsto[]{\Pi} (f_n, \left[ \ip{\psi_i}{f_n}_0 \right]_{i = -1}^{-2}, 0^2) \xrightarrow[n \to \infty]{\Pi} (\Delta_0 u; \left[ \ip{\psi_i}{\Delta_0 u}_0 \right]_{i = -1}^{-2}; 0, 0)\]
% Как это аналитически будем считать, вообще непонятно.
\end{elist}

% bordermatrix