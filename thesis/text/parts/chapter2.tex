\chapter{Описание реализованного подхода}
\label{chapter2}

Работаем в пространстве $\bbR^3$. Исходный гамильтониан: $H_0 = -\laplacian$ в $L_2(\bbR^3, \dd x)$.

Возьмем в качестве точечного взаимодействия функционал $\bra{\vb{0}} = \delta(\vb{0})$.

Далее удобно будет перейти в импульсное представление $\psi(\vb{p})$. В нем дельта-функция Дирака имеет постоянное значение (TODO указать конкретное), гамильтониан $H_0 = \hat{p}^2$, соответственно, $R_0(z_0) = \frac{1}{p^2 - z_0}$. На шкале дельта-функция первый раз встречается в пространстве $\hilb{1}$:

\begin{itemize}
\item $\delta_x(\vb{p}) \notin \hilb{0}$, так как $\int\limits_{\bbR^5} 1 \cdot 1 \cdot \dd \vb{x} = \infty$
\item $\delta_x(\vb{p}) R_0(z_0) \in L^2(\bbR^5)$, так как $\int\limits_{\bbR^5} \cconj{R_0(z_0)} R_0(z_0) \dd \vb{x} = \sqrt{2} \pi^2$
\end{itemize}


\section{Условие Дирихле}
TODO написать, что функции Грина не нужны, нужны производные по нормали по источнику

TODO написать, что производные функции Грина имеют неинтегрируемую особенность

$\psi = \pdv{n_s} \delta(\vb{r})$.

Рисунок 2. Нормаль направлена по координате $x_1$, $\psi = \delta'(x_1) \delta(x_2) \delta(x_3)$. Перейдем в импульсное представление:

В импульсном представлении гамильтониан $H_0 = \hat{p}^2$, соответственно, $R_0(z_0) = \frac{1}{p^2 - z_0}$.

Переведем функционал $\psi$ импульсное предсавление:

TODO: $\hbar$
\begin{align*}
\Psi(p_1, p_2, p_3)
&= \mathcal{F}(x_1, x_2, x_3) \\
&= \frac{1}{\sqrt{2 \pi}} \int\limits_{x_1} \frac{1}{\sqrt{2 \pi}} \int\limits_{x_2} \frac{1}{\sqrt{2 \pi}} \int\limits_{x_3} \psi(x_1, x_2, x_3) e^{-i \vb{x} \vdot \vb{p}} \\
&= \frac{1}{\sqrt{2 \pi}} \int\limits_{x_1} \delta'(x_1) e^{-i x_1 p_1} \frac{1}{\sqrt{2 \pi}} \int\limits_{x_2} \frac{1}{\sqrt{2 \pi}} \int\limits_{x_3} \delta(x_2) \delta(x_3) e^{-i x_2 p_2} e^{-i x_3 p_3} \\
&= \frac{1}{\sqrt{2 \pi}} \int\limits_{x_1} \delta'(x_1) e^{-i x_1 p_1} \mathcal{F}(\delta(x_2) \delta(x_3)) \\
&= \frac{1}{2 \pi} \mathcal{F}(\delta'(x_1)) \\
&= \frac{1}{2 \pi} i p_1 \mathcal{F}(\delta(x_1)) \\
&= \left(\frac{1}{\sqrt{2 \pi}}\right)^3 i p_1
\end{align*}

\begin{itemize}
\item $\Psi \notin \hilb{0}$ 
\item $\Psi \notin \hilb{-1}$
\item $\Psi \in \hilb{-2}$
\end{itemize}


Получаем предпонтрягинское пространство $\mcP_{-1}$, состоящее из функций вида:

\[
f(\vb{p}) = f_\phi(\vb{p}) + f_0 \frac{1}{(p^2 - i)^2} + f_{-1} \frac{1}{p^2 - i}
\]

Скалярное произведение:

\[
\ip{f}{g} = \ip{f_\phi}{g_\phi} + \dots + \cconj{f_0} G_{0,0} g_0 + \cconj{f_0} G_{0, -1} g_0 + \cconj{f_{-1}} G_{-1, 0} g_0 + \cconj{f_{-1}} G_{-1, -1} g_{-1}
\]

\begin{itemize}
\item $G_{0, 0}$ точно определено, так как $0 + 0 \ge 0$
\item $G_{0, -1} = \cconj{G_{-1, 0}} = ?$ — конечное
\item $G_{-1, -1}$ — свободный параметр
\end{itemize}