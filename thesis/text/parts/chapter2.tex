\chapter{Описание реализованного подхода}
\label{chapter2}

\section{Формула Крейна}
\[
R_\lambda(z) = R_0(z) - R_0(z) \ket{\psi} \Gamma^{-1}(z, \lambda) \bra{\psi} R_0(z)
\]

\[
\Gamma(z, \lambda) = \frac{1}{\lambda} + \frac{1}{2}(z - z_0) \bra{\psi} R_0(z) R_0(z_0) \ket{\psi} + \frac{1}{2}(z - \cconj{z_0}) \bra{\psi} R_0(z) R_0(\cconj{z_0}) \ket{\psi}
\]

\section{2D}
% Пусть $\frac{\hbar^2}{2 \mu} = 1$

$\psi(x_1, x_2) = \delta'(x_1) \delta(x_2)$

$\Psi(p_1, p_2) = \mcF(\psi(x_1, x_2)) = \frac{1}{2 \pi \hbar} i p_1$

Заметим, что так как в импульсном представлени резольвента — оператор умножения, то ее применение коммутирует с символами функций, и можно получить:

\begin{align*}
\ip{f}{g}_0
&= \ip{R_0^k(\cconj{z_0}) f}{R_0^{-k}(z_0) g} \\
&= \int\limits_{\bbR^2} \cconj{R_0^k(\cconj{z_0}) f(\vb{p})} R_0^{-k}(z_0) g(\vb{p}) \dd[2]{\vb{p}} \\
&= \int\limits_{\bbR^2} f(\vb{p}) g(\vb{p}) \dd[2]{\vb{p}} \\
&= \ip{f}{g}
\end{align*}

$\psi \notin \hilb{0}$, $L^2$-норма $\psi$ бесконечна

$\psi \notin \hilb{-1}$, $L^2$-норма $Z_0 \psi$ бесконечна

$\psi \in \hilb{-2}$, $L^2$-норма $Z_0^2 \psi$ конечна

Получаем предпонтрягинское пространство $\mcP_{-1}$, состоящее из функций вида TODO нормализацию пофиксить:

\[
f(\vb{p}) = f_\phi(\vb{p}) + f_0 \frac{1}{(\frac{p^2}{2 \mu} - i)^2} \Psi(\vb{p}) + f_{-1} \frac{1}{\frac{p^2}{2 \mu} - i} \Psi(\vb{p})
\]

Скалярное произведение:

\begin{align*}
\ip{f}{g}
&= \ip{f_\phi}{g_\phi} \\
&+ \sum\limits_{i = -1}^0 \ip{f_i \psi_i}{g_\phi}_0 + \sum\limits_{j = -1}^0 \ip{f_\phi}{g_j \psi_j}_0 \\
&+ \sum\limits_{i = -1}^0 \sum \limits_{j = -1}^0 \cconj{f_i} G_{i,j} g_j 
\end{align*}

\begin{itemize}
\item $G_{0, 0} = \ip{\psi_0}{\psi_0} = \frac{\mu^2}{4 \pi \hbar^2}$
\item $G_{0, -1} = \cconj{G_{-1, 0}} = \frac{\mu^2 (2 i + \pi)}{8 \pi \hbar^2}$. TODO и с какого? функция не в $\hilb{1}$ же???
\item $G_{-1, -1}$ — свободный (вещественный) параметр
\end{itemize}

Свободный параметр $G_{-1, -1}$ определим так, чтобы оно было «конечной» частью расходящегося скалярного произведения $\ip{\psi_{-1}}{\psi_{-1}}_0$, следующим образом:

Перейдем в полярные координаты $(p, \varphi)$

Рассмотрим интеграл в скалярном произведении $\ip{\cdot}{\cdot}_0$, на области с $p < N$. Для этого удобнее будет перейти в полярные координаты:

\begin{align*}
p_1 &\to p \cos \varphi \\
\dd[2]{\vb{p}} &\to p \dd{p} \dd{\varphi}
\end{align*}

\[
\int\limits_{\varphi = 0}^{2 \pi} \int\limits_{p = 0}^N
\cconj{\left(
\frac{1}{\frac{p^2}{2 \mu} + i} \frac{1}{\frac{p^2}{2 \mu} - i} \frac{1}{2 \pi \hbar} i p \cos \varphi
\right)}
\left(
\frac{1}{2 \pi \hbar} i p \cos \varphi
\right)
p \dd{p} \dd{\varphi}
=
\frac{\mu^2}{4 \pi \hbar^2} \log(1 + \frac{N^4}{4 \mu^2})
\]

TODO ссылка на Березина

Предел $\frac{1}{\log(N) \cdot} = \frac{\mu^2}{\pi \hbar^2}$.

Ну в общем, отнормированный интеграл:

\[
-\frac{\mu^2 \log (2 \mu)}{2 \pi  \hbar ^2}
\]

Ну ок.

TODO подставить в формулу Крейна и найти полюса резольвенты

Найдем $\Gamma^{-1}(z, \lambda)$. Заметим, что $R_0(z) \psi$ в общем случае не лежит в подпространстве, соответствующему $\psi_1$, поэтому надо выделить его коэффициенты $f_0, f_1$ и функцию $f_\phi$:

% Далее в нотации используется, что резольвента -- оператор умножения.

Далее будет полезно следущее тождество:

$R(z) = \frac{1}{H - z} = \frac{1}{H - z_0} + \frac{z - z_0}{(H - z)(H - z_0)} = \frac{1}{H - z_0} + \frac{z - z_0}{(H - z_0)^2} + \frac{(z - z_0)^2}{(H - z_0)^2 (H - z)}$

Используя тождество, получим компоненты элемента $R_0(z) \psi$ в пространстве $\mcP_1$:

\begin{itemize}
\item $f_{-1} = 1$
\item $f_0 = z - i$
\item $f_\phi = \frac{(z - i)^2}{(\frac{p^2}{2 \mu} - i)^2 (\frac{p^2}{2 \mu} - z)} \frac{1}{2 \pi \hbar} i p_1$
\end{itemize}

% Разложим резольвенту в пространстве $\mcP_1$

% $R_0(z) \psi = \frac{1}{\frac{p^2}{2 \mu} - z} \frac{1}{2 \pi \hbar} i p_1 = f_\varphi + f_0 \frac{1}{\left(\frac{p^2}{2 \mu} - z_0 \right)^2} \psi + f_{-1} \frac{1}{\frac{p^2}{2 \mu} - z_0} \psi$

Замыкание $\mcP_1$ дает нам понтрягинское пространство $\Pi_1$


\section{3D}
Рассмотрим уравнение Шредингера в двух областях $\Omega_1$ и $\Omega_2$ с условиями Дирихле на границе, общей границей $\partial \Gamma_{12}$, и точечным проколом $\vb{a_{12}} \in \partial \Gamma_{12}$. TODO картиночку

Типичный пример такой задачи — резонатор Гельмгольца TODO

TODO как физически реализуемо?

TODO написать, что функции Грина не нужны, нужны производные по нормали по источнику

TODO написать, что производные функции Грина имеют неинтегрируемую особенность

Далее необходимо выполнить расширение пространства отдельно в каждой области $\Omega_1, \Omega_2$, а потом «сшить» решения с некоторым граничным условием в окрестности точечного прокола $\vb{a_{12}}$. Далее, выберем систему координат так, чтобы точечный прокол оказался 

Работаем в пространстве $\bbR^3$. Исходный гамильтониан: $H_0 = \frac{\hat{p}^2}{2 \mu}$ в $L_2(\bbR^3, \dd x)$.

% H_0 = - \frac{\hbar^2}{2 \mu } \laplacian

$\psi = \pdv{n_s} \delta(\vb{r})$.

Спектром оператора является $\bbR$, соответственно, резольвентное множество — $\{ c \mid \Im c \ne 0 \}$. В качестве $z_0$ возьмем, например, $i$, (напомним, что его выбор не влияет на структуру шкалы).


Рисунок 2. Нормаль направлена по координате $x_1$, $\psi = \delta'(x_1) \delta(x_2) \delta(x_3)$. Перейдем в импульсное представление:

В импульсном представлении гамильтониан $H_0 = \frac{p^2}{2 \mu}$, соответственно, $R_0(z_0) = \frac{1}{p^2 - z_0}$ TODO fix.

Переведем функционал $\psi$ импульсное предсавление:

TODO: $\hbar$
\begin{align*}
\Psi(p_1, p_2, p_3)
&= \mathcal{F}(\psi(x_1, x_2, x_3)) \\
&= \frac{1}{\sqrt{2 \pi \hbar}} \int\limits_{x_1} \frac{1}{\sqrt{2 \pi \hbar}} \int\limits_{x_2} \frac{1}{\sqrt{2 \pi \hbar}} \int\limits_{x_3} \psi(x_1, x_2, x_3) e^{-i \vb{x} \vdot \vb{p}} \\
&= \frac{1}{\sqrt{2 \pi \hbar}} \int\limits_{x_1} \delta'(x_1) e^{-i x_1 p_1} \frac{1}{\sqrt{2 \pi \hbar}} \int\limits_{x_2} \frac{1}{\sqrt{2 \pi \hbar}} \int\limits_{x_3} \delta(x_2) \delta(x_3) e^{-i x_2 p_2} e^{-i x_3 p_3} \\
&= \frac{1}{\sqrt{2 \pi \hbar}} \int\limits_{x_1} \delta'(x_1) e^{-i x_1 p_1} \mathcal{F}(\delta(x_2) \delta(x_3)) \\
&= \frac{1}{2 \pi \hbar} \mathcal{F}(\delta'(x_1)) \\
&= \frac{1}{2 \pi \hbar} i p_1 \mathcal{F}(\delta(x_1)) \\
&= \left(\frac{1}{\sqrt{2 \pi \hbar}}\right)^3 i p_1
\end{align*}

\begin{itemize}
\item $\Psi \notin \hilb{0}$ 
\item $\Psi \notin \hilb{-1}$
\item $\Psi \in \hilb{-2}$
\end{itemize}


Получаем предпонтрягинское пространство $\mcP_{-1}$, состоящее из функций вида:

\[
f(\vb{p}) = f_\phi(\vb{p}) + f_0 \frac{1}{(p^2 - i)^2} + f_{-1} \frac{1}{p^2 - i}
\]

Скалярное произведение:

\[
\ip{f}{g} = \ip{f_\phi}{g_\phi} + \dots + \cconj{f_0} G_{0,0} g_0 + \cconj{f_0} G_{0, -1} g_0 + \cconj{f_{-1}} G_{-1, 0} g_0 + \cconj{f_{-1}} G_{-1, -1} g_{-1}
\]

\begin{itemize}
\item $G_{0, 0}$ точно определено, так как $0 + 0 \ge 0$
\item $G_{0, -1} = \cconj{G_{-1, 0}} = ?$ — можно определить, так как можно показать, что $\psi_0 \in \hilb{1}$, а значит, скалярное произведение конечное
\item $G_{-1, -1}$ — свободный параметр
\end{itemize}

TODO пока мутно, как определять свободный параметр. Пусть определили.

