\chapter{Описание реализованного подхода}
\label{chapter2}

\section{Модель двумерного волновода с прямоугольным резонатором}
\todo{Картиночку}

\todo{Про то, какой физический смысл этой модели}

Выберем начало координат в точке, в которой находится отверстие, то есть, $x_0 = 0, y_0 = 0$.

$\Omega_R$ — область резонатора, $\Omega_R = [- \frac{L_x}{2}, \frac{L_x}{2}] \times [0, L_y]$, где $L_x$ — ширина резонатора, $L_y$ — высота резонатора.

$\Omega_W$ — область волновода, $\Omega_W = [-\infty, \infty] \times [-H, 0]$, где $H$ — высота волновода.


\section{Собственные функции, собственные энергии и функции Грина областей задачи}
Далее будут необходимы \todo{блабла}
\subsection{Резонатор}
Уравнение Шредингера с условием Дирихле на границе
\begin{align*}
- \frac{1}{2}\laplacian \psi(x, y) &= E \psi(x, y) \\
\eval{\psi}_{\Gamma_R}{} &= 0
\end{align*}
, допускает разделение переменных в области $\Omega_R$. Разделив переменные, получим решения:

\[
\begin{cases}
E_{n,m} = E_n^x + E_m^y \\
\psi_{n, m}(x, y) = \psi_n(x) \psi_m(y) \\
k_n^x = \frac{\pi n}{L_x}, n = 1 \dots \infty \\
k_m^y = \frac{\pi m}{L_y}, m = 1 \dots \infty \\
E_n^x = \frac{1}{2} (k_n^x)^2 \\
E_m^y = \frac{1}{2} (k_m^y)^2 \\
\psi_n(x) = \sqrt{\frac{2}{L_x}} \sin(k_n^x (x + \frac{L_x}{2}))\\
\psi_m(y) = \sqrt{\frac{2}{L_y}} \sin(k_m^y y) \\
\end{cases}
\]

\todo{спектральное разложение функции Грина}

Функция Грина для одномерной потенциальной ямы $[a; b]$ имеет замкнутый вид:
% (from ex\_8370\_sol\_Y11.pdf):
% тут вроде все ок со знаком и константой

$$G_{1D}(x, s; E) = - 2 \begin{cases}
\frac{\sin(k(x - a)) \sin(k(s - b))}{k \sin(k(b - a))}, & x < s \\
\frac{\sin(k(x - b)) \sin(k(s - a))}{k \sin(k(b - a))}, & x > s \\
\end{cases}$$

Можно использовать этот замкнутый вид, чтобы упростить функцию Грина для резонатора:

\begin{align*}
G_R(x, y, x_s, y_s; E)
&= \sum\limits_{n, m = 1}^\infty \frac{\psi_{nm}(x, y) \psi^*_{nm}(x_s, y_s)}{E_{nm} - E} \\
&= \sum\limits_n \sum\limits_m \frac{\psi_n(x) \psi_n^*(x_s) \psi_m(y) \psi_m^*(y_s)}{E^x_n + E^y_m - E} \\
&= \sum\limits_n \psi_n(x) \psi_n^*(x_s) \sum\limits_m \frac{\psi_m(y) \psi_m^*(y_s)}{E^y_m - (E - E^x_n)} \\
&= \sum\limits_n \psi_n(x) \psi^*_n(x_s) G^y_{1D}(y, y_s; E - E^x_n)
\end{align*}

\subsection{Волновод}


\section{Построение модели с отверстием нулевой ширины}
Сузим оператор $H = H_R \oplus H_W$ до множества гладких функций, удовлетворяющих условию Дирихле на границах $\Gamma_R$ и $\Gamma_W$, они автоматически зануляются в точке $(x_0, y_0)$, так как она находится на границе.

Однако, у данного оператора индексы дефекта $(0, 0)$, то есть он существенно самосопряжен. Рассмотрим причину этого:

\todo{найти правильную формулу}

Мультипольное разложение элементов $\dom H_0^*$:

\[
f(x, y) = \sum\limits_{i = 0}^\infty c_i d^i \pdv[i]{G}{n_s} (x, y, x_0, y_0; E_0)
\]

, где $d$ — радиус отверстия.

Так как функция Грина удовлетворяет тем же граничным условиям, то нулевое слагаемое будет равно нулю. Следующий кандидат на дефектный элемент — первая производная функции Грина. Однако, она не принадлежит $\mcL^2(\Omega)$, что и является причиной зануления индексов дефекта.

Чтобы получить расширение, необходимо работать в более широком пространстве функций, в котором лежит первая производная функции Грина. Более того, для того, чтобы дать интерпретацию формальному условию зануления граничной формы (которое определяет возможный выбор домена расширения)

\[
J(f, g) = \ip{H_0^* f}{g} - \ip{f}{H_0^* g}
\]

, элементы вида $H_0^* f$ также должны лежать в расширенном пространстве, и должно быть переопределено скалярное произведение в этом пространстве.

Таким образом, мы переходим от гильбертова пространства $\mcL^2(\Omega_R) \oplus \mcL^2(\Omega_W)$ со скалярным произведением $\ip{}{} = \ip{}{}_{\Omega_R} + \ip{}{}_{\Omega_W}$ к расширенному понтрягинскому пространству $\Pi$ с новым скалярным произведением $\ip{}{}_\Pi$, которое будет содержать необходимые для расчетов функции и в котором будет корректно определено скалярное произведение на них.

% Гамильтониан в понтрягинском пространстве Dym 119 Albeverio 155

% Домен сопряженного оператора состоит из элементов вида

% \[
% \dom \Delta^* = \begin{cases}
% a_R \pdv{}{n_0} G_R(x, y, x_0, y_0; E_0) + u_R(x), & x \in \Omega_R \\
% a_W \pdv{}{n_0} G_W(x, y, x_0, y_0; E_0) + u_W(x), & x \in \Omega_W
% \end{cases}
% \]

% Хотелось бы воспользоваться формулой Грина для многомерного интегрирования по частям: \todo{сопряжение}
% \[
% \int\limits_\Omega (f \laplacian g - g \laplacian f) d \Omega = \oint\limits_\Gamma \left( f \pdv{g}{n} - g \pdv{f}{n} \right) d \Gamma
% \]


\subsection{Расчет дефектного элемента для резонатора}
Дефектный элемент — производная функции Грина по нормали в точке $x_s = x_0, y_s = y_0$, то есть для резонатора совпадает с направлением производной по $y$.

Функция Грина:

\begin{align*}
G(x, y, x_s, y_s; E) = N_x^2 N_y^2 \sum\limits_{n = 1, m = 1}^\infty \frac{\sin(k_n x + \frac{\pi n}{2}) \sin(k_n x_s  + \frac{\pi n}{2}) \sin(k_m y) \sin(k_m y_s)}{k_n^2 + k_m^2 - E}
\end{align*}

\begin{align*}
d(x, y)
&= N_x^2 N_y^2 \sum\limits_{n = 1, m = 1}^\infty \frac{\sin(k_n x + \frac{\pi n}{2}) \sin(\frac{\pi n}{2}) \sin(k_m y) k_m}{E_{n, m} - E} \\
& = N_x^2 N_y^2 \sum\limits_{n = 1,2, m = 1}^\infty \frac{\cos(k_n x) \sin(k_m y) k_m}{E_{n, m} - E}
\end{align*}

\begin{align*}
\psi(x, y)
& = H_0^* d(x, y) = - \laplacian d(x, y) \\
& = N_x^2 N_y^2 \sum\limits_{n = 1,2, m = 1}^\infty \frac{(k_n^2 + k_m^2) \cos(k_n x) \sin(k_m y) k_m}{E_{n, m} - E}
\end{align*}

\subsection{Построение предпонтрягинского пространства}
В качестве оператора, на котором будет построена шкала, выберем резольвенту $Z_0 = R_0(\iu S) = \frac{1}{H_0 - \iu S}$, где $S$ — произвольное ненулевое вещественное число. $\iu S$ лежит в резольвентном множестве оператора $H_0$ (его спектр — неотрицательные вещественные числа), поэтому оператор $Z_0$ корректно определен. Структура шкал не зависит от выбора $S$, поэтому пока оставим его без конкретного значения, его можно будет зафиксировать потом с учетом возможных упрощений вычислений.


Рассмотрим применение оператора $Z_0$:

\begin{align*}
& (Z_0 \psi)(x, y) \\
&= \int\limits_{\Omega_R} \sum\limits_{n = 1, m = 1}^\infty N_x^2 N_y^2 \frac{\sin(k_n x + \frac{\pi n}{2}) \sin(k_n x' + \frac{\pi n}{2}) \sin(k_m y) \sin(k_m y')}{k^2_n + k^2_m - \iu S} \\
& N_x^2 N_y^2 \sum\limits_{n' = 1, m' = 1}^\infty \frac{(k_{n'}^2 + k_{m'}^2) \sin(k_{n'} x' + \frac{\pi n'}{2}) \sin(\frac{\pi n'}{2}) \sin(k_{m'} y') k_{m'}}{k_{n'}^2 + k_{m'}^2 - E_D} \\
&= N_x^2 N_y^2 \sum\limits_{n = 1, m = 1}^\infty \frac{(k_n^2 + k_m^2) \sin(k_n x + \frac{\pi n}{2}) \sin(\frac{\pi n}{2}) \sin(k_m y) k_m}{(k_n^2 + k_m^2 - \iu S) (k_n^2 + k_m^2 - E_D)}
\end{align*}

Легко заметить, что 

\begin{align*}
(Z_0^t \psi)(x, y)
&= N_x^2 N_y^2 \sum\limits_{n = 1, m = 1}^\infty \frac{(k_n^2 + k_m^2) \sin(k_n x + \frac{\pi n}{2}) \sin(\frac{\pi n}{2}) \sin(k_m y) k_m}{(k_n^2 + k_m^2 - \iu S)^t (k_n^2 + k_m^2 - E_D)} \\
&= N_x^2 N_y^2 \sum\limits_{n = 1,2, m = 1}^\infty \frac{(k_n^2 + k_m^2) \cos(k_n x) \sin(k_m y) k_m}{(k_n^2 + k_m^2 - \iu S)^t (k_n^2 + k_m^2 - E_D)}
\end{align*}


Определим порядок предпонтрягинского пространства — такое минимальное $t$, что $Z_0^t \psi \in \hilb{0}$, то есть что $Z_0^t \psi$ квадратично интегрируема на $\Omega_R$.

\begin{align*}
& \ip{Z_0^t \psi}{Z_0^t \psi} = \int\limits_{\Omega_R} \cconj{Z_0^t \psi(x, y))} Z_0^t \psi(x, y) \dd{x} \dd{y} \\
&\propto \sum\limits_{n = 1,2, m = 1}^\infty \frac{(n^2 + m^2) m}{\cconj{(n^2 + m^2 - \iu S)^t (n^2 + m^2 - E_D)}} \frac{(n^2 + m^2) m}{(n^2 + m^2 - \iu S)^t (n^2 + m^2 - E_D)} \\
&= \sum\limits_{n = 1,2, m = 1}^\infty \frac{(n^2 + m^2)^2 m^2}{((n^2 + m^2)^{2t} + S^2) ((n^2 + m^2)^{2} - |E_D|^2)} 
\end{align*}

Для исследования сходимости данного ряда воспользуемся интегральным признаком сходимости и перейдем в полярные координаты:
\begin{align*}
\int\limits_{r = 1}^\infty \int\limits_{\phi = 0}^{\pi / 2} \frac{r^4 r^2 \sin^2 \phi}{(r^{4t} + S^2) (r^{4} - |E_D|^2)} r \dd{r} \dd{\phi}
= \int\limits_{r = 1}^\infty \frac{r^7}{(r^{4t} + S^2) (r^{4} - |E_D|^2)} \dd{r}
\end{align*}

Минимальное $t$, для которого интеграл конечен, равно $2$, значит, $\psi \in \hilb{-2}$. Пространство $\mcP$ состоит из:
\[
f(x, y) = f_\phi(x, y) + f_{1} (Z_0^3 \psi)(x, y) + f_{0} (Z_0^2 \psi)(x, y) + f_{-1} (Z_0 \psi)(x, y) + f_{-2} \psi(x, y)
\]

Матрица $G$ для скалярного произведения:

\kbordermatrix{
   & 1                        &       0 &      -1 & -2      \\
1  & \ip{}{}_0 & \ip{}{}_0 & \ip{}{}_0 & \ip{}{}_0 \\
0  & \ip{}{}_0 & \ip{}{}_0 & \ip{}{}_0 & G_{0, -2} \\
-1 & \ip{}{}_0 & \ip{}{}_0 & G_{-1,-1} & G_{-1, -2} \\
-2 & \ip{}{}_0 & G_{-2,0} & G_{-2,-1} & G_{-2,-2} \\
}

Имеем ограничения:

\begin{ilist}
# $G_{-2,0} = \cconj{G_{0, -2}}$
# $G_{-2,-1} = \cconj{G_{-1, -2}}$
# $\Im G_{-2,-2} = 0$
# $\Im G_{-1,-1} = 0$
# Ограничение на $G_{-2, 0}$ и $G_{-1, -1}$
# Ограничение на $G_{-2, -1}$ и $G_{-1, -1}$
\end{ilist}

\todo{После этого вроде как остается два параметра? Один из них относится к $G_{-2, -2}$ и вроде как не нужен, а другой подгонять?}


Напомним, что нас будет интересовать скалярное произведение $\ip{\Delta f}{g}$. Надо вложить в понтгрягинское пространство:

\begin{elist}
# $\Delta_0 d$

Пользуясь тем, что по определению, $\psi = \Delta_0 d$, получаем
\[
\Delta_0 d \xmapsto[]{\mcP} (0, 0, 0, 0, 1) \xmapsto[]{\Pi} (0; 0, 0; 0, 1)
\]
# $d$

Заметим:
\begin{align*}
& \psi_{-1} = Z_0 \psi = \frac{1}{\Delta_0 - z_0} \Delta_0 d = d + z_0 Z_0 d \in \hilb{-1} \\
& z_0 \psi_{0} = z_0 Z_0 d + z_0^2 Z_0^2 d \in \hilb{0} \\
& z_0^2 \psi_{1} = z_0^2 Z_0^2 d + z_0^3 Z_0^3 d \in \hilb{1} \\
& f_\phi = z_0^3 Z_0^3 d \in \hilb{2}
\end{align*}

Поочередно вычитаем и прибавляем, получаем  $\psi_{-1} - z_0 \psi_0 + z_0^2 \psi_1 - f_\phi =  d$, соответственно:
\[
d \xmapsto[]{\mcP} (-z_0^3 Z_0^3 d, z_0^2, -z_0, 1, 0) \xmapsto[]{\Pi} (-z_0 Z_0 d; \left[ \ip{\psi_i}{-z_0^3 Z_0^3 d}_0 + \sum\limits_{j = 0}^1 f_j G_{i, j} \right]_{i = -1}^{-2};1, 0)
\]
# $u \in W_2^2$

Элемент пространства $W_2^2$ находится в $\hilb{2}$ \todo{точно?}, поэтому:
\[
u \xmapsto[]{\mcP} (u, 0, 0, 0, 0) \xmapsto[]{\Pi} (u;\left[ \ip{\psi_i}{u}_0 \right]_{i = -1}^{-2}; 0, 0)
\]
# $\Delta_0 u$

$\Delta_0 u \in \hilb{0}$ по определению пространства $W_2^2$. $\hilb{2}$ плотно в $\hilb{0}$, значит, существует последовательность $f_n \in \hilb{2}$: $\lim\limits_{n \to \infty} f_n = \Delta_0 u$.

\[f_n \xmapsto[]{\mcP} (f_n, 0, 0, 0, 0) \xmapsto[]{\Pi} (f_n, \left[ \ip{\psi_i}{f_n}_0 \right]_{i = -1}^{-2}, 0^2) \xrightarrow[n \to \infty]{\Pi} (\Delta_0 u; \left[ \ip{\psi_i}{\Delta_0 u}_0 \right]_{i = -1}^{-2}; 0, 0)\]
% Как это аналитически будем считать, вообще непонятно.
\end{elist}

% bordermatrix