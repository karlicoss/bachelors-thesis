\chapter{Описание реализованного подхода}
\label{chapter2}

\section{Модель двумерного волновода с прямоугольным резонатором}
\todo{Картиночку}

\todo{Про то, какой физический смысл этой модели}

Выберем начало координат в точке, в которой находится отверстие, то есть, $x_0 = 0, y_0 = 0$.

$\Omega_R$ — область резонатора, $\Omega_R = [- \frac{L_x}{2}, \frac{L_x}{2}] \times [0, L_y]$, где $L_x$ — ширина резонатора, $L_y$ — высота резонатора.

$\Omega_W$ — область волновода, $\Omega_W = [-\infty, \infty] \times [-H, 0]$, где $H$ — высота волновода.

\subsection{Построение модели с отверстием нулевой ширины}
Сузим оператор $H = H_R \oplus H_W$ до множества гладких функций, удовлетворяющих условию Дирихле на границах $\Gamma_R$ и $\Gamma_W$, они автоматически зануляются в точке $(x_0, y_0)$, так как она находится на границе.

Однако, у данного оператора индексы дефекта $(0, 0)$, то есть он существенно самосопряжен. Рассмотрим причину этого:

\todo{найти правильную формулу}

Мультипольное разложение элементов $\dom H_0^*$:

\[
f(x, y) = \sum\limits_{i = 0}^\infty c_i d^i \pdv[i]{G}{n_s} (x, y, x_0, y_0; E_0)
\]

, где $d$ — радиус отверстия.

Так как функция Грина удовлетворяет тем же граничным условиям, то нулевое слагаемое будет равно нулю. Следующий кандидат на дефектный элемент — первая производная функции Грина. Однако, она не принадлежит $\mcL^2(\Omega)$, что и является причиной зануления индексов дефекта.

Чтобы получить расширение, необходимо работать в более широком пространстве функций, в котором лежит первая производная функции Грина. Более того, для того, чтобы дать интерпретацию формальному условию зануления граничной формы (которое определяет возможный выбор домена расширения)

\[
J(f, g) = \ip{H^* f}{g} - \ip{f}{H^* g}
\]

, элементы вида $H_0^* f$ также должны лежать в расширенном пространстве, и должно быть переопределено скалярное произведение в этом пространстве.

% Домен сопряженного оператора состоит из элементов вида

% \[
% \dom \Delta^* = \begin{cases}
% a_R \pdv{}{n_0} G_R(x, y, x_0, y_0; E_0) + u_R(x), & x \in \Omega_R \\
% a_W \pdv{}{n_0} G_W(x, y, x_0, y_0; E_0) + u_W(x), & x \in \Omega_W
% \end{cases}
% \]

% Хотелось бы воспользоваться формулой Грина для многомерного интегрирования по частям: \todo{сопряжение}
% \[
% \int\limits_\Omega (f \laplacian g - g \laplacian f) d \Omega = \oint\limits_\Gamma \left( f \pdv{g}{n} - g \pdv{f}{n} \right) d \Gamma
% \]

% Однако, возникает проблема, связанная с тем, что формула Грина применима только для гладких функций, а функции в $\dom \Delta^*$ могут иметь особенность в точке $(x_0, y_0)$.




\subsection{Дефектные элементы}
Пусть направление нормали к границе между областями совпадает с осью $y$, тогда: дефектный элемент $d(x, y) = \eval{\pdv{G(x, y, x_s, y_s; E_D)}{y_s}}_{x_s = x_0, y_s = y_0}$.

Далее понадобится действовать на этот дефектный элемент резольвентой, рассмотрим это действие. Напомним, что резольвента — интегральное преобразование, ядро которого — функция Грина:

$$(R(E) \psi)(x, y) = \int\limits_\Omega G(x', y', x, y) \psi(x', y') \dd{x'} \dd{y'}$$

Подействуем резольвентой $R(E_0)$ на $d(x, y)$:

\begin{align*}
(R(E_0) d)(x, y)
&= \int\limits_\Omega \sum\limits_{n, m} \frac{\psi^x_n(x') \psi^x_n(x) \psi^y_m(y') \psi^y_m(y)}{E^x_n + E^y_m - E_0} \sum\limits_{n, m} \frac{\psi_n^x(x') \psi_n^x(x_0) \psi_m^y(y') \pdv{\psi_m^y}{y_s} (y_0)}{E^x_n + E^y_m - E_D} \\
&= \sum\limits_{n, m} \frac{\psi_n^x(x) \psi_n^x(x_0) \psi_m^y(y) \pdv{\psi_m^y}{y_s} (y_0)}{(E^x_n + E^y_m - E_0)(E^x_n + E^y_m - E_D)} \\
\end{align*}

Легко заметить, что 

\[
(R^t(E_0) d)(x, y)
= \sum\limits_{n, m} \frac{\psi_n^x(x) \psi_n^x(x_0) \psi_m^y(y) \pdv{\psi_m^y}{y_s} (y_0)}{(E^x_n + E^y_m - E_0)^t(E^x_n + E^y_m - E_D)}
\]

\begin{align*}
& (R(E_0) d)(x, y) = \int\limits_\Omega \\
& \left(
\sum\limits_{n = 1}^\infty
\sqrt{\frac{2}{L_x}} \cos(\frac{\pi n}{L_x} x') \sqrt{\frac{2}{L_x}} \cos(\frac{\pi n}{L_x} x)
\left(-2\frac{\sin(kk_n(y' - L_y)) \sin(kk_n y)}{kk_n \sin(kk_n L_y)}\right)
\right) \\
& \left(
\sum\limits_{n = 1, n += 2}^\infty
\sqrt{\frac{2}{L_x}} \cos(\frac{\pi n}{L_x} x') \sqrt{\frac{2}{L_x}}
\left(-2\frac{\sin(kk_n(y' - L_y))}{\sin(kk_n L_y)}\right)
\right) = \\
& \sum\limits_{n = 1, n += 2}^\infty
\sqrt{\frac{2}{L_x}} \cos(\frac{\pi n}{L_x} x) \sqrt{\frac{2}{L_x}}\\
& \int\limits_y  \left(-2\frac{\sin(kk_n(y' - L_y)) \sin(kk_n y)}{kk_n \sin(kk_n L_y)}\right)
\left(-2\frac{\sin(kk_n(y' - L_y))}{\sin(kk_n L_y)}\right) = \\
& \sum\limits_{n = 1, n += 2}^\infty
\sqrt{\frac{2}{L_x}} \cos(\frac{\pi n}{L_x} x) \sqrt{\frac{2}{L_x}} 4 \frac{\sin(kk_n y) }{kk_n \sin^2(kk_n L_y)}\int\limits_y  \sin^2(kk_n(y' - L_y)) = \\
& \sum\limits_{n = 1, n += 2}^\infty
\sqrt{\frac{2}{L_x}} \cos(\frac{\pi n}{L_x} x) \sqrt{\frac{2}{L_x}} 4 \frac{\sin(kk_n y) }{kk_n \sin^2(kk_n L_y)}
\left( \frac{1}{2} L_y - \frac{\sin(2 kk_n L_y)}{4 kk_n} \right)
\end{align*}

\subsection{Определение порядка понтрягинского пространства}
\todo{а что, надо чтобы элемент прямой суммы был в $\Pi$?}

Напомним, что нам необходимо, чтобы $\Pi$ содержало дефектный элемент $d$ и $H_0^* d$.

В качестве оператора, на котором будет построена шкала, выберем резольвенту $Z_0 = R_0(z_0) = \frac{1}{H_0 - z_0 I}$, где $z_0$ — произвольный элемент резольвентного множества оператора $H_0$ (конкретный его выбор не влияет на построение шкалы).

В качестве функции $\psi$, на которой будет построено предпонтрягинское пространство, возьмем $\psi = H_0^* d$.

\todo{Короче, знаем, что в этом случае оно лежит в $\hilb{-2}$}.

\todo{Можно ли в общем случае сказать, что в $\bbR^2$ производная функции Грина на первом элементе шкалы лежит?}

Из построения шкал видно, что $d \in \hilb{-1}$.

Пространство $\mcP$ состоит из:
\[
f(x, y) = f_\phi(x, y) + f_{1} (Z_0^3 \psi)(x, y) + f_{0} (Z_0^2 \psi)(x, y) + f_{-1} (Z_0 \psi)(x, y) + f_{-2} \psi(x, y)
\]

Далее перейдем в импульсное представление. В нем:

\begin{ilist}
# $H_0 = p^2 - E$
# $Z_0 = \frac{1}{p^2 - z_0}$
# $\psi(x, y) \mapsto \Psi(p_x, p_y)$
\end{ilist}

Импульсное представление удобно тем, что гамильтониан и резольвента в нем являются операторами умножения и коммутируют с применениями к ним функции.

Обозначим $\Psi_{i} = Z_0^{2 + i} \Psi$, то есть $\Psi_{i} \in \hilb{i}$.

Рассмотрим скалярное произведение $\ip{\Psi_i}{\Psi_j}$ в $L^2(\Omega)$:

\begin{align*}
\ip{\Psi_i}{\Psi_j} = \int\limits_\Omega \frac{1}{(p^2 - \cconj{z_0})^i} \cconj{\Psi(\vb{p})} \frac{1}{(p^2 - z_0)^j} \Psi(\vb{p}) \dd{\vb{p}}
\end{align*}


% Рассмотрим действие оператора на дефектный элемент:

% \todo{Пока без коэффициентов вообще. Потом разберусь.}

% \begin{align*}
% & (-\laplacian - E) \pdv{G}{n} (x, y, x_0, y_0; E_0) \\
% & = \pdv{}{n} \left((-\laplacian - E) G(x, y, x_0, y_0; E_0)\right) \\
% & = \pdv{}{n} \left(((-\laplacian - E_0) + (E_0 - E)) G(x, y, x_0, y_0; E_0)\right) \\
% & = \pdv{}{n} \left( - \delta(x - x_0) \delta(y - y_0) + (E_0 - E) G(x, y, x_0, y_0; E_0)\right) \\
% & = - \pdv{}{n} \delta(x - x_0) \delta(y - y_0) + (E_0 - E) \pdv{G}{n} (x, y, x_0, y_0; E_0)
% \end{align*}

% Таким образом, необходимо, чтобы кроме исходного дефектного элемента, в расширенном пространстве $\Pi$ оказалась производная дельта-функции Дирака.





\subsection{Расчет дефектного элемента для резонатора}
Дефектный элемент — производная функции Грина по нормали в точке $x_s = x_0, y_s = y_0$, то есть для резонатора совпадает с направлением производной по $y$.

Функция Грина:

\begin{align*}
G(x, y, x_s, y_s; E) = N_x^2 N_y^2 \sum\limits_{n = 1, m = 1}^\infty \frac{\sin(k_n x + \frac{\pi n}{2}) \sin(k_n x_s  + \frac{\pi n}{2}) \sin(k_m y) \sin(k_m y_s)}{k_n^2 + k_m^2 - E}
\end{align*}

\begin{align*}
d(x, y)
&= N_x^2 N_y^2 \sum\limits_{n = 1, m = 1}^\infty \frac{\sin(k_n x + \frac{\pi n}{2}) \sin(\frac{\pi n}{2}) \sin(k_m y) k_m}{E_{n, m} - E} \\
& = N_x^2 N_y^2 \sum\limits_{n = 1,2, m = 1}^\infty \frac{\cos(k_n x) \sin(k_m y) k_m}{E_{n, m} - E}
\end{align*}

\begin{align*}
\psi(x, y)
& = H_0^* d(x, y) = - \laplacian d(x, y) \\
& = N_x^2 N_y^2 \sum\limits_{n = 1,2, m = 1}^\infty \frac{(k_n^2 + k_m^2) \cos(k_n x) \sin(k_m y) k_m}{E_{n, m} - E}
\end{align*}

Рассмотрим применение резольвенты $Z_0 = R(E_0)$:

\begin{align*}
& (Z_0 \psi)(x, y) \\
&= \int\limits_{\Omega_R} \sum\limits_{n = 1, m = 1}^\infty N_x^2 N_y^2 \frac{\sin(k_n x + \frac{\pi n}{2}) \sin(k_n x' + \frac{\pi n}{2}) \sin(k_m y) \sin(k_m y')}{k^2_n + k^2_m - E_0} \\
& N_x^2 N_y^2 \sum\limits_{n' = 1, m' = 1}^\infty \frac{(k_{n'}^2 + k_{m'}^2) \sin(k_{n'} x' + \frac{\pi n'}{2}) \sin(\frac{\pi n'}{2}) \sin(k_{m'} y') k_{m'}}{k_{n'}^2 + k_{m'}^2 - E_D} \\
&= N_x^2 N_y^2 \sum\limits_{n = 1, m = 1}^\infty \frac{(k_n^2 + k_m^2) \sin(k_n x + \frac{\pi n}{2}) \sin(\frac{\pi n}{2}) \sin(k_m y) k_m}{(k_n^2 + k_m^2 - E_0) (k_n^2 + k_m^2 - E_D)}
\end{align*}

Легко заметить, что 

\begin{align*}
(Z_0^t \psi)(x, y)
&= N_x^2 N_y^2 \sum\limits_{n = 1, m = 1}^\infty \frac{(k_n^2 + k_m^2) \sin(k_n x + \frac{\pi n}{2}) \sin(\frac{\pi n}{2}) \sin(k_m y) k_m}{(k_n^2 + k_m^2 - E_0)^t (k_n^2 + k_m^2 - E_D)} \\
&= N_x^2 N_y^2 \sum\limits_{n = 1,2, m = 1}^\infty \frac{(k_n^2 + k_m^2) \cos(k_n x) \sin(k_m y) k_m}{(k_n^2 + k_m^2 - E_0)^t (k_n^2 + k_m^2 - E_D)}
\end{align*}

Найдем минимальное $t$, что $Z_0^t \psi \in \hilb{0}$, то есть что $Z_0^t \psi$ квадратично интегрируема на $\Omega_R$. \todo{Опускаем константы в интегрировании}

\begin{align*}
& \ip{Z_0^t \psi}{Z_0^t \psi} = \int\limits_{\Omega_R} \cconj{Z_0^t \psi(x, y))} Z_0^t \psi(x, y) \dd{x} \dd{y} \\
&= \sum\limits_{n = 1,2, m = 1}^\infty \frac{(n^2 + m^2) m}{\cconj{(n^2 + m^2 - E_0)^t (n^2 + m^2 - E_D)}} \frac{(n^2 + m^2) m}{(n^2 + m^2 - E_0)^t (n^2 + m^2 - E_D)} \\
&= \sum\limits_{n = 1,2, m = 1}^\infty \frac{(n^2 + m^2)^2 m^2}{((n^2 + m^2)^{2t} - |E_0|^2) ((n^2 + m^2)^{2} - |E_D|^2)} 
\end{align*}

Воспользуемся интегральным признаком сходимости и перейдем в полярные координаты:

\begin{align*}
\int\limits_{r = 0}^\infty \int\limits_{\phi = 0}^{\pi / 2} \frac{r^4 r^2 \sin^2 \phi}{(r^{4t} - |E_0|^2) (r^{4} - |E_D|^2)} r \dd{r} \dd{\phi}
= \int\limits_{r = 0}^\infty \frac{r^7}{(r^{4t} - |E_0|^2) (r^{4} - |E_D|^2)} \dd{r}
\end{align*}

Минимальное $t$, для которого интеграл конечен, равно $2$, значит, $\psi \in \hilb{-2}$.


% \begin{align*}
% & D(x, y) = \mcF(d(x, y)) = \int\limits_{\Omega_R} N_x^2 N_y^2 \sum\limits_{n = 1,2, m = 1}^\infty \frac{\cos(k_x x) \sin(k_m y) k_m}{E_{n, m} - E} e^{-i p_x x} e^{-i p_y y} \dd{x} \dd{y} \\
% & = N_x^2 N_y^2 \sum\limits_{n = 1,2, m = 1}^\infty \frac{1}{E_{n, m} - E} k_m \int\limits_{\Omega_R} \cos(\frac{\pi n}{L_x} x) \sin(\frac{\pi m}{L_y} y) e^{-i p_x x} e^{-i p_y y} \dd{x} \dd{y} \\
% & = N_x^2 N_y^2 \sum\limits_{n = 1,2, m = 1}^\infty \frac{1}{E_{n, m} - E} \frac{\pi m}{L_y} \\
% & \frac{2 \pi  m L_x L_y e^{-i L_y p_y} \left(-(-1)^m+e^{i L_y p_y}\right) \left(\pi  n \sin \left(\frac{\pi  n}{2}\right) \cos \left(\frac{L_x p_x}{2}\right)-L_x \cos \left(\frac{\pi  n}{2}\right) p_x \sin \left(\frac{L_x p_x}{2}\right)\right)}{\left(\pi ^2 m^2-L_y^2 p_y^2\right) \left(\pi ^2 n^2-L_x^2 p_x^2\right)}
% \end{align*}

Уравнение Шредингера с условием Дирихле на границе
\begin{align*}
- \frac{1}{2}\laplacian \psi(x, y) &= E \psi(x, y) \\
\eval{\psi}_{\Gamma_R}{} &= 0
\end{align*}
, допускает разделение переменных в области $\Omega_R$. Разделив переменные, получим решения:

\[
\begin{cases}
E_{n,m} = E_n^x + E_m^y \\
\psi_{n, m}(x, y) = \psi_n(x) \psi_m(y) \\
k_n^x = \frac{\pi n}{L_x}, n = 1 \dots \infty \\
k_m^y = \frac{\pi m}{L_y}, m = 1 \dots \infty \\
E_n^x = \frac{1}{2} (k_n^x)^2 \\
E_m^y = \frac{1}{2} (k_m^y)^2 \\
\psi_n(x) = \sqrt{\frac{2}{L_x}} \sin(k_n^x (x + \frac{L_x}{2}))\\
\psi_m(y) = \sqrt{\frac{2}{L_y}} \sin(k_m^y y) \\
\end{cases}
\]

Функция Грина для одномерной потенциальной ямы $[a; b]$ имеет замкнутый вид:
% (from ex\_8370\_sol\_Y11.pdf):
% тут вроде все ок со знаком и константой

$$G_{1D}(x, s; E) = - 2 \begin{cases}
\frac{\sin(k(x - a)) \sin(k(s - b))}{k \sin(k(b - a))}, & x < s \\
\frac{\sin(k(x - b)) \sin(k(s - a))}{k \sin(k(b - a))}, & x > s \\
\end{cases}$$

Можно использовать этот замкнутый вид, чтобы упростить функцию Грина для резонатора:

\begin{align*}
G_R(x, y, x_s, y_s; E)
&= \sum\limits_{n, m = 1}^\infty \frac{\psi_{nm}(x, y) \psi^*_{nm}(x_s, y_s)}{E_{nm} - E} \\
&= \sum\limits_n \sum\limits_m \frac{\psi_n(x) \psi_n^*(x_s) \psi_m(y) \psi_m^*(y_s)}{E^x_n + E^y_m - E} \\
&= \sum\limits_n \psi_n(x) \psi_n^*(x_s) \sum\limits_m \frac{\psi_m(y) \psi_m^*(y_s)}{E^y_m - (E - E^x_n)} \\
&= \sum\limits_n \psi_n(x) \psi^*_n(x_s) G^y_{1D}(y, y_s; E - E^x_n)
\end{align*}

Дефектный элемент $d(x, y)$ — производная функции Грина по нормали к границе в точке отверстия, то есть:

\begin{align*}
& d(x, y) = \eval{\pdv{G_R(x, y, x_s, y_s; E)}{y_s}}_{x_s = \frac{L_x}{2}, y_s = 0} \\
&= \sum\limits_{n = 1}^\infty
\sqrt{\frac{2}{L_x}} \sin(k_n^x (x + \frac{L_x}{2}))
\sqrt{\frac{2}{L_x}} \sin(k_n^x (x_s + \frac{L_x}{2}))
\eval{\pdv{G_{1D}(y, y_s; E - E^x_n)}{y_s}}_{y_s = 0} \\
&= \frac{2}{L_x} \sum\limits_{n = 1}^\infty
\sin(\frac{\pi n}{L_x} x + \frac{\pi}{2} n)
\sin(\frac{\pi}{2} n)
\eval{\pdv{G_{1D}(y, y_s; E - E^x_n)}{y_s}}_{y_s = 0} \\
&= \frac{2}{L_x} \sum\limits_{n = 1, n += 2}^\infty
\cos(\frac{\pi n}{L_x} x)
\left(-2\frac{\sin(kk_n(y - L_y)) kk_n \cos(kk_n y_s)}{kk_n \sin(kk_n L_y)}\right) \\
&= \frac{2}{L_x} \sum\limits_{n = 1, n += 2}^\infty
\cos(\frac{\pi n}{L_x} x)
\left(-2\frac{\sin(kk_n(y - L_y))}{\sin(kk_n L_y)}\right)
\end{align*}

Дефектный элемент не лежит в $L^2(\Omega_R)$:

\begin{align*}
& \int\limits_{x = -\frac{L_x}{2}}^{\frac{L_x}{2}} \int\limits_{y = 0}^{L_y} D(x, y)^2 \\
& = \frac{2}{L_x} \frac{2}{L_x} \frac{2}{L_y} \frac{2}{L_y}  \sum\limits_{n = 1, n += 2}^\infty
\int\limits_{x = -\frac{L_x}{2}}^{\frac{L_x}{2}} \cos^2(\frac{\pi n}{L_x} x) \int\limits_{y = 0}^{L_y}
\left( \sum\limits_{m = 1}^\infty \frac{\sin(\frac{\pi m}{L_y}y) \frac{\pi m}{L_y}}{E_n^x + E_m^y - E} \right)^2 \\
& = \frac{2}{L_x} \frac{2}{L_x} \frac{2}{L_y} \frac{2}{L_y}  \sum\limits_{n = 1, n += 2}^\infty
\int\limits_{x = -\frac{L_x}{2}}^{\frac{L_x}{2}} \cos^2(\frac{\pi n}{L_x} x)
\sum\limits_{m = 1}^\infty \left( \frac{\frac{\pi m}{L_y}}{E_n^x + E_m^y - E} \right)^2 \int\limits_{y = 0}^{L_y} \sin^2(\frac{\pi m}{L_y}y) \\
& = \frac{2}{L_x} \frac{2}{L_y}  \sum\limits_{n = 1, n += 2}^\infty
\sum\limits_{m = 1}^\infty \left( \frac{\frac{\pi m}{L_y}}{E_n^x + E_m^y - E} \right)^2
\end{align*}

Этот ряд не сходится. \todo{почему?}. Требуется расширение пространства до понтрягинского.

\subsubsection{Построение понтрягинского пространства}
Нужно построить шкалы. Построим ее на резольвентном операторе $R(E_0)$.

Рассмотрим функцию $(R(E_0) d)$, воспользовавшись \todo{TODO}:

\[
(R(E_0) d)(x, y)
= \sum\limits_{n = 1, n += 2}^\infty \sum\limits_{m = 1}^\infty \frac{\frac{2}{L_x} \cos(\frac{\pi n}{L_x} x) \frac{2}{L_y} \sin(\frac{\pi m}{L_y} y)\frac{\pi m}{L_y}}{(E^x_n + E^y_m - E_0)(E^x_n + E^y_m - E_d)} \\
\]

Этот элемент уже принадлежит $L^2(\Omega_R)$, \todo{интегральный признак?}, значит, дефектный элемент находится в $\hilb{-1}$.

Обозначим $d_k(x, y) = R^{1 + k}(E_0) d$, то есть $d_k \in \hilb{k}$, в такой нотации $d_{-1} = d$.

Таким образом, предпонтрягинское пространство $\mcP$ состоит из функций вида:

\[
f(x, y) = f_\phi(x, y) + f_0 (Z_0 d)(x, y) + f_{-1} d(x, y)
\]

Рассчитаем элементы матрицы $G$:

\begin{ilist}
# $G_{0, 0} = $ \todo{TODO все очень плохо. Нифига аналитически не считается}
% попробовать https://en.wikipedia.org/wiki/Barnes_zeta_function
# $G_{0, -1} = $ \todo{TODO}
# $G_{-1, 0} = \cconj{G_{0, -1}} = $ \todo{TODO}
# $G_{-1, -1} = $ — свободный параметр \todo{TODO}
\end{ilist}

Дальше:

\begin{elist}
# \todo{рассчитали элементы матрицы, есть скалярное произведение в $\mcP$ }
# \todo{получили производные функции Грина в $\Pi$ — дефектные элементы}
# \todo{сшили решения в $\Omega_W$ и $\Omega_R$ с условием нулевого потока через отверстие}
## \todo{а чтобы сшивать, надо знать асимптотики :(}
# \todo{посчитали поток в асимптотическом регионе и получили коэффициент прохождения}
\end{elist}


Домен сопряженного оператора $\Delta^*$ состоит из элементов вида:

\[
u(x) = \begin{cases}
a_W \pdv{G_W(x, y, x_s, y_s)}{y_s} (x_s = x_0, x_s = y_0) + b_W, & x \in \Omega_W \\
a_R \pdv{G_R(x, y, x_s, y_s)}{y_s} (x_s = x_0, x_s = y_0) + b_R, & x \in \Omega_R \\
\end{cases}
\]

Домен самосоряженного расширения $\Delta_E$ исходного оператора $\Delta$ — линейное подмножество, на котором зануляется граничная форма. \todo{написать про возможные расширения}

\todo{использовать асимптотику производной. Не уверен, что расширение то же самое будет}

Воспользуемся расширением с «нулевым потоком через отверстие»:

\begin{ilist}
# $a_W = -a_R$
# $b_W = b_R$
\end{ilist}

\subsection{Волновод}







Функция Грина для 2D: $G_0(\vb{x}) = \frac{i}{4} H_0^{(1)}(k |\vb{x}|)$



% Пусть $\frac{\hbar^2}{2 \mu} = 1$

$\psi(x_1, x_2) = \delta'(x_1) \delta(x_2)$

Далее будет удобно перейти в импульсное представление, так как в нем гамильтониан будет оператором умножения: $H_0 = \frac{p^2}{2 \mu}$.

Резольвента, соответственно, также будет оператором умножения, $R_0(z) = \frac{1}{\frac{p^2}{2 \mu} - z}$, что позволит менять ее местами с символами функций в вычислениях.

Переведем функционал $\psi(x_1, x_2)$ в импульсное представление $\Psi(p_1, p_2)$:

$\Psi(p_1, p_2) = \mcF(\psi(x_1, x_2)) = \frac{1}{2 \pi \hbar} i p_1$

Также упрощается частичное скалярное произведение:

\begin{align*}
\ip{f}{g}_0
&= \ip{R_0^k(\cconj{z_0}) f}{R_0^{-k}(z_0) g} \\
&= \int\limits_{\bbR^2} \cconj{R_0^k(\cconj{z_0}) f(\vb{p})} R_0^{-k}(z_0) g(\vb{p}) \dd[2]{\vb{p}} \\
&= \int\limits_{\bbR^2} R_0^k(z_0) R_0^{-k}(z_0) f(\vb{p}) g(\vb{p}) \dd[2]{\vb{p}} \\
&= \int\limits_{\bbR^2} f(\vb{p}) g(\vb{p}) \dd[2]{\vb{p}} \\
&= \ip{f}{g}
\end{align*}

$\psi \notin \hilb{0}$, $L^2$-норма $\psi$ бесконечна

$\psi \notin \hilb{-1}$, $L^2$-норма $R_0(z_0) \psi$ бесконечна

$\psi \in \hilb{-2}$, $L^2$-норма $R_0^2(z_0) \psi$ конечна

Получаем предпонтрягинское пространство $\mcP_{-1}$, состоящее из функций вида:

\[
f(\vb{p}) = f_\phi(\vb{p}) + f_0 \frac{1}{(\frac{p^2}{2 \mu} - i)^2} \Psi(\vb{p}) + f_{-1} \frac{1}{\frac{p^2}{2 \mu} - i} \Psi(\vb{p})
\]

Скалярное произведение:

\begin{align*}
\ip{f}{g}
&= \ip{f_\phi}{g_\phi} \\
&+ \sum\limits_{i = -1}^0 \ip{f_i \Psi_i}{g_\phi}_0 + \sum\limits_{j = -1}^0 \ip{f_\phi}{g_j \Psi_j}_0 \\
&+ \sum\limits_{i = -1}^0 \sum \limits_{j = -1}^0 \cconj{f_i} G_{i,j} g_j 
\end{align*}

\begin{itemize}
\item $G_{0, 0} = \ip{\Psi_0}{\Psi_0} = \frac{\mu^2}{4 \pi \hbar^2}$
\item $G_{0, -1} = \ip{\Psi_0}{\Psi_1} = \frac{\mu^2 (2 i + \pi)}{8 \pi \hbar^2}$
\item $G_{-1, 0} = \cconj{G_{0, -1}}$
\item $G_{-1, -1}$ — свободный (вещественный) параметр
\end{itemize}

Свободный параметр $G_{-1, -1}$ определим так, чтобы оно было «конечной» частью расходящегося скалярного произведения $\ip{\Psi_{-1}}{\Psi_{-1}}_0$, следующим образом:

Далее будет удобно перейти в полярные координаты (так как $p^2$ встречается значительно чаще чем $p_1$ и $p_2$):

\begin{align*}
(p_1, p_2) &\to (p, \varphi) \\
p &\to p \\
p_1 &\to p \cos(\varphi) \\
p_2 &\to p \sin(\varphi) \\
\dd[2]{\vb{p}} &\to p \dd{p} \dd{\varphi}
\end{align*}

Рассмотрим интеграл в скалярном произведении $\ip{\cdot}{\cdot}_0$, на области с $p < N$:

\[
\int\limits_{\varphi = 0}^{2 \pi} \int\limits_{p = 0}^N
\cconj{\left(
\frac{1}{\frac{p^2}{2 \mu} + i} \frac{1}{\frac{p^2}{2 \mu} - i} \frac{1}{2 \pi \hbar} i p \cos \varphi
\right)}
\left(
\frac{1}{2 \pi \hbar} i p \cos \varphi
\right)
p \dd{p} \dd{\varphi}
=
\frac{\mu^2}{4 \pi \hbar^2} \log(1 + \frac{N^4}{4 \mu^2})
\]

TODO ссылка на Березина

Предел $\frac{1}{\log(N) \cdot} = \frac{\mu^2}{\pi \hbar^2}$.

Ну в общем, отнормированный интеграл:

\[
-\frac{\mu^2 \log (2 \mu)}{2 \pi  \hbar ^2}
\]

Ну ок.

TODO подставить в формулу Крейна и найти полюса резольвенты

Найдем $\Gamma^{-1}(z, \lambda)$. Заметим, что $R_0(z) \psi$ в общем случае не лежит в подпространстве, соответствующему $\psi_1$, поэтому надо выделить его коэффициенты $f_0, f_1$ и функцию $f_\phi$:

% Далее в нотации используется, что резольвента -- оператор умножения.

Далее будет полезно следущее тождество:

$R(z) = \frac{1}{H - z} = \frac{1}{H - z_0} + \frac{z - z_0}{(H - z)(H - z_0)} = \frac{1}{H - z_0} + \frac{z - z_0}{(H - z_0)^2} + \frac{(z - z_0)^2}{(H - z_0)^2 (H - z)}$

Используя тождество, получим компоненты элемента $R_0(z) \Psi$ в пространстве $\mcP_1$:

\begin{itemize}
\item $f_{-1} = 1$
\item $f_0 = z - i$
\item $f_\phi = R_0^2(i) R_0(z) (z - i)^2 \Psi$
\end{itemize}

% Разложим резольвенту в пространстве $\mcP_1$

% $R_0(z) \psi = \frac{1}{\frac{p^2}{2 \mu} - z} \frac{1}{2 \pi \hbar} i p_1 = f_\varphi + f_0 \frac{1}{\left(\frac{p^2}{2 \mu} - z_0 \right)^2} \psi + f_{-1} \frac{1}{\frac{p^2}{2 \mu} - z_0} \psi$

Замыкание $\mcP_1$ дает нам понтрягинское пространство $\Pi_1$


\section{3D}
Рассмотрим уравнение Шредингера в двух областях $\Omega_1$ и $\Omega_2$ с условиями Дирихле на границе, общей границей $\partial \Gamma_{12}$, и точечным проколом $\vb{a_{12}} \in \partial \Gamma_{12}$. TODO картиночку

Типичный пример такой задачи — резонатор Гельмгольца TODO

TODO как физически реализуемо?

TODO написать, что функции Грина не нужны, нужны производные по нормали по источнику

TODO написать, что производные функции Грина имеют неинтегрируемую особенность

Далее необходимо выполнить расширение пространства отдельно в каждой области $\Omega_1, \Omega_2$, а потом «сшить» решения с некоторым граничным условием в окрестности точечного прокола $\vb{a_{12}}$. Далее, выберем систему координат так, чтобы точечный прокол оказался 

Работаем в пространстве $\bbR^3$. Исходный гамильтониан: $H_0 = \frac{\hat{p}^2}{2 \mu}$ в $L_2(\bbR^3, \dd x)$.

% H_0 = - \frac{\hbar^2}{2 \mu } \laplacian

$\psi = \pdv{n} \delta(\vb{r})$.

Спектром оператора является $\bbR$, соответственно, резольвентное множество — $\{ c \mid \Im c \ne 0 \}$. В качестве $z_0$ возьмем, например, $i$, (напомним, что его выбор не влияет на структуру шкалы).


Рисунок 2. Нормаль направлена по координате $x_1$, $\psi = \delta'(x_1) \delta(x_2) \delta(x_3)$. Перейдем в импульсное представление:

В импульсном представлении гамильтониан $H_0 = \frac{p^2}{2 \mu}$, соответственно, $R_0(z_0) = \frac{1}{p^2 - z_0}$ TODO fix.

Переведем функционал $\psi$ импульсное предсавление:

TODO: $\hbar$
\begin{align*}
\Psi(p_1, p_2, p_3)
&= \mathcal{F}(\psi(x_1, x_2, x_3)) \\
&= \frac{1}{\sqrt{2 \pi \hbar}} \int\limits_{x_1} \frac{1}{\sqrt{2 \pi \hbar}} \int\limits_{x_2} \frac{1}{\sqrt{2 \pi \hbar}} \int\limits_{x_3} \psi(x_1, x_2, x_3) e^{-i \vb{x} \vdot \vb{p}} \\
&= \frac{1}{\sqrt{2 \pi \hbar}} \int\limits_{x_1} \delta'(x_1) e^{-i x_1 p_1} \frac{1}{\sqrt{2 \pi \hbar}} \int\limits_{x_2} \frac{1}{\sqrt{2 \pi \hbar}} \int\limits_{x_3} \delta(x_2) \delta(x_3) e^{-i x_2 p_2} e^{-i x_3 p_3} \\
&= \frac{1}{\sqrt{2 \pi \hbar}} \int\limits_{x_1} \delta'(x_1) e^{-i x_1 p_1} \mathcal{F}(\delta(x_2) \delta(x_3)) \\
&= \frac{1}{2 \pi \hbar} \mathcal{F}(\delta'(x_1)) \\
&= \frac{1}{2 \pi \hbar} i p_1 \mathcal{F}(\delta(x_1)) \\
&= \left(\frac{1}{\sqrt{2 \pi \hbar}}\right)^3 i p_1
\end{align*}

\begin{itemize}
\item $\Psi \notin \hilb{0}$ 
\item $\Psi \notin \hilb{-1}$
\item $\Psi \in \hilb{-2}$
\end{itemize}


Получаем предпонтрягинское пространство $\mcP_{-1}$, состоящее из функций вида:

\[
f(\vb{p}) = f_\phi(\vb{p}) + f_0 \frac{1}{(p^2 - i)^2} + f_{-1} \frac{1}{p^2 - i}
\]

Скалярное произведение:

\[
\ip{f}{g} = \ip{f_\phi}{g_\phi} + \dots + \cconj{f_0} G_{0,0} g_0 + \cconj{f_0} G_{0, -1} g_0 + \cconj{f_{-1}} G_{-1, 0} g_0 + \cconj{f_{-1}} G_{-1, -1} g_{-1}
\]

\begin{itemize}
\item $G_{0, 0}$ точно определено, так как $0 + 0 \ge 0$
\item $G_{0, -1} = \cconj{G_{-1, 0}} = ?$ — можно определить, так как можно показать, что $\psi_0 \in \hilb{1}$, а значит, скалярное произведение конечное
\item $G_{-1, -1}$ — свободный параметр
\end{itemize}

TODO пока мутно, как определять свободный параметр. Пусть определили.

\todo{А насколько маленькое может быть отверстие? Оно же в любом случае не меньше атома в диаметре?}