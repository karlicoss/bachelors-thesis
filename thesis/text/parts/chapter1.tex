\chapter{Обзор предметной области}
\label{chapter1}

\section{Аксиоматизация квантовой механики}

% http://thisquantumworld.com/wp/a-critique-of-quantum-mechanics/the-standard-axioms/

Существуют несколько математических формализаций аппарата квантовой механики, TODO

\begin{easylist}[itemize]
# с каждой физической системой ассоциировано сепарабельное гильбертово пространство $\hilb{}$ над полем комплексных чисел $\mathbb{C}$ со скалярным произведением $\ip{\cdot}{\cdot}$. Состояния квантовой системы — элементы $\hilb{}$, имеющие единичную норму, которые также называются \textit{волновыми функциями};
# каждая \textit{наблюдаемая} физическая величина $a$ (координата, импульс, спин, энергия, и т.д.) ассоциирована с самосопряженным оператором $A$ (возможно, неограниченным), плотно определенным в пространстве $\hilb{}$. Множество возможных значений наблюдаемой — её спектр $\sigma(A)$, который для самосопряженного оператора вещественнен и состоит из дискретной и непрерывной части;
# информация о наблюдаемой может быть получена проверкой принадлежности ее значения борелевскому множество $X \in \bbR$.
## вероятность значения наблюдаемой принадлежать $X$ рассчитывается как $\| \xi_A(X) \Psi \|^2 = \ip{\Psi}{\xi_A(X) \Psi}$
### в частности, математическое ожидание значения наблюдаемой $A$ в состоянии $\Psi$ равно $\ip{\Psi}{A \Psi}$.
### \textit{плотностью вероятности} волновой функции $\Psi$ называется функция $\rho(x) = \|\Psi(x)|^2$ \todo{соответствует единичному оператору? Что бы это значило?}
## после измерения состояние системы проецируется на подпространство, натянутое на собственные функции, соответствующие элементам спектра в $X$, то есть: $\Psi \mapsto \frac{\xi_A(X) \Psi}{\| \xi_A(X) \Psi \|}$;
# временная эволюция системы задается однопараметрической унитарной группой $U(t)$, генератором которой является гамильтониан $H$, то есть оператор полной энергии системы;
## в частности, $\Psi(t) = U(t) \Psi_0$
\end{easylist}

\section{Уравнение Шредингера}
Уравнение Шредингера описывает временную эволюцию чистых состояний нерелятивистсих квантовых систем во времени. Фактически, оно эквивалентно определению генератора группы $U(t)$, и имеет форму:
\[
i \hbar \pdv{t} \Psi(t) = \hat{H} \Psi(t)
\]
, где $i$ — мнимая единица, $\hbar$ — приведенная постоянная Планка, $\hat{H}$ — гамильтониан, оператор полной энергии системы, спектр представляет собой множество возможных значений энергии системы, которые можно получить при ее измерении.

Для каждой квантовой системы гамильтониан свой, и его форму надо «угадывать» по физическим экспериментам.

\subsection{Гамильтониан для частицы в потенциальном поле}
\[
\hat{H} = - \frac{\hbar^2}{2 m} \laplacian + V(\vb{r}, t)
\]

\subsection{Гамильтониан для заряженной частицы с нулевым спином в магнитном поле}
\[
\hat{H} = \frac{1}{2m} \left( -i \hbar \grad - q \vb{A} \right)^2 + q \phi
\]

\section{Ток вероятности}
Ток вероятности (англ. probability current) — величина, описывающая «течение» вероятности через единицу поверхности за единицу времени. \todo{мутно}

% TODO https://en.wikipedia.org/wiki/Probability_current#Spin-0_particle_in_an_electromagnetic_field

\subsection{Ток вероятности для частицы в потенциальном поле}
\[
\vb{j}(\vb{r}, t) = \frac{\hbar}{2 m i} (\psi(\vb{r}, t)^* \grad_r{\psi(\vb{r}, t)} - \psi(\vb{r}, t) \grad_r{\psi^*(\vb{r}, t))}
\]

\subsection{Ток вероятности для заряженной частицы с нулевым спином в магнитном поле}
\todo{TODO}


\section{Коэффициент прохождения}
Интуитивно, коэффициент прохождения — отношение «количества» волны, прошедшей через барьер к «количеству» волны, на этот барьер падающий. «Количество» волны должно быть связяно с током вероятности, однако он является векторная величина, и функция нескольких аргументов, тогда как хотелось бы получить характерное скалярное значение. Можно заметить, что:

\begin{easylist}[itemize]
# ток вероятности не зависит от времени для не изменяемой во времени конфигурации системы;
# в волноводе можно выделить направление распространения волны, по этому направлению идет основной поток, остальные либо зануляются из-за симметрии, либо ими можно пренебречь;
# в волноводе можно выделить сечение, по которому можно проинтегрировать ток вероятности по направлению распространения волны.
\end{easylist}

В результате этого получаем величины $J_{inc}$ и $J_{trans}$, которые характеризуют входящий и прошедний потоки соответственно. 

Коэффициент прохождения определяется как
\[
T = \frac{|J_{trans}|}{|j_{inc}|}
\]

\section{Фиксирование нотации}

\subsection{Цилиндрические координаты}

\[
\vb{r} = (r, \theta, z)
\]

\[
\hat {p_r} = - i \hbar \pdv{r};\quad 
\hat{p_\theta} = - i \hbar \frac{1}{r} \pdv{\theta};\quad 
\hat{p_z} = - i \hbar \pdv{z}
\]

$$\renewcommand\arraystretch{1.5}
\gradient f = 
\begin{pmatrix}
\pdv{f}{r} \\
\frac{1}{r} \pdv{f}{\theta}  \\
\pdv{f}{z} 
\end{pmatrix}
$$


$$
\divergence F =
\frac{1}{r} \pdv{(r F_r)}{r}
+ \frac{1}{r} \pdv{F_\theta}{\theta}
+ \pdv{F_z}{z}
$$


\subsection{Преобразование Фурье}
В задачах квантовой механики часто удобно и более естественно производить анализ в импульсном представлении, нежели чем в координатном. Они связаны между собой преобразованием Фурье:

\subsubsection{Из координатного представления в моментное}
Пусть имеется волновая функция $\psi: \bbR^n \to \bbC$. 

$\hat{p} = -i \hbar \grad$

$H_0 = - \frac{\hbar^2}{2 \mu} \laplacian$

$\Psi(\vb{p}) = \mcF(\psi(\vb{x})) = \left(\frac{1}{\sqrt{2 \pi \hbar}}\right)^n \int\limits_{\bbR^n} \psi(\vb{x}) e^{-\frac{i}{\hbar} \vb{p} \vdot \vb{x}} \dd^n \vb{x}$
TODO

\subsubsection{Из импульсного представления в координатное}

$H_0 = \frac{p^2}{2 \mu}$

TODO

\subsection{Функция Грина}
Термин ``функция Грина'' используется в двух значениях:

\begin{itemize}
\item как интегральное ядро резольвентного оператора: $G(x, s; \lambda)$ — ядро оператора $R(\lambda) = \frac{1}{\lambda I - L}$ \todo{а тут все правильно со знаком?}
\item как решение дифференциального уравнения $L_x G(x, s) = -\delta(x - s)$
\end{itemize}

Также в физике и математике сложилась небольшая неоднозначность в определении функции Грина:

\begin{itemize}
\item в физике: $L_x G(x, s) = -\delta(x - s)$;
\item в математике: $L_x G(x, s) = \delta(x - s)$.
\end{itemize} 

Мы будем использовать \textbf{физическую} нотацию в данной работе. В частности, это означает, что спектральное представление функции Грина выглядит как: 

\[
G(x, s) = -\sum\limits_n \frac{\psi_n(x) \psi_n^*(s)}{\lambda_n}
\]

\subsection{Скалярное произведение}
There is some ambiguity in defining antilinearity property of the inner product in vector space $V$ over the field $\mathbb{C}$ of complex numbers:

\begin{itemize}
\item Physics: \textbf{anti}linear in the \textit{first} argument, linear in the \textit{second} argument:
\[
\forall x, y, z \in V: \forall a, b \in \mathbb{C}: \ip{ax + by}{z} = \cconj{a} \ip{x}{z} + \cconj{b} \ip{y}{z}
\]
\[
\forall x, y, z \in V: \forall a, b \in \mathbb{C}: \ip{x}{ay + bz} = a \ip{x}{y} + b \ip{x}{y}
\]
\item Mathematics: linear in the \textit{first} argument, \textbf{anti}linear in the \textit{second} argument:
\[
\forall x, y, z \in V: \forall a, b \in \mathbb{C}: \ip{ax + by}{z} = a \ip{x}{z} + b \ip{y}{z}
\]
\[
\forall x, y, z \in V: \forall a, b \in \mathbb{C}: \ip{x}{ay + bz} = \cconj{a} \ip{x}{y} + \cconj{b} \ip{x}{y}
\]
\end{itemize}

We stick to the PHYSICS definition. For instance, that means that inner product in $L^2(E)$ is defined as $\ip{f}{g} = \int\limits_E \cconj{f(\vb{x})} g(\vb{x}) \dd \vb{x}$.

\subsection{Атомная система единиц}
\todo{Хотелось бы нормализовать массу электрона, заряд электрона, постоянную Планка и единицу длины}

В данной работе все расчеты ведутся в атомной системе единиц Хартри (АСЕХ). В ней нормализуются следующие константы:

\begin{table}[h]
\begin{tabular}{|l|l|l|}
\hline
Величина & Значение в АСЕХ & Значение в СИ \\\hline
Приведенная постонная Планка $\hbar$ & 1 & $\approx$ \num{1.5e-34}\si{\joule\second} \\\hline
Масса электрона $m_e$ & 1 &  $\approx$ \num{9.1e-31}\si{\kilo\meter} \\\hline
Заряд электрона $e$   & 1 & $\approx$ \num{1.6e-19}\si{\ampere\second} \\\hline
\end{tabular}
\end{table}

Производными единицами будут \todo{TODO}:

\begin{table}[h]
\begin{tabular}{|l|l|l|}
\hline
Величина & Значение в АСЕХ & Значение в СИ \\\hline
Приведенная постонная Планка $\hbar$ & 1 & \\\hline
Масса электрона $m_e$ & 1 &  9.1  \\\hline
Заряд электрона $e$   & 1 & \\\hline
\end{tabular}
\end{table}


\section{Различные граничные условия для уравнения Шредингера}

\subsection{Условия фон Неймана}
TODO

Интегрируемая особенность
\subsection{Условия Дирихле}
TODO

Неинтегрируемая особенность

\section{Обобщенные точечные взаимодействия}

TODO написать про дельту в одном измерении, что в больше чем одном зависит от способа стремления к нуля, а больше чем в трех, все совсем плохо

TODO
\subsection{Шкалы гильбертовых пространств}
Пусть $H$ — самосопряженный оператор с резольвентой $R(z) = \frac{1}{H - z I}$, действующий в гильбертовом пространстве $\hilb{0} = L^2(a, b)$.

Далее, выберем произвольную точку $z_0$ из резольвентного множества (дальнейшие построения не зависят от выбора конкретной точки), и для краткости записи, обозначим оператор $R(z_0)$ за $Z_0$.


TODO доказательства всюду плотности
\subsubsection{Положительная шкала}
Для $k > 0$, определим $\hilb{k} = Z_0^k \hilb{0} = \{ Z_0^k \phi \mid \phi \in \hilb{0} \}$.

\begin{prop}
\[
\dots \subseteq \hilb{k} \subseteq \hilb{k - 1} \subseteq \dots \subseteq \hilb{1} \subseteq{\hilb{0}}
\]
\end{prop}
Вначале покажем, что $\hilb{1} \subseteq \hilb{0}$. Действительно, любой $\psi \in \hilb{1}$ равен $Z_0 \phi$ для некоторого $\phi \in \hilb{0}$, а так как резольвента по определению ограничена в резольвентном множестве, то получим, что $\| \psi \| = \| Z_0 \phi \| \le \|Z_0\| \|\phi\| < \infty$, что означает, что $\psi \in \hilb{0}$.

Далее, покажем, что для любого $k$, $\hilb{k + 1} \subseteq \hilb{k}$. Для этого надо показать, что для любой $\psi \in \hilb{k + 1}$ (который имеет представление $\psi = R_0^{k + 1} \phi_1$), также лежит в $\hilb{k}$, то есть имеет вид $R_0^k \phi_2$ для некоторого $\phi_2 \in \hilb{0}$. Возьмем $\phi_2 = Z_0 \phi_1$:
\begin{itemize}
\item По ранее доказанному, он действительно лежит в $\hilb{0}$
\item $\psi = Z_0^{k + 1} \phi_1 = Z_0^k (Z_0 \phi_1) = Z_0^k \phi_2$, то есть $\psi \in \hilb{k}$
\end{itemize}

\subsubsection{Отрицательная шкала}
Для $k > 0$, определим $\hilb{-k}$ как пополнение пространства $\hilb{0}$ по норме $\|f \|_{-k} = \sup\limits_{u \in \hilb{k}} \frac{\ip{f}{u}_0}{\|u\|_j}$

Для $k > 0$, определим $\hilb{-k}$ как пространство, сопряженное к $\hilb{k}$, то есть пространство линейных ограниченных функционалов $\{ f: \hilb{k} \to \mathbb{C} \}$

TODO

For $k < 0$, $\hilb{k}$ actually is $\{ (H_0 - z_0 I)^k \phi \mid \phi \in \hilb{0} \}$.
For any $\phi \in \hilb{0}$, we can find such a $\psi \in \hilb{0}$, that $(H_0 - z_0 I) \psi = \phi$. Indeed, $\psi = R_0(z_0) \phi$. That means $\hilb{0} \subseteq \hilb{-1}$, and by induction, we get the negative scale
\[
\hilb{0} \subseteq \hilb{-1} \subseteq \hilb{-2} \subseteq \dots \subseteq \hilb{-m} \subseteq \dots
\] 
For each $k > 0$, we call the pair of spaces $\hilb{-k}, \hilb{k}$ \textit{compatible}.

Define the inner product on pairs of compatible spaces: for $\psi \in \hilb{-k}$, $\varphi \in \hilb{k}$:
\[
\ip{\psi}{\varphi} = (def) = \ip{\cconj{Z_0^k} \psi}{Z_0^{-k} \varphi}_0
\]

TODO motivation for such an inner product

Axioms: TODO

% \item Let us denote the elements of $\hilb{k}$ as ket vectors, that is, $\ket{\psi} \in \hilb{k}$.



\subsection{Предпонтрягинское пространство}
We want to define the operator $H_0 + \ket{\psi} \lambda \bra{\psi}$ (somewhat like $\delta$ potential?).

Пусть $\psi \in \hilb{-m - 1}$, $m \ge 1$. Для того, чтобы дать интерпретацию формальному выражению для формулы Крейна, надо расширить гильбертово пространство $\hilb{0}$ обобщенным дефектным элементом $Z_0 \psi$. Для этого нам необходимо пространство $\mathcal{P}_m$, сочетающее следующие свойства:

\begin{enumerate}
\item Содержит дефектный элемент $Z_0 \psi$
\item Содержит как можно меньше «лишних элементов» из  $\hilb{-m} \setminus \hilb{0}$
\item Содержит как можно больше элементов из $\hilb{0}$.
\item Скалярное произведение $\ip{\cdot}{\cdot}$ на $\mathcal{P}_m$ должно расширять частичное скалярное произведение $\ip{\cdot}{\cdot}_0$.
\item $Z_0$ должно быть резольвентой самосопряженного (относительно $\ip{\cdot}{\cdot}$) оператора в $\mathcal{P}_m$
\end{enumerate}

Таким образом, опеределим искомое пространство как:
\[
\mathcal{P}_m = 
\{ \varphi_m + \sum\limits_{i = -m}^{m - 1} c_i \psi_i \mid \varphi_m \in \hilb{m}, c_i \in \mathbb{C}\}
\]

, где $\psi_i = Z_0^{m + i + 1} \psi$, то есть, $\psi_i \in \hilb{i}$.

Определим скалярное произведение:

\[
\ip{\psi}{\psi'} =
\ip{\varphi_m}{\varphi'_m}_0 +
\sum\limits_{i = -m}^{m - 1} \cconj{c_i} \ip{\psi_i}{\varphi'_m}_0 +
\sum\limits_{j = -m}^{m - 1} c'_j \ip{\varphi_m}{\psi_j}_0 +
\sum\limits_{i = -m}^{m - 1} \sum\limits_{j = -m}^{m - 1} \cconj{c_i} c_j' g_{ij}
\]

TODO пояснить, почему так

Как видно, у нас есть свобода в выборе $g_{ij} = | \ip{\psi_i}{\psi_j} |_{ij}$. На нее наложены некоторые ограничения:

\begin{itemize}
\item $g_{ij}$ — эрмитова матрица, что необходимо для эрмитовой симметричности
\item некоторые из элементов $g_{ij}$ могут быть корректно определены, если $\psi_i$ и $\psi_j$ совместны. В частности, всегда можно корректно определить элементы, для которых $i + j \ge 0$, $\ip{f \in \hilb{i}}{g \in \hilb{j}}$
\item TODO weird resolvent costraint: $\mel{f}{R_0(z_0)}{g} = \cconj{\mel{g}{R_0(\cconj{z_0})}{f}}$, which means
\[
g_{i + 1, j} - g_{i, j + 1} = (z_0 - \cconj{z_0}) g_{i + 1, j + 1}
\]
\end{itemize}

\subsection{Понтрягинское пространство}
В пространстве $\mathcal{P}{m}$ все еще недостаточно структуры, так как:
\begin{itemize}
\item оно содержит не все элементы $\hilb{0}$
\item «norm topology still lacking» TODO и чо?
\end{itemize}

Для того чтобы решить эти проблемы, предпонтрягинское пространство будет пополнено в понтрягинское.

Определим 
\[
\Pi_m = \{f = (\phi_f, \vb{p_f}, \vb{n_f}) \mid \phi_f \in \hilb{0}, \vb{p_f}, \vb{n_f} \in \mathbb{C}^m \}
\]
с индефинитным скалярным произведением
\[
\ip{f}{g} =
\ip{\phi_f}{\phi_g}_0 +
\cconj{\vb{p_f}} \vdot \vb{n_g} +
\cconj{\vb{n_f}} \vdot \vb{p_g} + 
\cconj{\vb{n_f}} \vdot \vb{g_{ij}} \vdot \vb{n_g}
\]

Топология в этом пространстве естестенным образом определяется как топология произведения на $\hilb{0} \oplus \mathbb{C}^m \oplus \mathbb{C}^m$. TODO и чо?

Пусть $\vb{g_{ij}}$ — та самая матрица $g$ из $\mathcal{P}_m$. 

\begin{theorem}
Топологическим пополнем $\mathcal{P}_m$ является выше определенное пространство $\Pi_m$.
\end{theorem}

Доказательство:

Вложим $\mathcal{P}_m$ в $\Pi_m$ следующим образом:

\[
\varphi_m + \sum\limits_{i = -m}^{m - 1} c_i \psi_i \mapsto
\left(
\varphi_m + \sum\limits_{i = 0}^{m - 1} c_i \psi_i,
\left[ \ip{\psi_i}{\varphi_m}_0 + \sum\limits_{j = 0}^{m - 1} c_j g_{ij} \right]_{i = -1}^{-m},
\left[ c_i \right]_{i = -1}^{-m}
\right)
\]

Then restrict the mapping to the submanifold $\mathcal{P}'_m$ of $\mathcal{P}_m$, which is composed of elements $\varphi_m + \sum\limits_{i = - m}^{-1} c_i \psi_i$. $\mathcal{P}'_m$ is topologically dense in $\Pi_m$ because:

\begin{itemize}
\item $\hilb{m}$ is dense in $\hilb{0}$ (okay)
\item the functionals corresponding to $\psi_i$ for $i \le -1$ are unbounded (TODO so what?)
\end{itemize}

Заметим, что не известно, можно ли дать какой-то физический смысл пространству $\Pi_m$, поэтому требуется дополнительный анализ, чтобы убедиться, что полученные решения имеют некий физический смысл, к примеру, анализ асимптотик решений.