\chapter{Trash}
\label{trash}

\subsection{Дефектные элементы}
Пусть направление нормали к границе между областями совпадает с осью $y$, тогда: дефектный элемент $d(x, y) = \eval{\pdv{G(x, y, x_s, y_s; E_D)}{y_s}}_{x_s = x_0, y_s = y_0}$.

Далее понадобится действовать на этот дефектный элемент резольвентой, рассмотрим это действие. Напомним, что резольвента — интегральное преобразование, ядро которого — функция Грина:

$$(R(E) \psi)(x, y) = \int\limits_\Omega G(x', y', x, y) \psi(x', y') \dd{x'} \dd{y'}$$

Подействуем резольвентой $R(E_0)$ на $d(x, y)$:

\begin{align*}
(R(E_0) d)(x, y)
&= \int\limits_\Omega \sum\limits_{n, m} \frac{\psi^x_n(x') \psi^x_n(x) \psi^y_m(y') \psi^y_m(y)}{E^x_n + E^y_m - E_0} \sum\limits_{n, m} \frac{\psi_n^x(x') \psi_n^x(x_0) \psi_m^y(y') \pdv{\psi_m^y}{y_s} (y_0)}{E^x_n + E^y_m - E_D} \\
&= \sum\limits_{n, m} \frac{\psi_n^x(x) \psi_n^x(x_0) \psi_m^y(y) \pdv{\psi_m^y}{y_s} (y_0)}{(E^x_n + E^y_m - E_0)(E^x_n + E^y_m - E_D)} \\
\end{align*}

Легко заметить, что 

\[
(R^t(E_0) d)(x, y)
= \sum\limits_{n, m} \frac{\psi_n^x(x) \psi_n^x(x_0) \psi_m^y(y) \pdv{\psi_m^y}{y_s} (y_0)}{(E^x_n + E^y_m - E_0)^t(E^x_n + E^y_m - E_D)}
\]

\begin{align*}
& (R(E_0) d)(x, y) = \int\limits_\Omega \\
& \left(
\sum\limits_{n = 1}^\infty
\sqrt{\frac{2}{L_x}} \cos(\frac{\pi n}{L_x} x') \sqrt{\frac{2}{L_x}} \cos(\frac{\pi n}{L_x} x)
\left(-2\frac{\sin(kk_n(y' - L_y)) \sin(kk_n y)}{kk_n \sin(kk_n L_y)}\right)
\right) \\
& \left(
\sum\limits_{n = 1, n += 2}^\infty
\sqrt{\frac{2}{L_x}} \cos(\frac{\pi n}{L_x} x') \sqrt{\frac{2}{L_x}}
\left(-2\frac{\sin(kk_n(y' - L_y))}{\sin(kk_n L_y)}\right)
\right) = \\
& \sum\limits_{n = 1, n += 2}^\infty
\sqrt{\frac{2}{L_x}} \cos(\frac{\pi n}{L_x} x) \sqrt{\frac{2}{L_x}}\\
& \int\limits_y  \left(-2\frac{\sin(kk_n(y' - L_y)) \sin(kk_n y)}{kk_n \sin(kk_n L_y)}\right)
\left(-2\frac{\sin(kk_n(y' - L_y))}{\sin(kk_n L_y)}\right) = \\
& \sum\limits_{n = 1, n += 2}^\infty
\sqrt{\frac{2}{L_x}} \cos(\frac{\pi n}{L_x} x) \sqrt{\frac{2}{L_x}} 4 \frac{\sin(kk_n y) }{kk_n \sin^2(kk_n L_y)}\int\limits_y  \sin^2(kk_n(y' - L_y)) = \\
& \sum\limits_{n = 1, n += 2}^\infty
\sqrt{\frac{2}{L_x}} \cos(\frac{\pi n}{L_x} x) \sqrt{\frac{2}{L_x}} 4 \frac{\sin(kk_n y) }{kk_n \sin^2(kk_n L_y)}
\left( \frac{1}{2} L_y - \frac{\sin(2 kk_n L_y)}{4 kk_n} \right)
\end{align*}



Дефектный элемент $d(x, y)$ — производная функции Грина по нормали к границе в точке отверстия, то есть:

\begin{align*}
& d(x, y) = \eval{\pdv{G_R(x, y, x_s, y_s; E)}{y_s}}_{x_s = \frac{L_x}{2}, y_s = 0} \\
&= \sum\limits_{n = 1}^\infty
\sqrt{\frac{2}{L_x}} \sin(k_n^x (x + \frac{L_x}{2}))
\sqrt{\frac{2}{L_x}} \sin(k_n^x (x_s + \frac{L_x}{2}))
\eval{\pdv{G_{1D}(y, y_s; E - E^x_n)}{y_s}}_{y_s = 0} \\
&= \frac{2}{L_x} \sum\limits_{n = 1}^\infty
\sin(\frac{\pi n}{L_x} x + \frac{\pi}{2} n)
\sin(\frac{\pi}{2} n)
\eval{\pdv{G_{1D}(y, y_s; E - E^x_n)}{y_s}}_{y_s = 0} \\
&= \frac{2}{L_x} \sum\limits_{n = 1, n += 2}^\infty
\cos(\frac{\pi n}{L_x} x)
\left(-2\frac{\sin(kk_n(y - L_y)) kk_n \cos(kk_n y_s)}{kk_n \sin(kk_n L_y)}\right) \\
&= \frac{2}{L_x} \sum\limits_{n = 1, n += 2}^\infty
\cos(\frac{\pi n}{L_x} x)
\left(-2\frac{\sin(kk_n(y - L_y))}{\sin(kk_n L_y)}\right)
\end{align*}

Дефектный элемент не лежит в $L^2(\Omega_R)$:

\begin{align*}
& \int\limits_{x = -\frac{L_x}{2}}^{\frac{L_x}{2}} \int\limits_{y = 0}^{L_y} D(x, y)^2 \\
& = \frac{2}{L_x} \frac{2}{L_x} \frac{2}{L_y} \frac{2}{L_y}  \sum\limits_{n = 1, n += 2}^\infty
\int\limits_{x = -\frac{L_x}{2}}^{\frac{L_x}{2}} \cos^2(\frac{\pi n}{L_x} x) \int\limits_{y = 0}^{L_y}
\left( \sum\limits_{m = 1}^\infty \frac{\sin(\frac{\pi m}{L_y}y) \frac{\pi m}{L_y}}{E_n^x + E_m^y - E} \right)^2 \\
& = \frac{2}{L_x} \frac{2}{L_x} \frac{2}{L_y} \frac{2}{L_y}  \sum\limits_{n = 1, n += 2}^\infty
\int\limits_{x = -\frac{L_x}{2}}^{\frac{L_x}{2}} \cos^2(\frac{\pi n}{L_x} x)
\sum\limits_{m = 1}^\infty \left( \frac{\frac{\pi m}{L_y}}{E_n^x + E_m^y - E} \right)^2 \int\limits_{y = 0}^{L_y} \sin^2(\frac{\pi m}{L_y}y) \\
& = \frac{2}{L_x} \frac{2}{L_y}  \sum\limits_{n = 1, n += 2}^\infty
\sum\limits_{m = 1}^\infty \left( \frac{\frac{\pi m}{L_y}}{E_n^x + E_m^y - E} \right)^2
\end{align*}

\begin{align*}
& D(x, y) = \mcF(d(x, y)) = \int\limits_{\Omega_R} N_x^2 N_y^2 \sum\limits_{n = 1,2, m = 1}^\infty \frac{\cos(k_x x) \sin(k_m y) k_m}{E_{n, m} - E} e^{-i p_x x} e^{-i p_y y} \dd{x} \dd{y} \\
& = N_x^2 N_y^2 \sum\limits_{n = 1,2, m = 1}^\infty \frac{1}{E_{n, m} - E} k_m \int\limits_{\Omega_R} \cos(\frac{\pi n}{L_x} x) \sin(\frac{\pi m}{L_y} y) e^{-i p_x x} e^{-i p_y y} \dd{x} \dd{y} \\
& = N_x^2 N_y^2 \sum\limits_{n = 1,2, m = 1}^\infty \frac{1}{E_{n, m} - E} \frac{\pi m}{L_y} \\
& \frac{2 \pi  m L_x L_y e^{-i L_y p_y} \left(-(-1)^m+e^{i L_y p_y}\right) \left(\pi  n \sin \left(\frac{\pi  n}{2}\right) \cos \left(\frac{L_x p_x}{2}\right)-L_x \cos \left(\frac{\pi  n}{2}\right) p_x \sin \left(\frac{L_x p_x}{2}\right)\right)}{\left(\pi ^2 m^2-L_y^2 p_y^2\right) \left(\pi ^2 n^2-L_x^2 p_x^2\right)}
\end{align*}


\section{3D}
Рассмотрим уравнение Шредингера в двух областях $\Omega_1$ и $\Omega_2$ с условиями Дирихле на границе, общей границей $\partial \Gamma_{12}$, и точечным проколом $\vb{a_{12}} \in \partial \Gamma_{12}$. TODO картиночку

Типичный пример такой задачи — резонатор Гельмгольца TODO

Работаем в пространстве $\bbR^3$. Исходный гамильтониан: $H_0 = \frac{\hat{p}^2}{2 \mu}$ в $L_2(\bbR^3, \dd x)$.

% H_0 = - \frac{\hbar^2}{2 \mu } \laplacian

$\psi = \pdv{n} \delta(\vb{r})$.

Спектром оператора является $\bbR$, соответственно, резольвентное множество — $\{ c \mid \Im c \ne 0 \}$. В качестве $z_0$ возьмем, например, $i$, (напомним, что его выбор не влияет на структуру шкалы).


Рисунок 2. Нормаль направлена по координате $x_1$, $\psi = \delta'(x_1) \delta(x_2) \delta(x_3)$. Перейдем в импульсное представление:

В импульсном представлении гамильтониан $H_0 = \frac{p^2}{2 \mu}$, соответственно, $R_0(z_0) = \frac{1}{p^2 - z_0}$ TODO fix.

Переведем функционал $\psi$ импульсное предсавление:

TODO: $\hbar$
\begin{align*}
\Psi(p_1, p_2, p_3)
&= \mathcal{F}(\psi(x_1, x_2, x_3)) \\
&= \frac{1}{\sqrt{2 \pi \hbar}} \int\limits_{x_1} \frac{1}{\sqrt{2 \pi \hbar}} \int\limits_{x_2} \frac{1}{\sqrt{2 \pi \hbar}} \int\limits_{x_3} \psi(x_1, x_2, x_3) e^{-i \vb{x} \vdot \vb{p}} \\
&= \frac{1}{\sqrt{2 \pi \hbar}} \int\limits_{x_1} \delta'(x_1) e^{-i x_1 p_1} \frac{1}{\sqrt{2 \pi \hbar}} \int\limits_{x_2} \frac{1}{\sqrt{2 \pi \hbar}} \int\limits_{x_3} \delta(x_2) \delta(x_3) e^{-i x_2 p_2} e^{-i x_3 p_3} \\
&= \frac{1}{\sqrt{2 \pi \hbar}} \int\limits_{x_1} \delta'(x_1) e^{-i x_1 p_1} \mathcal{F}(\delta(x_2) \delta(x_3)) \\
&= \frac{1}{2 \pi \hbar} \mathcal{F}(\delta'(x_1)) \\
&= \frac{1}{2 \pi \hbar} i p_1 \mathcal{F}(\delta(x_1)) \\
&= \left(\frac{1}{\sqrt{2 \pi \hbar}}\right)^3 i p_1
\end{align*}

\begin{itemize}
\item $\Psi \notin \hilb{0}$ 
\item $\Psi \notin \hilb{-1}$
\item $\Psi \in \hilb{-2}$
\end{itemize}


Получаем предпонтрягинское пространство $\mcP_{-1}$, состоящее из функций вида:

\[
f(\vb{p}) = f_\phi(\vb{p}) + f_0 \frac{1}{(p^2 - i)^2} + f_{-1} \frac{1}{p^2 - i}
\]

Скалярное произведение:

\[
\ip{f}{g} = \ip{f_\phi}{g_\phi} + \dots + \cconj{f_0} G_{0,0} g_0 + \cconj{f_0} G_{0, -1} g_0 + \cconj{f_{-1}} G_{-1, 0} g_0 + \cconj{f_{-1}} G_{-1, -1} g_{-1}
\]

\begin{itemize}
\item $G_{0, 0}$ точно определено, так как $0 + 0 \ge 0$
\item $G_{0, -1} = \cconj{G_{-1, 0}} = ?$ — можно определить, так как можно показать, что $\psi_0 \in \hilb{1}$, а значит, скалярное произведение конечное
\item $G_{-1, -1}$ — свободный параметр
\end{itemize}

TODO пока мутно, как определять свободный параметр. Пусть определили.

\todo{А насколько маленькое может быть отверстие? Оно же в любом случае не меньше атома в диаметре?}

Скалярное произведение:

\begin{align*}
\ip{f}{g}
&= \ip{f_\phi}{g_\phi} \\
&+ \sum\limits_{i = -1}^0 \ip{f_i \Psi_i}{g_\phi}_0 + \sum\limits_{j = -1}^0 \ip{f_\phi}{g_j \Psi_j}_0 \\
&+ \sum\limits_{i = -1}^0 \sum \limits_{j = -1}^0 \cconj{f_i} G_{i,j} g_j 
\end{align*}

\begin{itemize}
\item $G_{0, 0} = \ip{\Psi_0}{\Psi_0} = \frac{\mu^2}{4 \pi \hbar^2}$
\item $G_{0, -1} = \ip{\Psi_0}{\Psi_1} = \frac{\mu^2 (2 i + \pi)}{8 \pi \hbar^2}$
\item $G_{-1, 0} = \cconj{G_{0, -1}}$
\item $G_{-1, -1}$ — свободный (вещественный) параметр
\end{itemize}

Свободный параметр $G_{-1, -1}$ определим так, чтобы оно было «конечной» частью расходящегося скалярного произведения $\ip{\Psi_{-1}}{\Psi_{-1}}_0$, следующим образом:

Далее будет удобно перейти в полярные координаты (так как $p^2$ встречается значительно чаще чем $p_1$ и $p_2$):

\begin{align*}
(p_1, p_2) &\to (p, \varphi) \\
p &\to p \\
p_1 &\to p \cos(\varphi) \\
p_2 &\to p \sin(\varphi) \\
\dd[2]{\vb{p}} &\to p \dd{p} \dd{\varphi}
\end{align*}

Рассмотрим интеграл в скалярном произведении $\ip{\cdot}{\cdot}_0$, на области с $p < N$:

\[
\int\limits_{\varphi = 0}^{2 \pi} \int\limits_{p = 0}^N
\cconj{\left(
\frac{1}{\frac{p^2}{2 \mu} + i} \frac{1}{\frac{p^2}{2 \mu} - i} \frac{1}{2 \pi \hbar} i p \cos \varphi
\right)}
\left(
\frac{1}{2 \pi \hbar} i p \cos \varphi
\right)
p \dd{p} \dd{\varphi}
=
\frac{\mu^2}{4 \pi \hbar^2} \log(1 + \frac{N^4}{4 \mu^2})
\]

TODO ссылка на Березина

Предел $\frac{1}{\log(N) \cdot} = \frac{\mu^2}{\pi \hbar^2}$.

Ну в общем, отнормированный интеграл:

\[
-\frac{\mu^2 \log (2 \mu)}{2 \pi  \hbar ^2}
\]

Ну ок.

TODO подставить в формулу Крейна и найти полюса резольвенты

Найдем $\Gamma^{-1}(z, \lambda)$. Заметим, что $R_0(z) \psi$ в общем случае не лежит в подпространстве, соответствующему $\psi_1$, поэтому надо выделить его коэффициенты $f_0, f_1$ и функцию $f_\phi$:

% Далее в нотации используется, что резольвента -- оператор умножения.

Далее будет полезно следущее тождество:

$R(z) = \frac{1}{H - z} = \frac{1}{H - z_0} + \frac{z - z_0}{(H - z)(H - z_0)} = \frac{1}{H - z_0} + \frac{z - z_0}{(H - z_0)^2} + \frac{(z - z_0)^2}{(H - z_0)^2 (H - z)}$

Используя тождество, получим компоненты элемента $R_0(z) \Psi$ в пространстве $\mcP_1$:

\begin{itemize}
\item $f_{-1} = 1$
\item $f_0 = z - i$
\item $f_\phi = R_0^2(i) R_0(z) (z - i)^2 \Psi$
\end{itemize}

% Разложим резольвенту в пространстве $\mcP_1$

% $R_0(z) \psi = \frac{1}{\frac{p^2}{2 \mu} - z} \frac{1}{2 \pi \hbar} i p_1 = f_\varphi + f_0 \frac{1}{\left(\frac{p^2}{2 \mu} - z_0 \right)^2} \psi + f_{-1} \frac{1}{\frac{p^2}{2 \mu} - z_0} \psi$

Замыкание $\mcP_1$ дает нам понтрягинское пространство $\Pi_1$


% Далее перейдем в импульсное представление. В нем:

% \begin{ilist}
% # $H_0 = p^2 - E$
% # $Z_0 = \frac{1}{p^2 - z_0}$
% # $\psi(x, y) \mapsto \Psi(p_x, p_y)$
% \end{ilist}

% Импульсное представление удобно тем, что гамильтониан и резольвента в нем являются операторами умножения и коммутируют с применениями к ним функции.

% Обозначим $\Psi_{i} = Z_0^{2 + i} \Psi$, то есть $\Psi_{i} \in \hilb{i}$.

% Рассмотрим скалярное произведение $\ip{\Psi_i}{\Psi_j}$ в $L^2(\Omega)$:

% \begin{align*}
% \ip{\Psi_i}{\Psi_j} = \int\limits_\Omega \frac{1}{(p^2 - \cconj{z_0})^i} \cconj{\Psi(\vb{p})} \frac{1}{(p^2 - z_0)^j} \Psi(\vb{p}) \dd{\vb{p}}
% \end{align*}


% Рассмотрим действие оператора на дефектный элемент:

% \todo{Пока без коэффициентов вообще. Потом разберусь.}

% \begin{align*}
% & (-\laplacian - E) \pdv{G}{n} (x, y, x_0, y_0; E_0) \\
% & = \pdv{}{n} \left((-\laplacian - E) G(x, y, x_0, y_0; E_0)\right) \\
% & = \pdv{}{n} \left(((-\laplacian - E_0) + (E_0 - E)) G(x, y, x_0, y_0; E_0)\right) \\
% & = \pdv{}{n} \left( - \delta(x - x_0) \delta(y - y_0) + (E_0 - E) G(x, y, x_0, y_0; E_0)\right) \\
% & = - \pdv{}{n} \delta(x - x_0) \delta(y - y_0) + (E_0 - E) \pdv{G}{n} (x, y, x_0, y_0; E_0)
% \end{align*}

% Таким образом, необходимо, чтобы кроме исходного дефектного элемента, в расширенном пространстве $\Pi$ оказалась производная дельта-функции Дирака.


% попробовать https://en.wikipedia.org/wiki/Barnes_zeta_function

Дальше:

\begin{elist}
# \todo{рассчитали элементы матрицы, есть скалярное произведение в $\mcP$ }
# \todo{получили производные функции Грина в $\Pi$ — дефектные элементы}
# \todo{сшили решения в $\Omega_W$ и $\Omega_R$ с условием нулевого потока через отверстие}
## \todo{а чтобы сшивать, надо знать асимптотики :(}
# \todo{посчитали поток в асимптотическом регионе и получили коэффициент прохождения}
\end{elist}


Домен сопряженного оператора $\Delta^*$ состоит из элементов вида:

\[
u(x) = \begin{cases}
a_W \pdv{G_W(x, y, x_s, y_s)}{y_s} (x_s = x_0, x_s = y_0) + b_W, & x \in \Omega_W \\
a_R \pdv{G_R(x, y, x_s, y_s)}{y_s} (x_s = x_0, x_s = y_0) + b_R, & x \in \Omega_R \\
\end{cases}
\]

Домен самосоряженного расширения $\Delta_E$ исходного оператора $\Delta$ — линейное подмножество, на котором зануляется граничная форма. \todo{написать про возможные расширения}

\todo{использовать асимптотику производной. Не уверен, что расширение то же самое будет}

Воспользуемся расширением с «нулевым потоком через отверстие»:

\begin{ilist}
# $a_W = -a_R$
# $b_W = b_R$
\end{ilist}

\subsection{Волновод}







Функция Грина для 2D: $G_0(\vb{x}) = \frac{i}{4} H_0^{(1)}(k |\vb{x}|)$



% Пусть $\frac{\hbar^2}{2 \mu} = 1$

$\psi(x_1, x_2) = \delta'(x_1) \delta(x_2)$

Далее будет удобно перейти в импульсное представление, так как в нем гамильтониан будет оператором умножения: $H_0 = \frac{p^2}{2 \mu}$.

Резольвента, соответственно, также будет оператором умножения, $R_0(z) = \frac{1}{\frac{p^2}{2 \mu} - z}$, что позволит менять ее местами с символами функций в вычислениях.

Переведем функционал $\psi(x_1, x_2)$ в импульсное представление $\Psi(p_1, p_2)$:

$\Psi(p_1, p_2) = \mcF(\psi(x_1, x_2)) = \frac{1}{2 \pi \hbar} i p_1$

Также упрощается частичное скалярное произведение:

\begin{align*}
\ip{f}{g}_0
&= \ip{R_0^k(\cconj{z_0}) f}{R_0^{-k}(z_0) g} \\
&= \int\limits_{\bbR^2} \cconj{R_0^k(\cconj{z_0}) f(\vb{p})} R_0^{-k}(z_0) g(\vb{p}) \dd[2]{\vb{p}} \\
&= \int\limits_{\bbR^2} R_0^k(z_0) R_0^{-k}(z_0) f(\vb{p}) g(\vb{p}) \dd[2]{\vb{p}} \\
&= \int\limits_{\bbR^2} f(\vb{p}) g(\vb{p}) \dd[2]{\vb{p}} \\
&= \ip{f}{g}
\end{align*}

$\psi \notin \hilb{0}$, $L^2$-норма $\psi$ бесконечна

$\psi \notin \hilb{-1}$, $L^2$-норма $R_0(z_0) \psi$ бесконечна

$\psi \in \hilb{-2}$, $L^2$-норма $R_0^2(z_0) \psi$ конечна

Получаем предпонтрягинское пространство $\mcP_{-1}$, состоящее из функций вида:

\[
f(\vb{p}) = f_\phi(\vb{p}) + f_0 \frac{1}{(\frac{p^2}{2 \mu} - i)^2} \Psi(\vb{p}) + f_{-1} \frac{1}{\frac{p^2}{2 \mu} - i} \Psi(\vb{p})
\]

\subsection{Цилиндрические координаты}

\[
\vb{r} = (r, \theta, z)
\]

\[
\hat {p_r} = - i \hbar \pdv{r};\quad 
\hat{p_\theta} = - i \hbar \frac{1}{r} \pdv{\theta};\quad 
\hat{p_z} = - i \hbar \pdv{z}
\]

$$\renewcommand\arraystretch{1.5}
\gradient f = 
\begin{pmatrix}
\pdv{f}{r} \\
\frac{1}{r} \pdv{f}{\theta}  \\
\pdv{f}{z} 
\end{pmatrix}
$$


$$
\divergence F =
\frac{1}{r} \pdv{(r F_r)}{r}
+ \frac{1}{r} \pdv{F_\theta}{\theta}
+ \pdv{F_z}{z}
$$



Можно использовать этот замкнутый вид, чтобы упростить функцию Грина для резонатора:

\begin{align*}
G_R(x, y, x_s, y_s; E)
&= \sum\limits_{n, m = 1}^\infty \frac{\psi_{nm}(x, y) \psi^*_{nm}(x_s, y_s)}{E_{nm} - E} \\
&= \sum\limits_n \sum\limits_m \frac{\psi_n(x) \psi_n^*(x_s) \psi_m(y) \psi_m^*(y_s)}{E^x_n + E^y_m - E} \\
&= \sum\limits_n \psi_n(x) \psi_n^*(x_s) \sum\limits_m \frac{\psi_m(y) \psi_m^*(y_s)}{E^y_m - (E - E^x_n)} \\
&= \sum\limits_n \psi_n(x) \psi^*_n(x_s) G^y_{1D}(y, y_s; E - E^x_n)
\end{align*}

\subsection{Волновод}


\begin{prop}
\[
\dots \subseteq \hilb{k} \subseteq \hilb{k - 1} \subseteq \dots \subseteq \hilb{1} \subseteq{\hilb{0}}
\]
\end{prop}
Вначале покажем, что $\hilb{1} \subseteq \hilb{0}$. Действительно, любой $\psi \in \hilb{1}$ равен $Z_0 \phi$ для некоторого $\phi \in \hilb{0}$, а так как резольвента по определению ограничена в резольвентном множестве, то получим, что $\| \psi \| = \| Z_0 \phi \| \le \|Z_0\| \|\phi\| < \infty$, что означает, что $\psi \in \hilb{0}$.

Далее, покажем, что для любого $k$, $\hilb{k + 1} \subseteq \hilb{k}$. Для этого надо показать, что для любой $\psi \in \hilb{k + 1}$ (который имеет представление $\psi = R_0^{k + 1} \phi_1$), также лежит в $\hilb{k}$, то есть имеет вид $R_0^k \phi_2$ для некоторого $\phi_2 \in \hilb{0}$. Возьмем $\phi_2 = Z_0 \phi_1$:
\begin{itemize}
\item По ранее доказанному, он действительно лежит в $\hilb{0}$
\item $\psi = Z_0^{k + 1} \phi_1 = Z_0^k (Z_0 \phi_1) = Z_0^k \phi_2$, то есть $\psi \in \hilb{k}$
\end{itemize}


\todo{TODO аваыаыва}

For $k < 0$, $\hilb{k}$ actually is $\{ (H_0 - z_0 I)^k \phi \mid \phi \in \hilb{0} \}$.
For any $\phi \in \hilb{0}$, we can find such a $\psi \in \hilb{0}$, that $(H_0 - z_0 I) \psi = \phi$. Indeed, $\psi = R_0(z_0) \phi$. That means $\hilb{0} \subseteq \hilb{-1}$, and by induction, we get the negative scale
\[
\hilb{0} \subseteq \hilb{-1} \subseteq \hilb{-2} \subseteq \dots \subseteq \hilb{-m} \subseteq \dots
\] 



\todo{написать тут что-то нормальное}
Сначала продолжим на $\Pi$ резольвенту:
\begin{align*}
R_\Pi(z) \left( \phi; \vb{a}; \vb{b} \right) = \left( \phi'; \vb{a'}; \vb{b'} \right)
\end{align*}
, где:
\begin{ilist}
#
\[
\phi' = R(z) \phi + R(z) \psi_{-1} \times \sum\limits_{j = -m}^{-1} b_j (z - z_0)^{-(j + 1)}
\]
#
\begin{align*}
a'_i
&= \sum\limits_{j = i + 1}^{-1} a_j (z - \cconj{z_0})^{(i + 1) - j} \\
&+ (z - \cconj{z_0})^{-(i + 1)} \ip{\psi_{-1}}{R(z) \phi} \\
&+ \ip{\psi_i}{R(z) \psi_{-1}} \sum\limits_{j = -m}^{-1} b_j (z - z_0)^{-(j + 1)}
\end{align*}
#
\begin{align*}
b'_i
&= \sum\limits_{j = -m}^{i - 1} b_i (z - z_0)^{i - (j + 1)}
\end{align*}
\end{ilist}


\todo{Кажется, надо сначала продолжить резольвенту, а из нее уже достать оператор??}

Необходимо продолжить оператор $A$ на $\Pi$. Для этого сначала определим его на $\mcP$ следующим образом.
\[
\dom A_\mcP = \{f = f_\phi + \sum\limits_{i = -m + 1}^{m} f_i \psi_i \mid f_\phi \in \hilb{m + 1}, f_{i}\in \bbC \}
\]



% Заметим, что не известно, можно ли дать какой-то физический смысл пространству $\Pi_m$, поэтому требуется дополнительный анализ, чтобы убедиться, что полученные решения имеют некий физический смысл, к примеру, анализ асимптотик решений.

% \section{Формула Крейна}
% \[
% R_\lambda(z) = R_0(z) - R_0(z) \ket{\psi} \Gamma^{-1}(z, \lambda) \bra{\psi} R_0(z)
% \]

% \[
% \Gamma(z, \lambda) = \frac{1}{\lambda} + \frac{1}{2}(z - z_0) \bra{\psi} R_0(z) R_0(z_0) \ket{\psi} + \frac{1}{2}(z - \cconj{z_0}) \bra{\psi} R_0(z) R_0(\cconj{z_0}) \ket{\psi}
% \]

\subsection{Преобразование Фурье}
В задачах квантовой механики часто удобно и более естественно производить анализ в импульсном представлении, нежели чем в координатном. Они связаны между собой преобразованием Фурье (напоминаем, что пользуемся АСЕ, засчет чего константа $\hbar$ опускается):

\subsubsection{Из координатного представления в моментное}
Пусть имеется волновая функция $\psi: \bbR^n \to \bbC$. 

$\hat{\vb{p}} = -i \grad$

$H_0 = - \frac{1}{2 \mu} \laplacian$

$\Psi(\vb{p}) = \mcF(\psi(\vb{x})) = \left(\frac{1}{\sqrt{2 \pi}}\right)^n \int\limits_{\bbR^n} \psi(\vb{x}) e^{-i \vb{p} \vdot \vb{x}} \dd[n]{\vb{x}}$

\subsubsection{Из импульсного представления в координатное}

$\hat{\vb{x}} = i \grad_p$

$H_0 = \frac{p^2}{2 \mu}$

$\psi(\vb{x}) = \mcF^{-1}(\Psi(\vb{p})) = \left(\frac{1}{\sqrt{2 \pi}}\right)^n \int\limits_{\bbR^n} \Psi(\vb{p}) e^{i \vb{p} \vdot \vb{x}} \dd[n]{\vb{p}}$


Заметим, что формально:
\[
A \psi_i = A Z_0^{m + i} \psi = A \frac{1}{A - z_0} Z_0^{m + i - 1} \psi = (1 + z_0 Z_0) Z_0^{m + i - 1} \psi = \psi_{i - 1} + z_0 \psi_i
\]

Напомним, что $\dom A$ — пространство $\hilb{1}$. Рассмотрим, как вкладывается в понтрягинское пространство область определения $A$:
\[
f + f_0 \psi_0 + \sum\limits_{i = -1}^{-m} f_i \psi_i \xmapsto[]{\Pi}
\begin{pmatrix}
f + f_0 \psi_0 \\
\left[ \ip{\psi_i}{f + f_0 \psi_0} \right]_{i = -1}^{-m} \\
\left[ f_i \right]_{i = -1}^{-m}
\end{pmatrix}
\]
, где $f_{-m} = 0$.

Далее рассмотрим, как вкладывается действие оператора $A$:
\begin{align*}
  & A(f + f_0 \psi_0 + \sum\limits_{i = -1}^{-m} f_i) \\
= & A f + f_0 z_0 \psi_0 + f_0 \psi_{-1} + \sum\limits_{i = -1}^{-m} f_i (z_0 \psi_i + \psi_{i - 1}) \\
= & A f + f_0 z_0 \psi_0 + \sum\limits_{i = -1}^{-m} (f_{i + 1} + z_0 f_i) \psi_i  \\
\xmapsto[]{\Pi} &
\begin{pmatrix}
A f + f_0 z_0 \psi_0 \\
\left[ \ip{\psi_i}{A f + f_0 z_0 \psi_0} \right]_{i = -1}^{-m} \\
\left[ f_{i + 1} + z_0 f_i\right]_{i = -1}^{-m}
\end{pmatrix}
\end{align*}

Пусть $p_i = \ip{\psi_i}{f + f_0 \psi_0}$. Рассмотрим $\ip{\psi_i}{A f + f_0 z_0 \psi_0}$:
\begin{align*}
\ip{\psi_i}{A f + f_0 z_0 \psi_0}
&= \ip{\psi_i}{A (f + f_0 \psi_0) - f_0 \psi_{-1}} \\
&= \ip{A \psi_i}{f + f_0 \psi_0} - f_0 \ip{\psi_i}{\psi_{-1}} \\
&= \ip{z_0 \psi_i + \psi_{i - 1}}{f + f_0 \psi_0} - f_0 G_{i, -1} \\
&= \cconj{z_0} \ip{\psi_i}{f + f_0 \psi_0} + \ip{\psi_{i - 1}}{f + f_0 \psi_0} - f_0 G_{i, -1} \\
&= \cconj{z_0} p_{i} + p_{i - 1} - f_0 G_{i, -1}
\end{align*}
\todo{блин, и что делать с $p_{-m - 1}$? И вообще непонятно, почему здесь можно перекидывтаь оператор.}

Таким образом, график оператора $A_\Pi$ выглядит как:
\[
A_\Pi = \left\{
\begin{pmatrix}
f + n_0 \psi_0 \\
\left[ p_i \right]_{i = -1}^{-m} \\
\left[ n_i \right]_{i = -1}^{-m} \\
\end{pmatrix},
\begin{pmatrix}
A f + z_0 f_0 \psi_0 \\
\left[ \cconj{z_0} p_i + p_{i - 1} - n_0 G_{i, -1} \right]_{i = -1}^{-m} \\
\left[ z_0 n_i + n_{i + 1} \right]_{i = -1}^{-m} \\
\end{pmatrix}
\middle|
f \in \hilb{1},
n_{-m} = 0
\right\}
\]

\begin{elist}
Пользуясь тем, что по определению, $\psi = \Delta_0 d$, получаем
\[
\Delta_0 d \xmapsto[]{\mcP} (0, 0, 0, 0, 1) \xmapsto[]{\Pi} (0; 0, 0; 0, 1)
\]
# $d$

Заметим:
\begin{align*}
& \psi_{-1} = Z_0 \psi = \frac{1}{\Delta_0 - z_0} \Delta_0 d = d + z_0 Z_0 d \in \hilb{-1} \\
& z_0 \psi_{0} = z_0 Z_0 d + z_0^2 Z_0^2 d \in \hilb{0} \\
& z_0^2 \psi_{1} = z_0^2 Z_0^2 d + z_0^3 Z_0^3 d \in \hilb{1} \\
& f_\phi = z_0^3 Z_0^3 d \in \hilb{2}
\end{align*}

Поочередно вычитаем и прибавляем, получаем  $\psi_{-1} - z_0 \psi_0 + z_0^2 \psi_1 - f_\phi =  d$, соответственно:
\[
d \xmapsto[]{\mcP} (-z_0^3 Z_0^3 d, z_0^2, -z_0, 1, 0) \xmapsto[]{\Pi}
\begin{pmatrix}
-z_0 Z_0 d \\
\left[ \ip{\psi_i}{-z_0^3 Z_0^3 d}_0 -z_0 G_{i, 0} + z_0^2 G_{i, 1}\right]_{i = -1}^{-2}\\
1\\
0
\end{pmatrix}
\]
# $u \in \dom \Delta_0$

Вкладывается непосредственно:
\[
u \xmapsto[]{\mcP} (u, 0, 0, 0, 0) \xmapsto[]{\Pi} (u;\left[ \ip{\psi_i}{u}_0 \right]_{i = -1}^{-2}; 0, 0)
\]
# $\Delta_0 u$, для $u \in \dom \Delta_0$
% TODO возможно, палево
$\Delta_0 u \in \hilb{2}$, также вкладывается непосредственно:
\[
\Delta_0 u \xmapsto[]{\mcP} (\Delta_0 u, 0, 0, 0, 0) \xmapsto[]{\Pi} (\Delta_0 u; \left[ \ip{\psi_i}{\Delta_0 u}_0 \right]_{i = -1}^{-2}; 0, 0)\]
% Как это аналитически будем считать, вообще непонятно.
\end{elist} % bordermatrix


\[
f \xmapsto[]{\Pi}
\begin{pmatrix}
- \alpha_f z_0 Z_0 d + u \\
\left[ \ip{\psi_i}{- \alpha_f z_0^3 Z_0^3 d + u}_0 - \alpha_f z_0 G_{i, 0} + \alpha_f z_0^2 G_{i, 1} \right]_{i = -1}^{-2} \\
\alpha_f \\
0
\end{pmatrix}
\]

\[
\Delta f = \alpha_f \Delta d + \Delta u  \xmapsto[]{\Pi}
\begin{pmatrix}
\Delta u \\
\left[ \ip{\psi_i}{\Delta u}_0\right]_{i = -1}^{-2} \\
0 \\
\alpha_f
\end{pmatrix}
\]

\begin{align*}
\ip{f}{\Delta g}_\Pi
&= \ip{- \alpha_R z_0 Z_0 d + u}{\Delta v}_0 \\
&+ \cconj{(\ip{\psi_{-2}}{- \alpha_R z_0^3 Z_0^3 d + u}_0 - \alpha_R z_0 G_{-2, 0} + \alpha_R z_0^2 G_{-2, 1})} \alpha_R \\
&+ \cconj{\alpha_R} \ip{\psi_{-1}}{\Delta v}_0 \\
&+ \cconj{\alpha_R} G_{-1, -2} \alpha_R
\end{align*}

\begin{align*}
\ip{\Delta f}{g}_\Pi
&= \ip{\Delta u}{- \alpha_R z_0 Z_0 d + v}_0 \\
&+ \cconj{\ip{\psi_{-1}}{\Delta u}_0} \alpha_R \\
&+ \cconj{\alpha_R} (\ip{\psi_{-2}}{- \alpha_R z_0^3 Z_0^3 d + v}_0 - \alpha_R z_0 G_{-2, 0} + \alpha_R z_0^2 G_{-2, 1})\\
&+ \cconj{\alpha_R} G_{-2, -1} \alpha_R
\end{align*}