\documentclass{beamer}
\usepackage[utf8]{inputenc}
\usepackage[english,russian]{babel}

\usepackage{xcolor}

\usepackage{amsmath,amsfonts,amssymb,amsthm,amscd,amsxtra}
\usepackage{physics}
\usepackage[sharp]{easylist}
\usepackage{comment}

\usepackage{beamerthemesplit}
\DeclareMathOperator*{\argmin}{arg\,min}
\usetheme{Boadilla}
\setbeamertemplate{caption}[numbered]

% \usepackage[pdftex]{graphicx}
\graphicspath{ {pic/} }

\newcommand{\hilb}[1]{\mathcal{H}_{#1}}
\newcommand{\cconj}[1]{\overline{#1}}

\newcommand{\mcL}{\mathcal{L}}
\newcommand{\mcP}{\mathcal{P}}
\newcommand{\bbC}{\mathbb{C}}
\newcommand{\bbR}{\mathbb{R}}


\title[]{\todo{Метод R-матриц и} резонансные эффекты в квантовых волноводах}
\author{Дмитрий Герасимов}
\institute[ИТМО]{Университет ИТМО}
\date{19 мая 2014 года}

\begin{document}

\maketitle

\begin{frame}[fragile]
\frametitle{Мотивация}
% http://ask.metafilter.com/49898/Why-does-computer-chip-process-size-have-to-keep-getting-smaller
% https://answers.yahoo.com/question/index?qid=20101122011450AAjpPHn
% http://www.reddit.com/r/askscience/comments/17uuvi/what_is_the_absolute_limit_for_transistor_sizes/
\begin{easylist}[itemize]
# Почти все полупроводники, использующиеся на данный момент для интегральных схем — полевые транзисторы (MOSFET)
# Необходимость уменьшать размеры полупроводниковых приборов
## производительность (конечная скорость распространения сигнала + скорость переключения состояния)
## тепловыделение (меньшее напряжение затвора)
## более эффективное использование полупроводниковой пластины (более плотная упакова)
# Текущие технологии: транзисторы размером 22 нм, в 2014 ожидается 14 нм
# На размерах около 10 нм проявляется туннелирование через затвор (ток утечки) и между транзисторами (интерференция)
# Необходимо учитывать квантовые (в частности, туннельные) эффекты
\end{easylist}
\end{frame}

\begin{frame}[fragile]
\frametitle{Альтернативная реализация транзистора}
\begin{easylist}[itemize]
# \dots или же использовать квантовые эффекты непосредственно для создания нелинейных микроэлектронных приборов
# Используем резонансный эффект \todo{Определить, что это такое}
# В окрестности резонанса линейное изменение параметров нелинейно меняет проводимость: \todo{Может, формулу какую-нибудь}
# Коэффициент прохождения непосредственно влияет на электрическую проводимость: $G(E) = G_0 \sum\limits_n T_n(E)$ % \todo{А точно такая формула?}
\end{easylist}
\end{frame}

\begin{frame}[fragile]
\frametitle{Использование квантовых точек для получения нелинейных эффектов}
\begin{easylist}[itemize]
# Добавление дефектов/резонаторов в структуру волновода позволяет получить резонансные эффекты
# Квантовые точки
## Квантовые точки — очень высокоточные «контроллируемые дефекты»
## Можно менять размер, например, электрическим полем \todo{может, эффективную массу}
## Cовременные технологии позволяют изготавливать точки размером 2-10 нм
# Регулируя параметры квантовой точки в окрестности резонанса, можно получать нелинейный эффект
\end{easylist}
\end{frame}

\begin{frame}[fragile]
\frametitle{Использование квантовых точек для получения нелинейных эффектов (продолжение)}
\begin{figure}
\includegraphics[width=0.8\textwidth]{resonator.png}
\caption{Модель волновода с резонатором Гельмгольца}
\end{figure}
% TODO \Omega_w \Omega_R квантовая точка
\end{frame}

\begin{frame}[fragile]
\frametitle{Использование квантовых точек для получения нелинейных эффектов (продолжение)}
\begin{figure}
\includegraphics[width=0.8\textwidth]{transmission-dependency-resonator.png}
\caption{}
\end{figure}
\end{frame}

\begin{frame}[fragile]
\frametitle{Использование квантовых точек для получения нелинейных эффектов (продолжение)}
\begin{figure}
\includegraphics[width=0.8\textwidth]{low-transmission.png}
\caption{Волновая функция в резонансной точке}
\end{figure}
\end{frame}

\begin{frame}[fragile]
\frametitle{Использование квантовых точек для получения нелинейных эффектов (продолжение)}
\begin{figure}
\includegraphics[width=0.8\textwidth]{high-transmission.png}
\caption{Волновая функция в отдалении от резонансной точки}
\end{figure}
\end{frame}

% TODO резонансы указывать соответствующие графику прохождения


\begin{frame}[fragile]
\frametitle{Постановка математической задачи}
\begin{easylist}[itemize]
# Стационарное уравнение Шредингера: $\hat{H} \psi(\vb{r}) = E \psi(\vb{r})$
## Решаем задачу рассеяния, т.е. работаем с непрерывным спектром
# Ток вероятности: $\vb{j}(\vb{r}, t) = \frac{\hbar}{2 m i} (\Psi(\vb{r}, t)^* \grad_r{\Psi(\vb{r}, t)} - \Psi(\vb{r}, t) \grad_r{\Psi^*(\vb{r}, t))}$. Векторная величина, функция нескольких аргументов, хотим получить некое скалярное характеристическое значение $J$:
## Не зависит от времени для не изменяемой во времени конфигурации системы
## В проводнике можно выделить направление распространения волны, по этому направлению идет основной поток, остальными можно пренебречь
## Интегрируем по сечению в асимптотической области

# Формула для коэффициента прохождения: $T = \frac{|J_{inc}|}{|J_{trans}|}$
\end{easylist}
\end{frame}


\begin{frame}[fragile]
\frametitle{Точные решения}
\begin{easylist}[itemize]
# Аналитически:
## не решается (трансцендентные уравнения для собственных чисел в простейшей модели одномерного прямоугольного барьера)
# Численно
## долго
## домен задачи бесконечен
## нельзя выбрать какой-то конечный поддомен с разумными граничными условиями, так как состояния рассеивания \todo{мутно написано}цу
\end{easylist}
\end{frame}


\begin{frame}[fragile]
\frametitle{Точечные взаимодействия: основная идея}
\begin{easylist}[itemize]
# Заменим конечное отверстие на точечное (нулевого радиуса), расположенное в точке $\vb{s}$, получаем аппроксимацию модели с конечным отверстием моделью с отверсием нулевого радиуса
# Модель с отверстием нулевого радиуса можно решать аналитически
# Можно экспериментировать с аналитической моделью, а после подбора параметров проверить ее свойства численно
\end{easylist}
\end{frame}


\begin{frame}[fragile]
\frametitle{Точечные взаимодействия и самосопряженные расширения}
\begin{easylist}[itemize]
# Сузим решения в областях $\Omega_W$ и $\Omega_R$ до функций, зануляющихся в $\vb{s}$
## $\Delta = \Delta_W \oplus \Delta_R$
### Симметрический оператор: формально $\Delta = \Delta^*$
### Но не обязательно самосопряжённый:$\dom{\Delta} \ne \dom{\Delta^*}$
# Нет взаимодействия между областями, надо его «включить». Математически — расширить оператор $\Delta$ до самосопряжённого $\Delta_E$
## Из физических соображений можем получить расширение, которое асимптотически совпадает с настоящим решением
\end{easylist}
\end{frame}


\begin{frame}[fragile]
\frametitle{Точечные взаимодействия и самосопряженные расширения (продолжение)}
\begin{easylist}[itemize]
# $\dom{\Delta} \subseteq \dom{\Delta^*}$
# Вместо расширения домена исходного оператора можно сужать домен сопряженного: $\dom{\Delta} \subseteq \dom{\Delta_E} = \dom{\Delta_E^*} \subseteq \dom{\Delta^*}$
% # $\dom{\Delta^*} = \{ \}$ \todo{TODO}
# Условие самосопряженности — зануление формы: $\forall f, g \in \dom{\Delta_E}: J(f, g) = \ip{\Delta^* f}{g} - \ip{f}{\Delta^* g}$ \todo{нормально выписать форму}
\end{easylist}
\end{frame}


\begin{frame}[fragile]
\frametitle{Точечные взаимодействия: условие Неймана}
\begin{easylist}[itemize]
# Условие на ноль производной: $\eval{\pdv{\psi}{n}}_{\Gamma_W} = 0$, $\eval{\pdv{\psi}{n}}_{\Gamma_R} = 0$
## Как самостоятельная задача, не имеет физического смысла
## Было рассмотрено в работах \todo{TODO нужно ли тут цитирование?}
# Ненулевые индексы дефекта, существуют самосопряженные расширения в $L^2(\Omega)$
# Дефектные элементы: функции Грина $G(\vb{x}, \vb{s}, k_0)$
## Имеют сингулярность при $\vb{x} = \vb{s}$
## Но при этом лежат в $L^2(\Omega)$
\end{easylist}
\end{frame}


\begin{frame}[fragile]
\frametitle{Точечные взаимодействия: условие Дирихле}
\begin{easylist}[itemize]
# Условие на ноль волновой функции: $\eval{\psi}_{\Gamma_W} = 0$, $\eval{\psi}_{\Gamma_R} = 0$
## Является физически обоснованным: частица имеет нулевую вероятность оказаться за пределами волновода + непрерывность
## Не было до этого проанализировано в применении к конкретным моделям волноводов
# Но $\Delta$ имеет нулевые индексы дефекта:
## Функции Грина имеют те же граничные условия, поэтому $G(\vb{x}, \vb{s}, k_0) = 0$
## В качестве дефектных элементов формально подходят производные функции Грина $\pdv{G(\vb{x}, \vb{s}, k_0)}{n_s}$, но они не лежат в $L^2(\Omega)$!
# Необходим выход в более широкое, чем $L^2(\Omega)$, пространство, в котором и будет строиться расширение
\end{easylist}
\end{frame}

% !!!!!!!
% \begin{comment}

\begin{frame}[fragile]
\frametitle{Расширение пространства: шкалы}
\begin{easylist}[itemize]
# $H$ — самосопряженный оператор, действующий в $\hilb{0}$ со скалярным произведением $\ip{\cdot}{\cdot}_0$
# Резольвента: $R(z) = \frac{1}{H - z}$. Пусть $z_0 \in \rho(H)$ (произвольное), $Z_0 = R(z_0)$
# Положительная шкала: для $k > 0$, определим $\hilb{k} = Z_0^k \hilb{0} = \{ Z_0^k \phi \mid \phi \in \hilb{0} \}$, «хорошие» функции
## $\dots \subseteq \hilb{k} \subseteq \hilb{k - 1} \subseteq \dots \subseteq \hilb{1} \subseteq \hilb{0}$
## В каждом $\hilb{k}$ определим скалярное произведение: $\ip{f}{g}_k = \ip{Z_0^{-k} f}{Z_0^{-k} g}_0$
# Отрицательная шкала: для $k > 0$, определим $\hilb{-k}$ как пополнение пространства $\hilb{0}$ по норме $\|f \|_{-k} = \sup\limits_{u \in \hilb{k}} \frac{\ip{f}{u}_0}{\|u\|_j}$, «плохие» функции
## $\hilb{0} \subseteq \hilb{-1} \subseteq \dots \subseteq \hilb{-k} \subseteq \hilb{-k - 1} \subseteq \dots $
\end{easylist}
\end{frame}


\begin{frame}[fragile]
\frametitle{Расширение пространства: шкалы (продолжение)}
\begin{easylist}[itemize]
# Скалярное произведение $\ip{\cdot}{\cdot}_0: \hilb{0} \times \hilb{0} \to \bbC$ можно расширить до $\ip{\cdot}{\cdot}_E: \hilb{-k} \times \hilb{k} \to \bbC$
\begin{align*}
\ip{f}{g}_E &\eqdef \lim\limits_{n \to \infty} \ip{f_n}{u}_0 \\
\ip{g}{f}_E &\eqdef \cconj{\ip{f}{g}}
\end{align*}
\end{easylist}
\end{frame}


\begin{frame}[fragile]
\frametitle{Расширение пространства: предпонтрягинское пространство}
\begin{easylist}[itemize]
# Пусть $\psi$ — «плохой» элемент не из $\hilb{0}$ (в нашем случае производная функции Грина)
## $\psi \in \hilb{-m} \setminus \hilb{-m + 1}$ 
# Хотим получить пространство $\mcP$ такое, что:
## Содержит элемент $\psi$
## Содержит как можно меньше «лишних» элементов $\hilb{-m} \setminus \hilb{0}$
## Содержит как можно больше элементов из $\hilb{0}$
## Скалярное произведение в $\mcP$ определено на всех парах элементов и расширяет исходное
\end{easylist}
\end{frame}


\begin{frame}[fragile]
\frametitle{Расширение пространства: предпонтрягинское пространство (продолжение)}
\[
\mathcal{P} = 
\{ \varphi_m + \sum\limits_{i = -m}^{m - 1} c_i \psi_i \mid \varphi_m \in \hilb{m}, c_i \in \mathbb{C}\}
\]
, где $\psi_i = Z_0^{m + i} \psi$, то есть, $\psi_i \in \hilb{i}$.

\begin{align*}
\ip{f}{f'}_{\mcP}
&= \ip{\varphi_m}{\varphi'_m}_E \\
&+ \sum\limits_{i = -m}^{m - 1} \ip{c_i \psi_i}{\varphi'_m}_E + \sum\limits_{j = -m}^{m - 1} \ip{\varphi_m}{c_j' \psi_j}_E \\
&+ \sum\limits_{i = -m}^{m - 1} \sum\limits_{j = -m}^{m - 1} \cconj{c_i} g_{ij} c_j'
\end{align*}
\end{frame}


\begin{frame}[fragile]
\frametitle{Расширение пространства: предпонтрягинское пространство (продолжение)}
\begin{easylist}[itemize]
# $g_{ij}$ — эрмитова матрица: $g_{ij} = \cconj{g_{ji}}$
# Некоторые из элементов $g_{ij}$ могут быть корректно определены, если $\psi_i$ и $\psi_j$ совместны, в частности, когда $i + j \ge 0$
# Для выполнения резольвентного тождества, необходимо $\mel{f}{R_0(z_0)}{g} = \cconj{\mel{g}{R_0(\cconj{z_0})}{f}}$, что означает
$g_{i + 1, j} - g_{i, j + 1} = (z_0 - \cconj{z_0}) g_{i + 1, j + 1}$
# Оставшиеся элементы (а это, как минимум, $g_{-m, -m}$) — свободные параметры, определяются из некоторых физических соображений (т.н. ренормализация)
\end{easylist}
\end{frame}


\begin{frame}[fragile]
\frametitle{Расширение пространства: понтрягинское пространство}
Но в $\mcP$ все еще не все элементы $\hilb{0}$!

Определим 
\[
\Pi = \{f = (\phi_f, \vb{p_f}, \vb{n_f}) \mid \phi_f \in \hilb{0}, \vb{p_f}, \vb{n_f} \in \mathbb{C}^m \}
\]
с индефинитным скалярным произведением
\[
\ip{f}{g}_\Pi =
\ip{\phi_f}{\phi_g}_E +
\cconj{\vb{p_f}} \vdot \vb{n_g} +
\cconj{\vb{n_f}} \vdot \vb{p_g} + 
\cconj{\vb{n_f}} \vdot \vb{g_{ij}} \vdot \vb{n_g}
\]

Вложим $\mathcal{P}$ в $\Pi$:
\begin{align*}
\varphi_m + \sum\limits_{i = -m}^{m - 1} c_i \psi_i \mapsto
\big(
& \varphi_m + \sum\limits_{i = 0}^{m - 1} c_i \psi_i, \\
& \left[ \ip{\psi_i}{\varphi_m}_E + \sum\limits_{j = 0}^{m - 1} c_j g_{ij} \right]_{i = -1}^{-m}, \\
& \left[ c_i \right]_{i = -1}^{-m}  \big)
\end{align*}
\end{frame}


\begin{frame}[fragile]
\frametitle{Расширение пространства: понтрягинское пространство (продолжение)}
\begin{easylist}[itemize]
# Вложение — плотное, пополнением $\mcP$ будет понтрягинское пространство $\Pi$
# Операторы, ранее определенные на $\hilb{0}$, также замыкаются в $\Pi$
\end{easylist}
\end{frame}
% \end{comment}

\begin{frame}[fragile]
\frametitle{Расширение оператора в понтрягинском пространстве $\Pi$}
\begin{easylist}[itemize]
# Строим пространство $\Pi = (\hilb{0} \times \bbC^m \times \bbC^m, \ip{\cdot}{\cdot}_\Pi)$, которое расширяет $(\hilb{0}, \ip{\cdot}{\cdot}_0)$
# Теперь расширение строится в пространстве $\Pi$
# Оператор в $\Pi$ симметрический, и теперь у него есть ненулевые индексы дефекта, можно расширить до самосопряженного
# Поступаем аналогично: строим домен сопряженного оператора, и смотрим, на каких элементах зануляется граничная форма $J$. \todo{TODO должна быть переписана в терминах понтрягинского скалярного произведения}
\end{easylist}
\end{frame}


\begin{frame}[fragile]
\frametitle{Расширение оператора в понтрягинском пространстве $\Pi$ (продолжение)}
\begin{easylist}[itemize]
# Из условия аннулирования граничной формы, получаем ограничения на коэффициенты перед дефектными элементами. Выбираем условие «отсутствия потока из отверстия».
# После этого можем:
## Получить решения уравнения
## Коэффициент прохождения
## Обнаружить резонансы
## Построить зависимость проводимости от параметров резонатора
\end{easylist}
\end{frame}


\begin{frame}[fragile]
\frametitle{Полученные результаты}
\begin{easylist}[itemize]
# \todo{Есть программа, чтобы все это быстро можно было посчитать и визуализовать}
# \todo{Есть куски вычислений}
\end{easylist}
\end{frame}


\begin{frame}[fragile]
\frametitle{Проверка результатов}
\begin{easylist}[itemize]
# Последовательность численных решений
# Проанализировать какие-нибудь свойства, которые можно получить аналически
## Асимптотики
## \todo{???}
\end{easylist}
\end{frame}


\begin{frame}[fragile]
\frametitle{Дальнейшие вариации задачи}
\begin{easylist}[itemize]
# Двумерный волновод, квантовая точка внутри проводника
# Трехмерный волновод, квантовая точка внутри проводника
# Трехмерный волновод, тороидальный резонатор Гельмгольца \todo{Взаимодействие не точечное, но симметрия может спасти}
# Полупрозрачная перегородка \todo{Сложно, не факт что удастся аналитически}
# Магнитное поле \todo{Не нашел асимптотик функции Грина для гамильтониана с магнитным полем, не факт что успею вывести}
# \todo{Картиночки сюда}
\end{easylist}
\end{frame}

\end{document}