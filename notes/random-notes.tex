\documentclass[12pt, a4paper]{article}
%Gummi|065|=)
\usepackage[utf8]{inputenc}
\usepackage[T2A]{fontenc}
\usepackage{amsmath,amsthm,amsfonts,amssymb,amscd}
\usepackage[russian, english]{babel}
\usepackage{enumerate}
\usepackage{comment}
\usepackage{cite}
\usepackage{mathrsfs}
\usepackage{fullpage}
\usepackage{url}
\usepackage{tabularx}
\usepackage{graphicx}
\usepackage{indentfirst}
\usepackage{bm}
\usepackage{braket}

\def\res{\mathop{\operatorname{Res}}\limits}

\begin{document}

Impenetrable wall + delta function potential

$$V(x) = \begin{cases}
\infty, & x < 0 \\
0, & x < 0 < a \\
-\infty, & x = a \\
0, & x > a
\end{cases}$$

Bound state: $E < 0$, $k = \sqrt{-E}$

Region 1:

$$\psi_1(x) = A e^{-kx} + B e^{kx}$$

Boundary condition: $\psi_1(0) = 0$, $A = -B$

Therefore,

$$\psi_1(x) = A(e^{-kx} - e^{kx})$$

Region 2:

$$\psi_2(x) = C e^{-kx} + D e^{kx}$$

Boundary condition: $\psi_2(\infty) = 0$

Therefore,

$$\psi_2(x) = C e^{-kx}$$

Continuoty conditions:

$$\psi_1(a) = \psi_2(a)$$

$$\Delta \frac{d \psi}{d x} = - \frac{2m}{\hbar^2}\psi(a)$$

!!! Assume $\Delta \frac{d \psi}{d x} = - \psi(a)$ for simplicity

From the continuoty:

$$A(e^{-ka} - e^{ka}) = C e^{-ka}$$

$$C = A(1 - e^{2ka})$$

Derivatives:

$$ \left. \frac{d \psi_1}{dx} \right|_{x = a} = \left. A (-k e^{-kx} - k e^{kx}) \right|_{x = a} = -kA(e^{-ka} + e^{ka})$$

$$ \left. \frac{d \psi_2}{dx} \right|_{x = a} = -k C e^{-ka} = -k A(1 - e^{2ka}) e^{-ka} = -k A(e^{-ka} - e^{ka})$$

$$ \Delta \frac{d \psi}{dx} = 2 k A e^{ka} = -A(e^{-ka} - e^{ka})$$

Assume $a = 1$ for simplicity.

$$2k e^k = e^k - e^{-k}$$

Scattering states:

$$\psi_1(x) = A e^{-ikx} + B e^{ikx}$$

$\psi_1(-1) = 0 = A e^{ik} + B e^{-ik}$. 

TODO: suppose both A and B are real. 

$\frac{A}{B} e^{2ik} = -1$, therefore, $A = B$ or $A = -B$ to match magnitudes. TODO is that important? We should distinguish, I suppose.

Suppose $A = -B$

$2k = 2 \pi n$, $k = \pi n, n \in \mathbb{Z}$

$$\psi_1(x) = A (e^{-i \pi n x} - e^{i \pi n x})$$
$$\psi'_1(x) = -i \pi n A (e^{-i \pi n x} + e^{i \pi n x})$$
$$\psi'_1(0) = -2 i \pi n A$$



$$\psi_2(x) = C e^{-i k x} + D e^{ikx}$$


$\psi_1(0) = \psi_2(0) = 0$

$0 = C + D$, $D = -C$

$$\psi_2(x) = C (e^{-i \pi n x} - e^{i \pi n x})$$
$$\psi'_2(x) = -i \pi n C (e^{-i \pi n x} + e^{i \pi n x})$$
$$\psi'_2(0) = -2 i \pi n C$$

$$\Delta \frac{d \psi}{dx} = -\psi(0) = 0$$

So, $-C + A = 0$, $A = C$

Ok, what does all that mean? For energies $\pi^2 n^2$, there is no delta barrier, there is just stationary point at zero?


\section{Infinite well + step}

$$V(x) = \begin{cases}
\infty, & x < -1 \\
0, & -1 < x < 0 \\
-V_0, & 0 < x < 1 \\
\infty, & x > 1
\end{cases}$$

Suppose $-V_0 < E < 0$

$$k = \sqrt{-E}$$

$$\psi_1(x) = A e^{-kx} + B e^{kx}$$

$\psi_1(-1) = 0 = A e^k + B e^{-k}$, therefore, $B = -A e^{2k}$

$$\psi_1(x) = A (e^{-kx} - e^{2k} e^{kx})$$

$$\psi'_1(x) = -k A (e^{-kx} + e^{2k} e^{kx})$$

$$k' = \sqrt{E + V_0} = \sqrt{-k^2 + V_0}$$

$$\psi_2(x) = C e^{-ik'x} + D e^{ik'x}$$

$\psi_2(1) = 0 = C e^{-ik'} + D e^{ik'}$, therefore, $D = -C e^{-2ik'}$

$$\psi_2(x) = C (e^{-ik'x} - e^{-2ik'} e^{ik'x})$$

$$\psi_1(0) = \psi_2(0)$$

$$A (1 - e^{2k}) = C (1 - e^{-2ik'})$$

$$\psi'_1(0) + \psi'_2(0)$$

$$-k A (1 + e^{2k}) = -ik' C (1 + e^{-2ik'})$$

Divide the second equation by the first:

$$-k \frac{1 + e^{2k}}{1 - e^{2k}} = -ik' \frac{1 + e^{-2ik'}}{1 - e^{-2ik'}}$$

$\frac{1 + e^{-2ik'}}{1 - e^{-2ik'}}$ is pure complex and equal to $\coth(ik') = -i \cot(k')$

$$k \coth(k) = k' \cot(k')$$

That actually has some solutions!

\section{some weird basis}
Particle in infinite well with $L = 1$.

$$(-\frac{d^2}{dx} - E) \psi(x) = \psi(x)$$

Let

$$f_1(x) = x (1 - x)$$
$$f_2(x) = x (\frac{1}{2} - x) (1 - x)$$

(without normalization). Orthonormal because $f_1$ is symmetric around 0.5 and $f_2$ is antisymmetric. Same applies to the derivatives.

$\langle f_1 \mid H - E \mid f_1 \rangle = 10 - E$

$\langle f_2 \mid H - E \mid f_2 \rangle = 42 - E$

Therefore, $E = 10$, $a_1 = 1$ and $E = 42$, $a_2 = 1$. These are clearly not solutions. Reason: incorrect assumption that $\psi(x) = a_1 f_1(x) + a_2 f_2(x)$.


TODO something to do with bound states?

TODO investigate scattering states.

\section{Free particle + self-adjoint boundary conditions}

Domain: $[0, a]$

Boundary conditions:

$$\psi(0) = 0$$

$$a\frac{\psi'(a)}{\psi(a)} = B$$

\begin{enumerate}
\item Negative energy case:

$k = \sqrt{-E}$

$$\psi(x) = A (e^{-kx} - e^{kx})$$

$$a \frac{-k e^{-ka} - k e^{ka}}{e^{-ka} - e^{ka}} = B$$

$$-k a \coth (-ka) = k a \coth (ka) = B$$

For $B > 1$, there exists a single solution with $k > 0$.

\item Zero energy:

$$\psi(x) = Ax$$

$$a\frac{1}{a} = 1 = B$$

\item Positive energy:

$k = \sqrt{E}$

$$\psi(x) = A (e^{-ikx} - e^{ikx}) = -2 I A \sin(kx)$$

$$a \frac{-ik e^{-ika} - ik e^{ika}}{e^{-ika} - e^{ika}} = B$$

$$-i k a \coth (-ika) = -i^2 k a \cot(ka) = k a \cot(ka) = B$$

A lot of solutions for every value of $B$.

For $B < 1$: extra solution.

\end{enumerate}

Suppose $B = -1.0$, $a = 1.0$.

No solutions for negative and zero energy.

$k \cot(k)$ tends to straight lines in the neighbourhood of zero.

$k_n \sim \frac{\pi}{2} + \pi n$

$u_n(1) \to 0$


$$R(E) = \frac{1}{a} \sum\limits_{\lambda = 1}^{\infty} \frac{u_{\lambda}^2(a)}{k^2_{\lambda} - k^2}$$

\subsection{Bounds on the sum}
Suppose $a = 1$.

Let $k_n$ be the nth positive root of $x \cot(x) = B$.

Let $b_n$ be its asymptotic approximation: $b_n = \frac{pi}{2} + \pi n$

If $B < 0$, $b_n < k_n$

If $B > 0$, $b_n > k_n$

Value of $n$th eigenfunction square on the boundary is $$\psi^2_n(a) = \frac{1}{4 (\frac{a}{2} - \frac{\sin(2 k_n a)}{4k_n})} (2 \sin(k_na))^2 = \frac{\sin^2 k_n}{\frac{1}{2} - \frac{\sin (2 k_n)}{4 k_n}}$$

As function of $k$:

\includegraphics[scale=0.6]{fun.png}

That is, for $B < 0$, $\psi^2(b_n) > \psi^2(k_n)$ if $k_n$ is slightly greater than $\frac{\pi}{2}$.

So, starting from $i = n$ for which $k^2_i > k^2$:
$$R_i = \sum\limits_{i = n}^\infty \frac{\psi_i^2(1.0)}{k_i^2 - k^2} < \sum \frac{\phi^2_i(1.0)}{b_i^2 - k^2} = \sum \frac{2.0}{b_i^2 - k^2}$$

$$\sum \frac{1}{(\pi/2 + \pi i)^2 - k^2} = \frac{1}{\pi^2} \sum \frac{1}{i^2 + i + 1/4 - \frac{k^2}{\pi^2}}$$

$$D = \sqrt{1 - (1 - 4 \frac{k^2}{\pi^2})} = 2 \frac{k}{\pi}$$

$$i_{1, 2} = -\frac{1}{2} \pm \frac{k}{\pi}$$

$$\frac{1}{i^2 + i + 1/4 - \frac{k^2}{\pi^2}} = \frac{\pi}{2 k} \left( \frac{1}{i + 0.5 - \frac{k}{\pi}} - \frac{1}{i + 0.5 + \frac{k}{\pi}} \right)$$

$$R_n \le -\frac{1}{\pi k} (-PG(0.5 + n + \frac{k}{\pi}) + PG(0.5 + n - \frac{k}{\pi}))$$

Where PG is the polygamma function.

Seems to be working..

% $k = 1.0$: $R \le 1.557$, real $R \approx 0.376$

% $k = 2.0$: $R \le -1.09$, real $R \approx 0.922$. Fuck.

\subsection{Bounds on the sum: more general case}
$$a k \cot(ak) = B$$
$$k_n \sim \frac{1}{a}(\frac{\pi}{2} + \pi n)$$

$$\psi_n^2(a) = \frac{4 \sin^2(k_n a)}{2 a - \frac{\sin (2 k_n a)}{k_n}}$$

Asymptotically, $\psi_n^2(a) \le 2 / a$.


$$\left( \frac{1}{a} (\frac{\pi}{2} + \pi i) \right)^2 - k^2 \le k_i^2 - k^2$$

$$\frac{1}{a^2 } \left( (\frac{\pi}{2} + \pi i)^2 - a^2 k^2\right)$$

$$R \le \frac{1}{a} \sum \frac{2 / a}{\frac{1}{a^2 } \left( (\frac{\pi}{2} + \pi i)^2 - a^2 k^2\right)} = \sum \frac{2}{(\frac{\pi}{2} + \pi i)^2 - a^2 k^2} = \frac{1}{\pi^2} \frac{\pi}{ak} \sum \frac{1}{i + 0.5 - \frac{ak}{\pi}} - \frac{1}{i + 0.5 + \frac{ak}{\pi}}$$

$$R_{n_0} \le \frac{1}{\pi^2} \frac{\pi}{ak} (PG(n_0 + 0.5 + \frac{ak}{\pi}) - PG(n_0 + 0.5 - \frac{ak}{\pi}))$$



\section{Trash}
$$\psi(x) = e^{-ikx} + U e^{ikx}$$

$$\frac{\psi'(a)}{\psi(a)} = X$$

$$U = e^{-2 i k a} \frac{ik + X}{ik - X}$$

\section{Delta potential}

Suppose we have delta potential at $x = d$: $- a \delta(x - d)$

$$\psi_1(x) = e^{-i k x} - e^{i k x}$$ 

, we omit the $A$ coefficient and calculate the normalization later.

$$\psi_2(x) = B e^{-i k x} + C e^{i k x}$$

$$\begin{cases}
\psi_1(d) = \psi_2(d) \\
\psi_2'(d) - \psi_1'(d) = - a \psi_1(d) 
\end{cases}$$

After solving, we get:

$$B = \frac{i a e^{2 i d k}-i a+2 k}{2 k}$$

$$C = -\frac{e^{-2 i d k} \left(i a e^{2 i d k}-i a+2 k e^{2 i d k}\right)}{2 k}$$

Suppose $d = 1.0$ and $a = 1.0$

$$\psi_2(x) = \frac{\left(i e^{2 i k} - i+2 k\right) e^{-i k x}}{2 k}-\frac{\left(i e^{2 i k}+2 k e^{2 i k} -i\right) e^{-2 i k+i k x}}{2 k}$$

\section{Random definitions}
Resonance: pole of resolvent/Green's function continued meromorphically to the lower half-plance.

Real part: energy of the resonance, imaginary part: rate of decay.

\subsection{Resolvent of the free particle}
$$G(x, s; E) = \int\limits_{0}^{\infty} \frac{\psi_\lambda(x) \psi_\lambda^*(s)}{E - \lambda} d \lambda$$

$\psi_\lambda(x) = $

\section{Question}
Green's function is kernel of the resolvent:

$R_\lambda(f) = \int G(x, s) f(s) ds$

What is incoming and outgoing Green's function/resolvent?

\section{Cylinder delta scattering}

$$\Delta \Psi(r, \theta, z) = E \Psi(r, \theta, z)$$

Symmetric w.r.t. to $\theta$, separate variables:

$$\Psi_m(E; r, \theta, z) = \frac{e^{i m \theta}}{\sqrt{2 \pi}} \psi_m(E; r, z)$$

Each $m$ defines a 2D scattering problem.

$$\left( - \left( \frac{\partial^2}{\partial r^2} + \frac{1}{r} \frac{\partial}{\partial r} - \frac{m^2}{r^2} + \frac{\partial^2}{\partial z^2} \right) + V(r, z) \right) \psi_m(E; r, z) = E \psi_m(E; r, z)$$

Again, separate variables:

$$\psi_m(E; r, z) = \phi(r) \varphi(z)$$

$\phi(r)$: radial equation

$$- \left( \frac{d^2}{dr^2} + \frac{1}{r} \frac{d}{dr} - \frac{m^2}{r^2} + V_\perp(r) \right) \phi(r) = E_\perp \phi(r)$$

Solutions: 

$$\phi_n^m(r) = \frac{\sqrt{2}}{RJ_{|m| + 1}(x_{mn})}J_m(x_{mn} \frac{r}{R}), n = 1, 2 \dots$$

$$E_{\perp n}^m = \left( \frac{x_{mn}}{R} \right)^2, n = 1, 2 \dots$$

$\varphi(z)$: one-dimensional spatial equation

$$\left( -\frac{d^2}{dz^2} + V_s \right) \varphi(z) = (E - E_\perp) \varphi(z)$$

For fixed $n, m$ two linearly independent solutions.

\subsection{Next attempt}
Fix channels $m$ and $n$:

Left region:
$$\psi^1_n(r, z) = \exp(I k_n z) \phi_n(r) + \sum\limits_{t = 1}^\infty R_{nt} \exp(-i k_t z) \phi_t(r)$$

Right region:
$$\psi^2_n(r, z) = \sum\limits_{t = 1}^\infty T_{nt} \exp(i k_t z) \phi_t(r)$$


Boundary conditions at $z = 0$: for any $r$:

$$\psi_n^1(r, 0) = \psi_n^2(r, 0)$$

$$\partial_z \psi^2(r, 0) - \partial_z \psi_n^1(r, 0) = u \psi^1_n(r, 0)$$

Substituting expressions:

$$\phi_n(r) + \sum\limits_{t = 1}^\infty R_{nt} \phi_t(r) = \sum\limits_{t = 1}^\infty T_{nt} \phi_t(r)$$

$$\sum\limits_{t = 1}^\infty T_{nt} i k_t \phi_t(r) - (ik_n \phi_n(r) - \sum\limits_{t = 1}^\infty R_{nt} i k_t \phi_t(r)) =  u \sum\limits_{t = 1}^\infty T_{nt} \phi_t(r)$$

For each $q$, multiply by $\phi_q(r)$ and integrate from $0$ to $R$.

For $q = n$:

$$1 + R_{nn} = T_{nn}$$

$$k_n T_{nn} - (k_n - k_n R_{nn}) = \frac{u}{i} T_{nn}$$

$$\begin{cases}
R = \frac{2 k_n}{2 k_n + i u} \\
T = \frac{- iu}{2 k_n + iu} \\
\end{cases}$$

For $q \ne n$:

$$0 + R_{nq} = T_{nq}$$

$$k_q T_{nq} - (0 - k_q R_{nq}) = \frac{u}{i} T_{nq} $$

$$\begin{cases}
R = 0\\
T = 0\\
\end{cases}$$

\section{Cylindric delta well, surrounded by host material}
Fix channels $m$ and $n$:

Left region:
$$\psi^1_n(r, z) = \exp(i k_n z) \phi_n(r) + \sum\limits_{t = 1}^\infty R_{t} \exp(-i k_t z) \phi_t(r)$$

Right region:
$$\psi^2_n(r, z) = \sum\limits_{t = 1}^\infty T_{t} \exp(i k_t z) \phi_t(r)$$


Boundary conditions at $z = 0$:

$\forall 0 \le r \le RR: \psi_n^1(r, 0) = \psi_n^2(r, 0)$

$\forall 0 \le r \le R: \partial_z \psi^2(r, 0) = \partial_z \psi_n^1(r, 0) + u \psi^1_n(r, 0)$

$\forall R \le r \le RR: \partial_z \psi^2(r, 0) = \partial_z \psi_n^1(r, 0)$

Substituting expressions:

$\forall 0 \le r \le RR: \phi_n(r) + \sum\limits_{t = 1}^\infty R_{t} \phi_t(r) = \sum\limits_{t = 1}^\infty T_{t} \phi_t(r)$

$\forall 0 \le r \le R: \sum\limits_{t = 1}^\infty T_{t} k_t \phi_t(r) = k_n \phi_n(r) - \sum\limits_{t = 1}^\infty R_{t} k_t \phi_t(r) +  \frac{u}{i} \sum\limits_{t = 1}^\infty T_{t} \phi_t(r)$

$\forall R \le r \le RR: \sum\limits_{t = 1}^\infty T_{t} k_t \phi_t(r) = k_n \phi_n(r) - \sum\limits_{t = 1}^\infty R_{t} k_t \phi_t(r)$

For each $q$, multiply first equation by $r \phi_q(r)$ and integrate from $0$ to $RR$:

\begin{itemize}
\item for $q = n$: $1 + R_{n} = T_{n}$
\item for $q \ne n$ :$R_{q} = T_{q}$
\end{itemize}

For each $q$, multiply second and third equation by $r \phi_q(r)$, integrate second from $0$ to $R$, third from $R$ to $RR$, and add them up:

$\int\limits_0^{RR} \sum\limits_{t = 1}^\infty T_{t} k_t r \phi_t(r) \phi_q(r) = \int\limits_0^{RR} k_n r \phi_n(r) \phi_q(r) - \int\limits_0^{RR} \sum\limits_{t = 1}^\infty R_{t} k_t r \phi_t(r) \phi_q(r) + \int\limits_0^R \frac{u}{i} \sum\limits_{t =1}^\infty T_{t} r \phi_t(r) \phi_q(r)$

\begin{itemize}
\item for $q = n$: $T_{n} k_n = k_n - R_{n} k_n + \frac{u}{i} \sum\limits_{t = 1}^{\infty} T_{t} \int\limits_{0}^R r \phi_t(r) \phi_n(r)$
\item for $q \ne n$: $T_{q} k_q = -R_{q} k_q + \frac{u}{i} \sum\limits_{t = 1}^\infty T_{t} \int\limits_{0}^R r \phi_t(r) \phi_q(r)$
\end{itemize}

Substitute $T_{n} = R_{n} + 1, T_{q} = R_{q}$:

\begin{itemize}
\item for $q = n$: $(R_{n} + 1) k_n = k_n - R_{n} k_n + \frac{u}{i} (\sum\limits_{t = 1}^{\infty} R_{t} \int\limits_{0}^R r \phi_t(r) \phi_n(r) + \int\limits_{0}^R r \phi_n(r) \phi_n(r))$
\item for $q \ne n$: $R_{q} k_q = -R_{q} k_q + \frac{u}{i} (\sum\limits_{t = 1}^\infty R_{t} \int\limits_{0}^R r \phi_t(r) \phi_q(r) + \int\limits_0^R r \phi_n(r) \phi_q(r))$
\end{itemize}

Simplify,

\begin{itemize}
\item for $q = n$: $2 R_{n} k_n - \frac{u}{i} \sum\limits_{t = 1}^{\infty} R_{t} \int\limits_{0}^R r \phi_t(r) \phi_n(r) =  \frac{u}{i} \int\limits_{0}^R r \phi_n(r) \phi_n(r)$
\item for $q \ne n$: $2 R_q k_q - \frac{u}{i} \sum\limits_{t = 1}^\infty R_{t} \int\limits_{0}^R r \phi_t(r) \phi_q(r) = \frac{u}{i} \int\limits_0^R r \phi_n(r) \phi_q(r)$
\end{itemize}

DO NOT FORGET TO SET $u = \frac{2 \mu}{\hbar^2} u$ IN REAL WORLD CALCULATIONS.

\section{Zero width slit, 2D geometry}
Domain of the adjoint operator:

$$u(r) = \begin{cases}
\beta_1 G_1^{E}(r, r_{12}, k_0) + u_1(r), & r \in \Omega_1 \\
\beta_{12} G^{I}(r, r_{12}, k_0) + \beta_{23} G^{I}(r, r_{23}, k_0) + u_2(r), & r \in \Omega_2 \\
\beta_3 G_3^{E}(r, r_{23}, k_0) + u_3(r), & r \in \Omega_3 \\
\end{cases}$$

$k_0$ is some regular (TODO: what's that) value of the spectral parameter.

$u_i \in W_2^2(\Omega_i)$ (TODO what's $W_2^2$?)

Boundary conditions: conservation of flux:

$$\begin{cases}
\beta_1 = - \beta_{12} \\
\beta_3 = - \beta_{23} \\
u_1(r_{12}) = u_2(r_{12}) \\
u_3(r_{23}) = u_2(r_{23}) \\
\end{cases}$$

Search solution of scattering problem in form:

$$u(r, k) = \begin{cases}
\alpha_1 G^E_1(r, r_{12}, k) + {\tilde u}(x, k), & x \in \Omega_1 \\
\alpha_{12} G^I(r, r_{12}, k) + \alpha_{23} G^I(r, r_{23}, k), & x \in \Omega_1 \\
\alpha_3 G^E_3(r, r_{23}, k), & x \in \Omega_1 \\
\end{cases}$$

\subsection{Particle in a 2D box: von Neumann boundary conditions}

$$-\frac{\hbar^2}{2 m} \left( \frac{\partial^2}{\partial x^2} + \frac{\partial^2}{\partial y^2} \right) \psi(x, y) = E \psi(x, y) $$

$$\begin{cases}
\psi(x, y) = X(x) Y(y) \\
- \frac{\hbar^2}{2 m} \frac{\partial^2}{\partial x^2} X(x) = E_x X(x) \\
- \frac{\hbar^2}{2 m} \frac{\partial^2}{\partial y^2} Y(x) = E_y Y(y) \\
E = E_x + E_y
\end{cases}$$

$$\begin{cases}
X(x) = A_x \sin(k_x x) + B_x \cos(k_x x) \\
Y(y) = A_y \sin(k_y y) + B_y \cos(k_y y) \\
k_x = \sqrt{\frac{2 m E_x}{\hbar^2}} \\
k_y = \sqrt{\frac{2 m E_y}{\hbar^2}} \\
\end{cases}$$

Boundary conditions:

$$\begin{cases}
X'(0) = 0, A_x = 0 \\
X'(L_x) = 0, k_x L_x = \pi n_x \\
Y'(0) = 0, A_y = 0 \\
Y'(L_y) = 0, k_y L_y = \pi n_y \\
\end{cases}$$

$$\begin{cases}
\psi(x, y) = \frac{2}{\sqrt{L_x L_y}} \cos(k_x x) \cos(k_y y) \\
k_x = \frac{\pi n_x}{L_x} \\
k_y = \frac{\pi n_y}{L_y} \\
E = \frac{\hbar^2 \pi^2}{2 m} \left(\frac{n_x^2}{L_x^2} + \frac{n_y^2}{L_y^2} \right) \\
\end{cases}$$

Let $n = n_x$, $m = n_y$.

\subsubsection{Green's function}

$$G(x, y, x_s, y_s; E) = \sum\limits_{n, m = 1}^\infty \frac{\psi_{nm}(x, y) \psi^*_{nm}(x_s, y_s)}{E_{nm} - E}$$

TODO investigate convergence on the boundary

$$G(x, y, x_0, y_0; E) = G(x, y, \frac{L_x}{2}, 0; E) = \sum\limits_{n, m = 1}^\infty \frac{4}{L_x L_y} \frac{\cos(k^x_n x) \cos(k^y_m y) \cos(\frac{\pi}{2}m)}{E_{nm} - E}$$

$$G(x_0, y_0, x_0, y_0; E) = \frac{4}{L_x L_y} \sum\limits_{n, m = 1}^\infty \frac{\cos(\frac{\pi}{2}n) \cos(\frac{\pi}{2}n)}{E_{nm} - E} = \frac{4}{L_x L_y} \sum\limits_{n, m = 1}^\infty \frac{\cos^2(\frac{\pi}{2}n)}{E_{nm} - E}$$

$$G(x_0, y_0, x_0, y_0; E) = \frac{4}{L_x L_y} \sum\limits_{n = 1}^\infty \cos^2(\frac{\pi}{2}n) \sum\limits_{m = 1}^\infty \frac{1}{E_{nm} - E} = \frac{4}{L_x L_y} \sum\limits_{n' = 1}^\infty \sum\limits_{m = 1}^\infty \frac{1}{E_{(2n')m} - E}$$

\subsection{1D free particle}
We have two fold-eigenvalue degeneracy: for each $E$, there is $\psi^E_+(x) = N_E e^{i k x}$ and $\psi^E_-(x) = N_E e^{-i k x}$, where $k = \frac{\sqrt{2 m E}}{\hbar}$. $\psi_+$ and $\psi_-$ are orthonormal: $\int \psi_+(x) \psi^*_-(x) dx = \int |N_E|^2 e^{2 i \sqrt{E} x} dx = \int |N_E|^2 \frac{1}{2} e^{i \sqrt{E} x} = \frac{1}{2} |N_E|^2 2 \pi \delta(k)$, which is zero for non-zero $E$.

Eigenstates normalization: $\bra{\psi_E} \ket{\psi_{E'}} = \int\limits_{x = - \infty}^\infty \psi_E(x) \psi_{E'}^*(x) = N_E N_{E'}^* \int\limits_{x = - \infty}^\infty e^{i (k - k') x} dx = \\ N_E N^*_{E'} 2 \pi \delta(k - k')$

Next, use $\delta(g(x)) = \frac{\delta(x - x_0)}{g'(x_0)}$ (the function has only one root).

$\bra{\psi_E} \ket{\psi_{E'}} = N_E N^*_{E'} 2 \pi \delta(\frac{\sqrt{2 m E}}{\hbar} - \frac{\sqrt{2 m E'}}{\hbar}) = N_E N^*_{E'} 2 \pi \sqrt{\frac{2 E }{m}} \hbar \delta(E - E')$

It has to be equal to $\delta(E - E')$. For $E \ne E'$, the normalization could be arbitrary, for $E = E'$ the coefficient should be equal to 1:

$|N_E|^2 2 \pi \sqrt{\frac{2 E}{m}} \hbar = 1$, therefore, $|N_E| = \left( \frac{m}{2E}\right)^{1/4}  \frac{1}{\sqrt{2 \pi \hbar}}$

\subsubsection{Green's function}
$$G(x, s; E) = \frac{1}{2 \pi \hbar} \left( \frac{m}{2}\right)^{1/2}  \left( \int\limits_{0}^{\infty} \frac{1}{\sqrt{\lambda}}\frac{e^{i \frac{\sqrt{2 m \lambda}}{\hbar} (x - s)}}{\lambda - E} d \lambda + \int\limits_{0}^{\infty} \frac{1}{\sqrt{\lambda}}\frac{e^{-i \frac{\sqrt{2 m \lambda}}{\hbar} (x - s)}}{\lambda - E} d \lambda \right)$$

Change the variable of integration: 

$\begin{cases}
k = \frac{\sqrt{2 m \lambda}}{\hbar} \\
\lambda = \frac{\hbar^2 k^2}{2m}\\
d \lambda = 2\frac{\hbar^2}{2 m} k dk
\end{cases}$


\begin{align*}
G(x, s; E) &= C (\int + \int) \\
&= C \int \frac{1}{\frac{\hbar k}{\sqrt{2 m}}} \frac{e^{ikx}}{\frac{\hbar^2 k^2}{2m} - E} 2 \frac{\hbar^2}{2 m} k dk + C \int \\
&= C 2 \frac{\sqrt{2m}}{\hbar} \int\limits_{-\infty}^\infty \frac{e^{ikx}}{k^2 - \frac{2mE}{\hbar^2}} dk \\
&= \frac{1}{2 \pi \hbar} \sqrt{\frac{m}{2}} 2 \frac{\sqrt{2 m}}{\hbar} \int  \\
&= \frac{2m}{\hbar^2} \frac{i}{2 k_0} e^{i k_0 |x - s|}
\end{align*}
, where $k_0 = \frac{\sqrt{2 m E}}{\hbar}$. That was for positive energy. For negative energy:

$k_0 = \frac{\sqrt{2m|E|}}{\hbar}$.


$$\int\limits_{-\infty}^\infty \frac{e^{ik|x -s|}}{k^2 - E} dk = 2 \pi i \res\limits_{k = i k_0} f(k) = \frac{e^{i i k_0 |x - s|}}{2 i k_0} = 2 \pi \frac{e^{-k_0 |x - s|}}{2 k_0}$$

$$G(x, s; E) = \frac{2m}{\hbar^2} \frac{1}{2 k_0} e^{-k_0 |x - s|}$$

That is, we have to choose branch of the square root with positive imaginary part.

\subsection{Pipe}

Height of the pipe $H$.

$$\begin{cases}
X(x) = N_E e^{i k_x x} \\
Y(y) = \frac{2}{\sqrt{H}} \cos(k_y y) \\
k_x = \frac{\sqrt{2 \mu E}}{\hbar} \\
k_y = \frac{\pi m}{H} \\
\end{cases}$$

TODO don't forget the energy normalization and wavefunction normalization

\begin{comment}
$$\begin{cases}
\psi(x, y) = \frac{2}{\sqrt{L_x L_y}} \cos(k_x x) \cos(k_y y) \\
k_x = \frac{\pi n_x}{L_x} \\
E = \frac{\hbar^2 \pi^2}{2 m} \left(\frac{n_x^2}{L_x^2} + \frac{n_y^2}{L_y^2} \right) \\
\end{cases}$$
\end{comment}

\subsubsection{Green's function}

Suppose $M(E)$ is the count of open transversal modes for the given energy $E$.

\begin{align*}
G(x, y, x_s, y_s; E)
&= \int\limits_{0}^{\infty} \frac{\sum\limits_{m = 0}^{M(\lambda)} Y_m(y) Y^*_m(y_s) X'(x) X'^*(x_s)  }{\lambda - E} d \lambda \\
&= \sum_{m = 0}^\infty \int\limits_{E^Y_m}^\infty \frac{Y_m(y) Y^*_m(y_s) X'(x) X'^*(x_s)}{\lambda - E} d\lambda \\
&= \sum_{m = 0}^\infty \int\limits_{0}^\infty \frac{Y_m(y) Y_m^*(y_s) X(x) X^*(x_s)}{\lambda + E^y_m - E} d\lambda \\
&= \sum_{m = 0}^\infty Y_m(y) Y^*_m(y_s) \int\limits_{0}^\infty \frac{X(x) X^*(x_s)}{\lambda - (E - E^y_m)} d\lambda \\
&= \sum_{m = 0}^\infty Y_m(y) Y^*_m(y_s) G^x(x, x_s; E - E^y_m) \\
\end{align*}

TODO make clear that negative wavevectors were taken into account.

\end{document}