\documentclass[12pt, a4paper]{article}
%Gummi|065|=)
\usepackage[utf8]{inputenc}
\usepackage[T2A]{fontenc}
\usepackage{amsmath,amsthm,amsfonts,amssymb,amscd}
\usepackage[russian, english]{babel}
\usepackage{enumerate}
\usepackage{comment}
\usepackage{cite}
\usepackage{mathrsfs}
\usepackage{fullpage}
\usepackage{url}
\usepackage{tabularx}
\usepackage{graphicx}
\usepackage{indentfirst}
\usepackage{bm} % \bm for bold math in formulas
% \usepackage{braket} % \bra and \ket commands
\usepackage{physics}

\newcommand{\hilb}[1]{\mathcal{H}_{#1}}
\newcommand{\cconj}[1]{\overline{#1}}

\begin{document}

\section{Notations}

\subsection{Complex inner product}
There is some ambiguity in defining antilinearity property of the inner product in vector space $V$ over the field $\mathbb{C}$ of complex numbers:

\begin{itemize}
\item Physics: \textbf{anti}linear in the \textit{first} argument, linear in the \textit{second} argument:
\[
\forall x, y, z \in V: \forall a, b \in \mathbb{C}: \ip{ax + by}{z} = \cconj{a} \ip{x}{z} + \cconj{b} \ip{y}{z}
\]
\[
\forall x, y, z \in V: \forall a, b \in \mathbb{C}: \ip{x}{ay + bz} = a \ip{x}{y} + b \ip{x}{y}
\]
\item Mathematics: linear in the \textit{first} argument, \textbf{anti}linear in the \textit{second} argument:
\[
\forall x, y, z \in V: \forall a, b \in \mathbb{C}: \ip{ax + by}{z} = a \ip{x}{z} + b \ip{y}{z}
\]
\[
\forall x, y, z \in V: \forall a, b \in \mathbb{C}: \ip{x}{ay + bz} = \cconj{a} \ip{x}{y} + \cconj{b} \ip{x}{y}
\]
\end{itemize}

We stick to the PHYSICS definition. For instance, that means that inner product in $L^2(a, b)$ is defined as $\ip{f}{g} = \int\limits_a^b \cconj{f(x)} g(x) dx$.


\subsection{Green's function}

\subsubsection{As the kernel of the resolvent}
Consider the operator $L: X \to X$.

\begin{itemize}
\item Physics notation: $$R(\lambda) = \frac{1}{\lambda I - L}$$
\item Mathematics notation: $$R(\lambda) = \frac{1}{L - \lambda I}$$
\end{itemize} 

TODO: proof that it is an integral operator

$G(x, s; \lambda)$ is the kernel of $R(\lambda)$.


\subsubsection{Green's function of the differential operator}
Differential operator $L$, defined for distributions.

Green's function at point $s$ is a solution (TODO what if there are many?) of 

\begin{itemize}
\item Physics notation: $$L_x G(x, s) = -\delta(x - s)$$
\item Mathematics notation: $$L_x G(x, s) = \delta(x - s)$$
\end{itemize} 

That is, they differ up to sign. In particular, from the notation follows that eigenfunction expansion is defined as: 

\begin{itemize}
\item Physics notation: $$G(x, s) = -\sum\limits_n \frac{\psi_n(x) \psi_n^*(s)}{\lambda_n}$$
\item Mathematics notation: $$G(x, s) = \sum\limits_n \frac{\psi_n(x) \psi_n^*(s)}{\lambda_n}$$
\end{itemize} 

We use PHYSICS notation. 

\section{Short intro into quantum physics}
TODO

\section{Schrodinger's equation}

\subsection{Time-dependent}

$$i \hbar \pdv{\Psi(x, t)}{t} = H \Psi(x, t)$$

Particle in an electric (!!!) field:

$$H = - \frac{\hbar^2}{2 m} \laplacian + V(x, t)$$


% https://en.wikipedia.org/wiki/Hamiltonian_(quantum_mechanics)#Charged_particle_in_an_electromagnetic_field
TODO what about charge?

\subsection{Time-independent}

If $V(x, t)$ does not depend on time, we can separate the variables:

$\Psi(x, t) = \psi(x) T(t)$.

$$\begin{cases}
i \hbar \pdv{T(t)}{t} = E T(t) \\
(- \frac{\hbar^2}{2 m} \laplacian + V(x)) \psi(x) = E \psi(x)
\end{cases}$$

Now that we have time-independent states, we can represent any initial state $\psi_0$ (TODO: complete set of eigenstates) in its basis:

$$\psi_0 = \sum\limits_\lambda A_\lambda \psi_\lambda$$

$$\psi(x, t) = \sum\limits_\lambda A_\lambda \psi_\lambda(x) e^{- \frac{i}{\hbar} E_\lambda t}$$

\section{Reflection coefficient and transmission coefficient}
Reflection/transmission coefficients are quantities describing the behaviour of scattered wave (usually at infinity).

Probability current for spin-0 particle.

% TODO https://en.wikipedia.org/wiki/Probability_current#Spin-0_particle_in_an_electromagnetic_field

$$\bm{j}(x, t) = \frac{\hbar}{2 m i} (\psi(x, t)^* \grad \psi(x, t) - \psi(x, t) \grad \psi^*(x, t))$$

TODO: write about stationary (scattering) states and omitting time.

$$T = \frac{|j_{trans}|}{|j_{inc}|}$$

$$R = \frac{|j_{refl}|}{|j_{inc}|}$$

These are position dependent and non-scalar values in general. Usually, geometry and symmetries of the scattering problem allows to define the asymptotic region and calculate its integral over scattering cross section, yielding a scalar value.

\section{Free particle in magnetic field}
$$H = \frac{1}{2m} (- i \hbar \grad - q A)^2$$

Field $B = (0, 0, B)$, vector potential $A = (0, \frac{1}{2} B r, 0)$. 

\begin{comment}

$$H = \frac{1}{2m}( - \hbar^2 \nabla^2 + i q \hbar (\nabla \cdot A + A \cdot \nabla) + q^2 A \cdot A)$$

\begin{itemize}
\item In this gauge, $\nabla \cdot A = 0$.
\item $A \cdot \nabla = \frac{1}{2} B \frac{\partial \cdot}{\partial \theta}$
\item $A \cdot A = \frac{1}{4} B^2 r^2$
\end{itemize}

$$H = \frac{1}{2m} (- \hbar^2 \nabla^2 + i q \hbar \frac{1}{2} B \frac{\partial \cdot}{\partial \theta} + q^2 \frac{1}{4} B^2 r^2)$$

$$\Delta = \nabla^2 \cdot = {1 \over r}{\partial \over \partial r}\left(r {\partial \cdot \over \partial r}\right)
+ {1 \over r^2}{\partial^2 \cdot \over \partial \theta^2}
+ {\partial^2 \cdot \over \partial z^2}$$

Separate the variables:

$\psi(r, \theta, z) = C(r, \theta) T(z)$

$$\nabla^2 \psi = T(z) {1 \over r}{\partial \over \partial r}\left(r {\partial C \over \partial r}\right)
+ T(z) {1 \over r^2}{\partial^2 C \over \partial \theta^2}
+ C(r, \theta) {\partial^2 T \over \partial z^2}$$

$$H \psi = \frac{1}{2m} \left( -\hbar^2 \nabla^2 C(r, \theta) T(z) + i q \hbar \frac{1}{2} B T(z) \frac{\partial C}{\partial \theta} + q^2 \frac{1}{4} B^2 r^2 C(r, \theta) T(z) \right) = E \psi$$

Divide all by $\psi(r, \theta, z)$:

$$-\hbar^2 
\left( 
{1 \over r} \frac{1}{C(r, \theta)} {\partial \over \partial r} \left(r {\partial C \over \partial r}\right)
+ {1 \over r^2} \frac{1}{C(r, \theta)}{\partial^2 C \over \partial \theta^2}
+  \frac{1}{T(z)}{\partial^2 T \over \partial z^2}
\right) 
 + i q \hbar \frac{1}{2} B  \frac{1}{C(r, \theta)} \frac{\partial C}{\partial \theta} + q^2 \frac{1}{4} B^2 r^2 = 2m E$$

$$\begin{cases}
- \hbar^2 \frac{1}{T(z)}{\partial^2 T \over \partial z^2} = E_z \\
-\hbar^2 
\left( 
{1 \over r} \frac{1}{C(r, \theta)} {\partial \over \partial r} \left(r {\partial C \over \partial r}\right)
+ {1 \over r^2} \frac{1}{C(r, \theta)}{\partial^2 C \over \partial \theta^2}
\right) 
 + i q \hbar \frac{1}{2} B  \frac{1}{C(r, \theta)} \frac{\partial C}{\partial \theta} + q^2 \frac{1}{4} B^2 r^2 = E_{r\theta} \\
 E_{r \theta} + E_z = 2 m E
\end{cases}$$

Second separation: $C(r, \theta) = R(r) \Theta(\theta)$:

$$-\hbar^2 
\left( 
{1 \over r} \frac{1}{R(r) \Theta(\theta)} \Theta(\theta) {\partial \over \partial r} \left(r {\partial R \over \partial r}\right)
+ {1 \over r^2} \frac{1}{R(r) \Theta(\theta)} R(r) {\partial^2 \Theta \over \partial \theta^2}
\right) 
 + i q \hbar \frac{1}{2} B  \frac{1}{R(r) \Theta(\theta)} R(r) \frac{\partial \Theta}{\partial \theta} + q^2 \frac{1}{4} B^2 r^2 = E_{r\theta}$$

$$\begin{cases}

\end{cases}$$

\end{comment}

Polar coordinates:

$$\renewcommand\arraystretch{1.5}
(-i \hbar \nabla - q A) \psi =
\begin{pmatrix}
-i \hbar \frac{\partial \psi}{\partial r} \\
-i \hbar \frac{1}{r} \frac{\partial \psi}{\partial \theta} - q \frac{1}{2} B r \psi  \\
-i \hbar \frac{\partial \psi}{\partial z} 
\end{pmatrix}
$$

$$
(-i \hbar \nabla - qA) \cdot F =
-i \hbar \frac{1}{r} \frac{\partial (r F_r)}{\partial r}
-i \hbar  \frac{1}{r} \frac{\partial F_\theta}{\partial \theta} - q \frac{1}{2} B r F_\theta
-i \hbar  \frac{\partial F_z}{\partial z}
$$

\begin{align*}
2 m H \psi
&= - \hbar^2 \frac{1}{r} \frac{\partial}{\partial r} \left( r \frac{\partial \psi}{\partial r}\right ) \\
&- \hbar^2 \frac{1}{r^2} \frac{\partial^2 \psi}{\partial \theta^2} + i \hbar \frac{1}{r} q \frac{1}{2} B r \frac{\partial \psi}{\partial \theta} + q \frac{1}{2} B r i \hbar \frac{1}{r} \frac{\partial \psi}{\partial \theta} + q^2 \frac{1}{4} B^2 r^2 \psi \\
&- \hbar^2 \frac{\partial^2 \psi}{\partial z^2} \\
&= {\hat p_r}^2 - \frac{i \hbar}{r} {\hat p_r} \\
&+ ({\hat p_\theta} - \frac{1}{2} q B r)^2\\
&+ {\hat p_z}^2 \\
\end{align*}

Commutes with $p_z$ and seems to commute with $p_\lambda$.

Eigenvectors of $p_z$:
$$\psi(r, \theta, z) = \xi(r, \theta) e^{\frac{i}{\hbar} p_z z}$$

Eigenvectors of $p_\lambda$:
$$\psi(r, \theta, z) = \xi(r, z) e^{\frac{i}{\hbar} p_\lambda \theta}$$

Periodicity requires $\frac{1}{\hbar} p_\lambda 2 \pi = 2 \pi k, k \in \mathbb{Z}$, therefore, $p_\lambda = \hbar k, k \in \mathbb{Z}$.

$$\psi(r, \theta, z) = R(r) e^{i k \theta} e^{\frac{i}{\hbar} p_z z}$$

Substitute in the equation:

\begin{align*}
2 m H \psi
&= -\hbar^2 \left( \frac{\partial^2 \psi}{\partial r^2} + \frac{1}{r} \frac{\partial \psi}{\partial r} \right) =
e^{...} (-\hbar^2 R''(r) -\frac{\hbar^2}{r} R'(r)) \\
&+ (\frac{1}{r} \hbar k - \frac{1}{2} q B r)^2 \psi \\
&+ p_z^2  \psi \\
&= 2 m E \psi
\end{align*}

Divide everything by $e^{...}$:

$$- \hbar^2 \left( \frac{\partial^2 R}{\partial r^2} + \frac{1}{r} \frac{\partial R}{\partial r} \right)
+ \left(
(\frac{1}{r} \hbar k - \frac{1}{2} q B r)^2
+ p^2_z
- 2 m E
\right)
R(r) = 0$$

Divide by $-\hbar^2$:

$$\frac{\partial^2 R}{\partial r^2} + \frac{1}{r} \frac{\partial R}{\partial r}
- \frac{1}{\hbar^2}
\left(
(\frac{1}{r} \hbar k - \frac{1}{2} q B r)^2
+ p^2_z
- 2 m E
\right)
R(r) = 0$$

Use substitution: $R(r) = \frac{U(r)}{\sqrt{r}}$

$$\left(
\frac{1}{\sqrt r} \frac{\partial^2 U}{\partial r^2}
+ \frac{1}{4 r^{5/2}} U(r) \right)
- \frac{1}{\hbar^2}
\left(
(\frac{1}{r} \hbar k - \frac{1}{2} q B r)^2
+ p^2_z
- 2 m E
\right)
\frac{U(r)}{\sqrt{r}} = 0$$

Multiply all by $\sqrt{r}$:

$$\frac{\partial^2 U}{\partial r^2}
+ \frac{1}{4 r^2} U(r)
+ \left(
- \frac{1}{\hbar^2} (\frac{1}{r} \hbar k - \frac{1}{2} q B r)^2
+ \frac{2 m E - p_z^2}{\hbar^2}
\right)
U(r) = 0$$


If $B = 0$, reduces to 

$$\frac{\partial^2 U}{\partial r^2}
+ \left(
\frac{1}{4 r^2}
- \frac{k^2}{r^2}
+ \frac{2 m E - p_z^2}{\hbar^2}
\right)
U(r) = 0$$

With solution:

$$U(r) = c_1 \sqrt{r} J_k\left(\frac{\sqrt{\text{EE}} r}{\hbar}\right)+c_2 \sqrt{r} Y_k\left(\frac{\sqrt{\text{EE}} r}{\hbar}\right)$$

\begin{itemize}
\item $J_k$: Bessel's function of first kind
\item $Y_k$: Bessel's function of second kind, singularity at origin
\end{itemize}

Looks kinda correct.



% https://en.wikipedia.org/wiki/Quantum_harmonic_oscillator#N-dimensional_harmonic_oscillator

\begin{comment}
Multiply all by $r^2$:

$$\left(
- \hbar^2 \left( r^2 \frac{\partial^2 R}{\partial r^2} + r \frac{\partial R}{\partial r} \right)
+ \hbar^2 k^2
+ \frac{1}{4} q^2 B^2 r^4
+ r^2 (p^2_z
- 2 m E
- q B \hbar k)
\right) R(r) = 0$$
\end{comment}

\begin{comment}

\subsubsection{Another gauge}
$$
A = \begin{pmatrix}
- B r \theta \\
0 \\
0
\end{pmatrix}
$$

$$\renewcommand\arraystretch{1.5}
(-i \hbar \nabla - q A) \psi =
\begin{pmatrix}
-i \hbar \frac{\partial \psi}{\partial r} + q B r \theta \psi \\
-i \hbar \frac{1}{r} \frac{\partial \psi}{\partial \theta}\\
-i \hbar \frac{\partial \psi}{\partial z}
\end{pmatrix}
$$


\begin{align*}
2 m H \psi
&= - \hbar^2 \frac{1}{r} \frac{\partial}{\partial r} \left( r \frac{\partial \psi}{\partial r}\right ) \\
&- \hbar^2 \frac{1}{r^2} \frac{\partial^2 \psi}{\partial \theta^2} \\
&- \hbar^2 \frac{\partial^2 \psi}{\partial z^2}
\end{align*}

Apparenly, this gauge is not nice one :(

\end{comment}

\section{Point interactions}
Let $H_0$ be an unbounded s.a. operator with the resolvent $R_0(z) = (H_0 - z I)^{-1}$ in a Hilbert space $\hilb{0}$. We introduce a scale of partial inner product spaces:

\[ \dots \subseteq \hilb{m} \subseteq \dots \subseteq \hilb{1} \subseteq \hilb{0} \subseteq \hilb{-1} \subseteq \dots \subseteq \hilb{-m} \subseteq \dots \]

Where $\hilb{k}$ is the set

\[
\{\bra{\phi} R_0^k(z_0) \mid \bra{\phi} \in \hilb{0}\}
\]
, where $z_0$ is any complex number from the resolvent set $\rho(H_0)$ (the scale is independent of its choice).

The following fact is extremely useful (and in fact the motivation for scale choice): If $\ket{\psi} \in \hilb{k}$, then: $H_0 \ket{\psi} \in \hilb{k + 1}$.

Proof:

\begin{enumerate}
\item Since $\ket{\psi} \in \hilb{k}$, we have $\bra{\psi} \in \hilb{k}$,
\item Since $\bra{\psi} \in \hilb{-k}$, for some $\bra{\phi} \in \hilb{0}$, $\bra{\psi} = \bra{\phi} R_0^{-k}$
\item $H_0 \ket{\psi}$ is dual to $\bra{\psi} H_0^{\dagger} = \bra{\psi} H_0$
\item $\bra{\psi} H_0 = \bra{\phi} R_0^{-k} H_0 = \bra{\phi} R_0^{-k - 1} \frac{1}{H_0 - \lambda I} H_0 = (\bra{\phi} + \lambda \bra{\phi} R_0) R_0^{-k - 1}\in \hilb{-k - 1}$, since the element $\bra{\phi} + \lambda \bra{\phi} R_0 \in \hilb{0}$
\item By the duality, we have $H_0 \ket\psi \in \hilb{k + 1}$
\end{enumerate}



Note that due to the choice of the operator defining the scale (the resolvent operator), if some function $f \in \hilb{k}$, we have $H_0 f \in \hilb{k - 1}$. Thus, the action of the differential operator is a right shift on the scale.

Note that in some sense, the more space to the left of the scale, the better functions in this space behave in the sense of action of differential operator:


TODO: somewhat fucked-up, bra-ket notation is being used before defining inner product :(

\begin{enumerate}
\item $\hilb{0} = L^2(a, b)$ with the usual inner product $\ip{f}{g}_0 = \int\limits_a^b \cconj{f(x)} g(x) \dd{x}$.
\item Let $\hilb{k} = \{ R^k_0(z_0) \phi \mid \phi \in \hilb{0} \}$.

For $k > 0$, $R_0^k(z_0)$ is a bounded linear operator (by the definition of the resolvent). For any $\phi \in \hilb{0}: R_0(z_0) \phi \in \hilb{0}$, which means $\hilb{1} \subseteq \hilb{0}$ and by induction we immediately get the positive scale
\[
\dots \subseteq \hilb{m} \subseteq \dots \subseteq \hilb{2} \subseteq \hilb{1} \subseteq \hilb{0}
\] 

For $k < 0$, $\hilb{k}$ actually is $\{ (H_0 - z_0 I)^k \phi \mid \phi \in \hilb{0} \}$.
For any $\phi \in \hilb{0}$, we can find such a $\psi \in \hilb{0}$, that $(H_0 - z_0 I) \psi = \phi$. Indeed, $\psi = R_0(z_0) \phi$. That means $\hilb{0} \subseteq \hilb{-1}$, and by induction, we get the negative scale
\[
\hilb{0} \subseteq \hilb{-1} \subseteq \hilb{-2} \subseteq \dots \subseteq \hilb{-m} \subseteq \dots
\] 
\item For each $k > 0$, we call the pair of spaces $\hilb{-k}, \hilb{k}$ \textit{compatible}.

Define the inner product on pairs of compatible spaces: for $\psi \in \hilb{-k}$, $\varphi \in \hilb{k}$:
\[
\ip{\psi}{\varphi} = (def) = \ip{R_0^k(\cconj{z_0}) \psi}{R_0^{-k}(z_0)\varphi}_0
\]
TODO HOW TO DEFINE??? :(((
\begin{comment}
\begin{itemize}
\item Axioms are kind of obvious
\item Existence: $\psi = R_0^{-k}(z_0) \phi_1$, $\varphi = R_0^k(z_0) \phi_2$, therefore \\
$\ip{\psi}{\varphi} = \ip{R_0^k(\cconj{z_0}) R_0^{-k}(z_0) \phi_1}{R_0^{-k}(z_0) R_0^k(z_0) \phi_2}$
\end{itemize}
\end{comment}

% \item Let us denote the elements of $\hilb{k}$ as ket vectors, that is, $\ket{\psi} \in \hilb{k}$.
\end{enumerate}

We want to define the operator $H_0 + \ket{\psi} \lambda \bra{\psi}$ (somewhat like $\delta$ potential?).

Suppose $\bra{\psi} \in \hilb{-m - 1}$, $m \ge 1$. TODO what does that mean? Why $m \ge 1$

$\hilb{0}$ has to be extended with so-called generalized defect element $R_0(z) \ket{\psi}$. TODO why not $z_0$? Why ket? Isn't it from $\hilb{m + 1} \subseteq \hilb{0}$?


\subsection{Constructing Pontryagin space}
Construct a submanifold $\mathcal{P}_m$ of $\hilb{-m}$.

\begin{enumerate}
\item Contains the defect element $R_0(z_0) \psi$
\item As little from $\hilb{-m} \setminus \hilb{0}$ as possible.
\item As much from $\hilb{0}$ as possible.
\item Inner product $\ip{}{}$ on $\mathcal{P}_m$ is an extension of the partial inner product $\ip{}{}_0$.
\item $R_0(z_0)$ is the resolvent of an s.a. (w.r.t. to $\ip{}{}$) opeartor in $\mathcal{P}_m$
\end{enumerate}

\[
\mathcal{P}_m = 
\{ \varphi_m + \sum\limits_{i = -m}^{m - 1} c_i \psi_i \mid \varphi_m \in \hilb{m}, c_i \in \mathbb{C}\}
\]

, where $\psi_i = R_0^{m + i + 1}(z_0) \psi$, that is, $\psi_i \in \hilb{i}$.

Define inner product:

\[
\ip{\psi}{\psi'} =
\ip{\varphi_m}{\varphi'_m}_0 +
\sum\limits_{i = -m}^{m - 1} \cconj{c_i} \ip{\psi_i}{\varphi'_m}_0 +
\sum\limits_{j = -m}^{m - 1} c'_j \ip{\varphi_m}{\psi_j}_0 +
\sum\limits_{i = -m}^{m - 1} \sum\limits_{j = -m}^{m - 1} \cconj{c_i} c_j' g_{ij}
\]

We have freedom in the choice of $g_{ij} \ip{\psi_i}{\psi_j} |_{ij}$. There are some constraints, though:

\begin{itemize}
\item $g_{ij}$ should be a Hermetian matrix to retain the conjugate symmetry
\item Some of the elements are already defined by the partial inner product: if $i + j \ge 0$, $\ip{f \in \hilb{i}}{g \in \hilb{j}}$ is properly defined.
\item TODO weird resolvent costraint: $\mel{f}{R_0(z_0)}{g} = \cconj{\mel{g}{R_0(\cconj{z_0})}{f}}$, which means
\[
g_{i + 1, j} - g_{i, j + 1} = (z_0 - \cconj{z_0}) g_{i + 1, j + 1}
\]
\end{itemize}

TODO still some ambiguity?

Next, define 
\[
\Pi_m = \{f = (\phi_f, \vb{p_f}, \vb{n_f}) \mid \phi_f \in \hilb{0}, \vb{p_f}, \vb{n_f} \in \mathbb{C}^m \}
\]
with indefinite inner product:
\[
\ip{f}{g} =
\ip{\phi_f}{\phi_g}_0 +
\cconj{\vb{p_f}} \vdot \vb{n_g} +
\cconj{\vb{n_f}} \vdot \vb{p_g} + 
\cconj{\vb{n_f}} \vdot \vb{g_{ij}} \vdot \vb{n_g}
\]

TODO why positive parts are not interacting?

Norm topology: product topology of $\hilb{0} \oplus \mathbb{C}^m \oplus \mathbb{C}^m$. TODO so what?

We identify $\vb{g_{ij}}$ with that of $\mathcal{P}_m$. Then $\mathcal{P}_m$ is a pre-Pontryagin space with topological completion $\Pi_m$. TODO whaaaat?

Embedding of $\mathcal{P}_m$ in $\Pi_m$:

\[
\varphi_m + \sum\limits_{i = -m}^{m - 1} c_i \psi_i \mapsto
\left(
\varphi_m + \sum\limits_{i = 0}^{m - 1} c_i \psi_i,
\left[ \ip{\psi_i}{\varphi_m}_0 + \sum\limits_{j = 0}^{m - 1} c_j g_{ij} \right]_{i = -1}^{-m},
\left[ c_i \right]_{i = -1}^{-m}
\right)
\]

Then restrict the mapping to the submanifold $\mathcal{P}'_m$ of $\mathcal{P}_m$, which is composed of elements $\varphi_m + \sum\limits_{i = - m}^{-1} c_i \psi_i$. $\mathcal{P}'_m$ is topologically dense in $\Pi_m$ because:

\begin{itemize}
\item $\hilb{m}$ is dense in $\hilb{0}$ (okay)
\item the functionals corresponding to $\psi_i$ for $i \le -1$ are unbounded (TODO so what?)
\end{itemize}

\section{Notes}
\subsection{Cylindrical coordinates}
$$\vb{r} = (r, \theta, z)$$. 


$$\hat {p_r} = - i \hbar \pdv{r}$$

$$\hat{p_\theta} = - i \hbar \frac{1}{r} \pdv{\theta}$$

$$\hat{p_z} = - i \hbar \pdv{z}$$

$$\renewcommand\arraystretch{1.5}
\gradient f = 
\begin{pmatrix}
\pdv{f}{r} \\
\frac{1}{r} \pdv{f}{\theta}  \\
\pdv{f}{z} 
\end{pmatrix}
$$


$$
\divergence F =
\frac{1}{r} \pdv{(r F_r)}{r}
+ \frac{1}{r} \pdv{F_\theta}{\theta}
+ \pdv{F_z}{z}
$$


\end{document}