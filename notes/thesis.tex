\documentclass[12pt, a4paper]{article}
%Gummi|065|=)
\usepackage[utf8]{inputenc}
\usepackage[T2A]{fontenc}
\usepackage{amsmath,amsthm,amsfonts,amssymb,amscd}
\usepackage[russian, english]{babel}
\usepackage{enumerate}
\usepackage{comment}
\usepackage{cite}
\usepackage{mathrsfs}
\usepackage{fullpage}
\usepackage{url}
\usepackage{tabularx}
\usepackage{graphicx}
\usepackage{indentfirst}
\usepackage{bm}
\usepackage{braket}

\def\res{\mathop{\operatorname{Res}}\limits}

\begin{document}

\section{Notations}

\subsection{Green's function}

\subsubsection{As the kernel of the resolvent}
Consider the operator $L: X \to X$.

\begin{itemize}
\item Physics notation: $$R(\lambda) = \frac{1}{\lambda I - L}$$
\item Mathematics notation: $$R(\lambda) = \frac{1}{L - \lambda I}$$
\end{itemize} 

TODO: proof that it is an integral operator

$G(x, s; \lambda)$ is the kernel of $R(\lambda)$.


\subsubsection{Green's function of the differential operator}
Differential operator $L$, defined for distributions.

Green's function at point $s$ is a solution (TODO what if there are many?) of 

\begin{itemize}
\item Physics notation: $$L_x G(x, s) = -\delta(x - s)$$
\item Mathematics notation: $$L_x G(x, s) = \delta(x - s)$$
\end{itemize} 

That is, they differ up to sign. We use PHYSICS notation.

Eigenfunction expansion: 

\begin{itemize}
\item Physics notation: $$G(x, s) = -\sum\limits_n \frac{\psi_n(x) \psi_n^*(s)}{\lambda_n}$$
\item Mathematics notation: $$G(x, s) = \sum\limits_n \frac{\psi_n(x) \psi_n^*(s)}{\lambda_n}$$
\end{itemize} 

\section{Short intro into quantum physics}
TODO

\section{Schrodinger's equation}

\subsection{Time-dependent}

$$i \hbar \partial_t \Psi(x, t) = H \Psi(x, t)$$

Particle in an electric (!!!) field:

$$H = - \frac{\hbar^2}{2 m} \nabla^2 + V(x, t)$$

TODO what about charge?

\subsection{Time-independent}

If $V(x, t)$ does not depend on time, we can separate the variables:

$\Psi(x, t) = \psi(x) T(t)$.

$$\begin{cases}
i \hbar \partial_t T(t) = E T(t) \\
(- \frac{\hbar^2}{2 m} \nabla^2 + V(x)) \psi(x) = E \psi(x)
\end{cases}$$

Now that we have time-independent states, we can represent any initial state $\psi_0$ (TODO: complete set of eigenstates) in its basis:

$$\psi_0 = \sum\limits_\lambda A_\lambda \psi_\lambda$$

$$\psi(x, t) = \sum\limits_\lambda A_\lambda \psi_\lambda(x) e^{- \frac{i}{\hbar} E_\lambda t}$$

\end{document}
