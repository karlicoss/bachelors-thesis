\documentclass[12pt, a4paper]{article}
%Gummi|065|=)
\usepackage[utf8]{inputenc}
\usepackage[T2A]{fontenc}
\usepackage{amsmath,amsthm,amsfonts,amssymb,amscd}
\usepackage[russian, english]{babel}
\usepackage{enumerate}
\usepackage{comment}
\usepackage{cite}
\usepackage{mathrsfs}
\usepackage{fullpage}
\usepackage{url}
\usepackage{tabularx}
\usepackage{graphicx}
\usepackage{indentfirst}
\usepackage{bm}
\usepackage{braket}

\def\res{\mathop{\operatorname{Res}}\limits}

\begin{document}

\section{Notations}

\subsection{Green's function}

\subsubsection{As the kernel of the resolvent}
Consider the operator $L: X \to X$.

\begin{itemize}
\item Physics notation: $$R(\lambda) = \frac{1}{\lambda I - L}$$
\item Mathematics notation: $$R(\lambda) = \frac{1}{L - \lambda I}$$
\end{itemize} 

TODO: proof that it is an integral operator

$G(x, s; \lambda)$ is the kernel of $R(\lambda)$.


\subsubsection{Green's function of the differential operator}
Differential operator $L$, defined for distributions.

Green's function at point $s$ is a solution (TODO what if there are many?) of 

\begin{itemize}
\item Physics notation: $$L_x G(x, s) = -\delta(x - s)$$
\item Mathematics notation: $$L_x G(x, s) = \delta(x - s)$$
\end{itemize} 

That is, they differ up to sign. We use PHYSICS notation.

Eigenfunction expansion: 

\begin{itemize}
\item Physics notation: $$G(x, s) = -\sum\limits_n \frac{\psi_n(x) \psi_n^*(s)}{\lambda_n}$$
\item Mathematics notation: $$G(x, s) = \sum\limits_n \frac{\psi_n(x) \psi_n^*(s)}{\lambda_n}$$
\end{itemize} 

\section{Short intro into quantum physics}
TODO

\section{Schrodinger's equation}

\subsection{Time-dependent}

$$i \hbar \partial_t \Psi(x, t) = H \Psi(x, t)$$

Particle in an electric (!!!) field:

$$H = - \frac{\hbar^2}{2 m} \nabla^2 + V(x, t)$$


% https://en.wikipedia.org/wiki/Hamiltonian_(quantum_mechanics)#Charged_particle_in_an_electromagnetic_field
TODO what about charge?

\subsection{Time-independent}

If $V(x, t)$ does not depend on time, we can separate the variables:

$\Psi(x, t) = \psi(x) T(t)$.

$$\begin{cases}
i \hbar \partial_t T(t) = E T(t) \\
(- \frac{\hbar^2}{2 m} \nabla^2 + V(x)) \psi(x) = E \psi(x)
\end{cases}$$

Now that we have time-independent states, we can represent any initial state $\psi_0$ (TODO: complete set of eigenstates) in its basis:

$$\psi_0 = \sum\limits_\lambda A_\lambda \psi_\lambda$$

$$\psi(x, t) = \sum\limits_\lambda A_\lambda \psi_\lambda(x) e^{- \frac{i}{\hbar} E_\lambda t}$$

\section{Reflection coefficient and transmission coefficient}
Reflection/transmission coefficients are quantities describing the behaviour of scattered wave (usually at infinity).

Probability current for spin-0 particle.

% TODO https://en.wikipedia.org/wiki/Probability_current#Spin-0_particle_in_an_electromagnetic_field

$$\bm{j}(x, t) = \frac{\hbar}{2 m i} (\psi(x, t)^* \nabla \psi(x, t) - \psi(x, t) \nabla \psi^*(x, t))$$

TODO: write about stationary (scattering) states and omitting time.

$$T = \frac{|j_{trans}|}{|j_{inc}|}$$

$$R = \frac{|j_{refl}|}{|j_{inc}|}$$

These are position dependent and non-scalar values in general. Usually, geometry and symmetries of the scattering problem allows to define the asymptotic region and calculate its integral over scattering cross section, yielding a scalar value.

\section{Free particle in magnetic field}
$$H = \frac{1}{2m} (- i \hbar \nabla - q A)^2$$

Field $B = (0, 0, B)$, vector potential $A = (0, \frac{1}{2} B r, 0)$. 

\begin{comment}

$$H = \frac{1}{2m}( - \hbar^2 \nabla^2 + i q \hbar (\nabla \cdot A + A \cdot \nabla) + q^2 A \cdot A)$$

\begin{itemize}
\item In this gauge, $\nabla \cdot A = 0$.
\item $A \cdot \nabla = \frac{1}{2} B \frac{\partial \cdot}{\partial \theta}$
\item $A \cdot A = \frac{1}{4} B^2 r^2$
\end{itemize}

$$H = \frac{1}{2m} (- \hbar^2 \nabla^2 + i q \hbar \frac{1}{2} B \frac{\partial \cdot}{\partial \theta} + q^2 \frac{1}{4} B^2 r^2)$$

$$\Delta = \nabla^2 \cdot = {1 \over r}{\partial \over \partial r}\left(r {\partial \cdot \over \partial r}\right)
+ {1 \over r^2}{\partial^2 \cdot \over \partial \theta^2}
+ {\partial^2 \cdot \over \partial z^2}$$

Separate the variables:

$\psi(r, \theta, z) = C(r, \theta) T(z)$

$$\nabla^2 \psi = T(z) {1 \over r}{\partial \over \partial r}\left(r {\partial C \over \partial r}\right)
+ T(z) {1 \over r^2}{\partial^2 C \over \partial \theta^2}
+ C(r, \theta) {\partial^2 T \over \partial z^2}$$

$$H \psi = \frac{1}{2m} \left( -\hbar^2 \nabla^2 C(r, \theta) T(z) + i q \hbar \frac{1}{2} B T(z) \frac{\partial C}{\partial \theta} + q^2 \frac{1}{4} B^2 r^2 C(r, \theta) T(z) \right) = E \psi$$

Divide all by $\psi(r, \theta, z)$:

$$-\hbar^2 
\left( 
{1 \over r} \frac{1}{C(r, \theta)} {\partial \over \partial r} \left(r {\partial C \over \partial r}\right)
+ {1 \over r^2} \frac{1}{C(r, \theta)}{\partial^2 C \over \partial \theta^2}
+  \frac{1}{T(z)}{\partial^2 T \over \partial z^2}
\right) 
 + i q \hbar \frac{1}{2} B  \frac{1}{C(r, \theta)} \frac{\partial C}{\partial \theta} + q^2 \frac{1}{4} B^2 r^2 = 2m E$$

$$\begin{cases}
- \hbar^2 \frac{1}{T(z)}{\partial^2 T \over \partial z^2} = E_z \\
-\hbar^2 
\left( 
{1 \over r} \frac{1}{C(r, \theta)} {\partial \over \partial r} \left(r {\partial C \over \partial r}\right)
+ {1 \over r^2} \frac{1}{C(r, \theta)}{\partial^2 C \over \partial \theta^2}
\right) 
 + i q \hbar \frac{1}{2} B  \frac{1}{C(r, \theta)} \frac{\partial C}{\partial \theta} + q^2 \frac{1}{4} B^2 r^2 = E_{r\theta} \\
 E_{r \theta} + E_z = 2 m E
\end{cases}$$

Second separation: $C(r, \theta) = R(r) \Theta(\theta)$:

$$-\hbar^2 
\left( 
{1 \over r} \frac{1}{R(r) \Theta(\theta)} \Theta(\theta) {\partial \over \partial r} \left(r {\partial R \over \partial r}\right)
+ {1 \over r^2} \frac{1}{R(r) \Theta(\theta)} R(r) {\partial^2 \Theta \over \partial \theta^2}
\right) 
 + i q \hbar \frac{1}{2} B  \frac{1}{R(r) \Theta(\theta)} R(r) \frac{\partial \Theta}{\partial \theta} + q^2 \frac{1}{4} B^2 r^2 = E_{r\theta}$$

$$\begin{cases}

\end{cases}$$

\end{comment}

Polar coordinates:

$$\renewcommand\arraystretch{1.5}
(-i \hbar \nabla - q A) \psi =
\begin{pmatrix}
-i \hbar \frac{\partial \psi}{\partial r} \\
-i \hbar \frac{1}{r} \frac{\partial \psi}{\partial \theta} - q \frac{1}{2} B r \psi  \\
-i \hbar \frac{\partial \psi}{\partial z} 
\end{pmatrix}
$$

$$
(-i \hbar \nabla - qA) \cdot F =
-i \hbar \frac{1}{r} \frac{\partial (r F_r)}{\partial r}
-i \hbar  \frac{1}{r} \frac{\partial F_\theta}{\partial \theta} - q \frac{1}{2} B r F_\theta
-i \hbar  \frac{\partial F_z}{\partial z}
$$

\begin{align*}
2 m H \psi
&= - \hbar^2 \frac{1}{r} \frac{\partial}{\partial r} \left( r \frac{\partial \psi}{\partial r}\right ) \\
&- \hbar^2 \frac{1}{r^2} \frac{\partial^2 \psi}{\partial \theta^2} + i \hbar \frac{1}{r} q \frac{1}{2} B r \frac{\partial \psi}{\partial \theta} + q \frac{1}{2} B r i \hbar \frac{1}{r} \frac{\partial \psi}{\partial \theta} + q^2 \frac{1}{4} B^2 r^2 \psi \\
&- \hbar^2 \frac{\partial^2 \psi}{\partial z^2} \\
&= {\hat p_r}^2 - \frac{i \hbar}{r} {\hat p_r} \\
&+ ({\hat p_\theta} - \frac{1}{2} q B r)^2\\
&+ {\hat p_z}^2 \\
\end{align*}

Commutes with $p_z$ and seems to commute with $p_\lambda$.

Eigenvectors of $p_z$:
$$\psi(r, \theta, z) = \xi(r, \theta) e^{\frac{i}{\hbar} p_z z}$$

Eigenvectors of $p_\lambda$:
$$\psi(r, \theta, z) = \xi(r, z) e^{\frac{i}{\hbar} p_\lambda \theta}$$

Periodicity requires $\frac{1}{\hbar} p_\lambda 2 \pi = 2 \pi k, k \in \mathbb{Z}$, therefore, $p_\lambda = \hbar k, k \in \mathbb{Z}$.

$$\psi(r, \theta, z) = R(r) e^{i k \theta} e^{\frac{i}{\hbar} p_z z}$$

Substitute in the equation:

\begin{align*}
2 m H \psi
&= -\hbar^2 \left( \frac{\partial^2 \psi}{\partial r^2} + \frac{1}{r} \frac{\partial \psi}{\partial r} \right) =
e^{...} (-\hbar^2 R''(r) -\frac{\hbar^2}{r} R'(r)) \\
&+ (\frac{1}{r} \hbar k - \frac{1}{2} q B r)^2 \psi \\
&+ p_z^2  \psi \\
&= 2 m E \psi
\end{align*}

Divide everything by $e^{...}$:

$$- \hbar^2 \left( \frac{\partial^2 R}{\partial r^2} + \frac{1}{r} \frac{\partial R}{\partial r} \right)
+ \left(
(\frac{1}{r} \hbar k - \frac{1}{2} q B r)^2
+ p^2_z
- 2 m E
\right)
R(r) = 0$$

Divide by $-\hbar^2$:

$$\frac{\partial^2 R}{\partial r^2} + \frac{1}{r} \frac{\partial R}{\partial r}
- \frac{1}{\hbar^2}
\left(
(\frac{1}{r} \hbar k - \frac{1}{2} q B r)^2
+ p^2_z
- 2 m E
\right)
R(r) = 0$$

Use substitution: $R(r) = \frac{U(r)}{\sqrt{r}}$

$$\left(
\frac{1}{\sqrt r} \frac{\partial^2 U}{\partial r^2}
+ \frac{1}{4 r^{5/2}} U(r) \right)
- \frac{1}{\hbar^2}
\left(
(\frac{1}{r} \hbar k - \frac{1}{2} q B r)^2
+ p^2_z
- 2 m E
\right)
\frac{U(r)}{\sqrt{r}} = 0$$

Multiply all by $\sqrt{r}$:

$$\frac{\partial^2 U}{\partial r^2}
+ \frac{1}{4 r^2} U(r)
+ \left(
- \frac{1}{\hbar^2} (\frac{1}{r} \hbar k - \frac{1}{2} q B r)^2
+ \frac{2 m E - p_z^2}{\hbar^2}
\right)
U(r) = 0$$


If $B = 0$, reduces to 

$$\frac{\partial^2 U}{\partial r^2}
+ \left(
\frac{1}{4 r^2}
- \frac{k^2}{r^2}
+ \frac{2 m E - p_z^2}{\hbar^2}
\right)
U(r) = 0$$

With solution:

$$U(r) = c_1 \sqrt{r} J_k\left(\frac{\sqrt{\text{EE}} r}{\hbar}\right)+c_2 \sqrt{r} Y_k\left(\frac{\sqrt{\text{EE}} r}{\hbar}\right)$$

\begin{itemize}
\item $J_k$: Bessel's function of first kind
\item $Y_k$: Bessel's function of second kind, singularity at origin
\end{itemize}

Looks kinda correct.



% https://en.wikipedia.org/wiki/Quantum_harmonic_oscillator#N-dimensional_harmonic_oscillator

\begin{comment}
Multiply all by $r^2$:

$$\left(
- \hbar^2 \left( r^2 \frac{\partial^2 R}{\partial r^2} + r \frac{\partial R}{\partial r} \right)
+ \hbar^2 k^2
+ \frac{1}{4} q^2 B^2 r^4
+ r^2 (p^2_z
- 2 m E
- q B \hbar k)
\right) R(r) = 0$$
\end{comment}

\begin{comment}

\subsubsection{Another gauge}
$$
A = \begin{pmatrix}
- B r \theta \\
0 \\
0
\end{pmatrix}
$$

$$\renewcommand\arraystretch{1.5}
(-i \hbar \nabla - q A) \psi =
\begin{pmatrix}
-i \hbar \frac{\partial \psi}{\partial r} + q B r \theta \psi \\
-i \hbar \frac{1}{r} \frac{\partial \psi}{\partial \theta}\\
-i \hbar \frac{\partial \psi}{\partial z}
\end{pmatrix}
$$


\begin{align*}
2 m H \psi
&= - \hbar^2 \frac{1}{r} \frac{\partial}{\partial r} \left( r \frac{\partial \psi}{\partial r}\right ) \\
&- \hbar^2 \frac{1}{r^2} \frac{\partial^2 \psi}{\partial \theta^2} \\
&- \hbar^2 \frac{\partial^2 \psi}{\partial z^2}
\end{align*}

Apparenly, this gauge is not nice one :(

\end{comment}

\subsection{Notes}
\subsubsection{Cylindrical coordinates}
$$\bm{r} = (r, \theta, z)$$


$$\hat {p_r} = - i \hbar \frac{\partial}{\partial r}$$

$$\hat{p_\theta} = - i \hbar \frac{1}{r} \frac{\partial}{\partial \theta}$$

$$\hat{p_z} = - i \hbar \frac{\partial}{\partial z}$$

$$\renewcommand\arraystretch{1.5}
\nabla f = 
\begin{pmatrix}
\frac{\partial f}{\partial r} \\
\frac{1}{r} \frac{\partial f}{\partial \theta}  \\
\frac{\partial f}{\partial z} 
\end{pmatrix}
$$


$$
\nabla \cdot F =
\frac{1}{r} \frac{\partial (r F_r)}{\partial r}
+ \frac{1}{r} \frac{\partial F_\theta}{\partial \theta}
+ \frac{\partial F_z}{\partial z}
$$


\end{document}